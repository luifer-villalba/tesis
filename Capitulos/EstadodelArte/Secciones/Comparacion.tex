\section{Comparación entre plataformas}
Una investigación para comprobar otros proyectos o productos con las mismas características propuestas, buscando innovación para el mercado es expuesta en la tabla \ref{relacion-sistemas}.

\begin{table}[H]
\centering
\resizebox{\columnwidth}{!}{%
	\begin{tabular}{lllccl}
		\toprule
		\multicolumn{3}{l}{Características}                                                & CurricUNET                       & CourseLeaf            & DECA         \\
		\midrule
		\multicolumn{3}{l}{Creación y versionamiento de competencias.}                     &                                  &                       &              \\
		\multicolumn{3}{l}{Creación y versionamiento de cursos.}                           & $\checkmark$                     & $\checkmark$          & $\checkmark$ \\
		\multicolumn{3}{l}{Creación y versionamiento de programas de estudio.}             & $\checkmark$                     & $\checkmark$          &              \\
		\multicolumn{3}{l}{Cumple los Estándares de códigos de California.} 			   & $\checkmark$                     &                       &              \\
		\multicolumn{3}{l}{Historial de versiones de competencias.}     			       & 			                      & 		              &  			 \\
		\multicolumn{3}{l}{Historial de versiones de cursos.}     			               & $\checkmark$                     & $\checkmark$          & $\checkmark$ \\
		\multicolumn{3}{l}{Historial de versiones de programas de estudio.}     		   & $\checkmark$                     &  			          & 			 \\
		\multicolumn{3}{l}{Reporte de Comparación entre versiones de cursos.}              & $\checkmark$                     &                       & $\checkmark$ \\
		\multicolumn{3}{l}{Soporta competencias de aprendizaje del estudiante.}            &                      			  &                       &              \\
		\multicolumn{3}{l}{Plantilla de flujo de trabajo customizable.}                    & $\checkmark$                     &                       &              \\
		\multicolumn{3}{l}{Permite asignar roles evaluadores en la aplicación.}            & $\checkmark$                     & $\checkmark$          &              \\
		\multicolumn{3}{l}{Permite asignar usuarios como colaboradores.}                   & $\checkmark$                     &                       &              \\
		\multicolumn{3}{l}{Sistema de alertas para colaboradores y evaluadores.}           & $\checkmark$                     & $\checkmark$          &              \\
		\multicolumn{3}{l}{Buzón de entrada para colaboradores y autoridades.} 			   & $\checkmark$                     &                       & $\checkmark$ \\
		\multicolumn{3}{l}{Soporte de correlatividades entre cursos.}                      & $\checkmark$ 					  &						  &              \\
		\multicolumn{3}{l}{Incluye un catálogo de cursos.}                   		   	   & $\checkmark$					  &	$\checkmark$		  & $\checkmark$ \\
		\multicolumn{3}{l}{Incluye un catálogo de programas de estudio.}                   & $\checkmark$					  &	            		  &              \\
		\multicolumn{3}{l}{Incluye un catálogo de competencias.}                   	       & 								  &						  & 			 \\
		\multicolumn{3}{l}{UX intuitiva y efectiva.}     			   					   &                                  & $\checkmark$          & $\checkmark$ \\
		\bottomrule
	\end{tabular}
}
\caption{Relación entre sistemas de gestión curricular.}
\label{relacion-sistemas}
\end{table}

Los puntos de la tabla \ref{relacion-sistemas} fueron elegidos a base de una encuesta a los clientes, donde se les preguntó cuáles son las necesidades que tiene una universidad a la hora de proceder al diseño de cursos y programas de estudio. Además, como funcionalidad adicional se les ofreció la capacidad de soportar las competencias de sus universidades, y como se obtuvo una respuesta positiva de su parte se decidió agregar como un punto más en la tabla, para corroborar luego con el análisis de las diferentes herramientas si no soportaban dichas funcionalidades. Fueron tres los CMS que existen que cumplen con las expectativas mínimas del mercado y son a las cuáles se les hizo el análisis.

De los CMS existentes, CurricUNET es el que más satisface las necesidades de las universidades, pero al no soportar competencias ni ser capaz de adaptarse a un AMS de las universidades comunitarias del estado de California impide la creación automatizada de entidades validadas en un solo proceso.
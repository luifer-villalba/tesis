\section{Relevancia del módulo curricular}
La importancia del módulo reside en la posibilidad de automatizar formularios y procesos que requieren la participación de personas ajenas al flujo de trabajo, para iniciar y validar propuestas de creación o revisión de cursos y programas. Sin embargo, hay alternativas que buscan solucionar la misma problemática pero no existe alternativa que pueda soportar el uso de competencias ni que pueda comunicarse con un sistema de gestión de evaluación basadas en competencias.

Como se hablo en la sección \ref{procesoCurricular} de proceso curricular; una vez finalizado el proceso de diseño y revisión curricular de parte de las oficina, se procede a publicar la nueva competencia, curso, o programa para que se puedan cada universidad tenga la información necesaria para ir cargando dicha información en sus correspondientes sistemas, en el caso de las universidades comunitarias del estado de California se utilizan los AMS para gestionar y evaluar las competencias de sus estudiantes. El proceso de registro de las nuevas entidades en los AMS es individual; eso quiere decir que un encargado del AMS debe encargarse de cargar uno por uno las nuevas entidades aprobadas y publicadas por el comité curricular.

En la propuesta de solución del capítulo \ref{capitulo5} hablaremos de como el proyecto final busca solucionar la problemática y unir los procesos de los cuales hablamos en el párrafo anterior.
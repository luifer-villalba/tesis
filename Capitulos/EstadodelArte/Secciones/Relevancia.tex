\section{Relevancia del módulo curricular}
Frecuentemente, descrito como complejo e ineficaz, los métodos tradicionales basados en papel de gestión de programas de estudio proporcionan una visibilidad limitada, lo que resulta en una visión restringida de las etapas implicadas en la creación, modificación y aprobación de planes de estudio. Además, busca eliminar esta complejidad al acelerar el desarrollo curricular y el proceso de aprobación.

La importancia del módulo reside en la posibilidad de automatizar formularios y procesos que requieren la participación de personas ajenas al flujo de trabajo, para iniciar y validar propuestas de creación o revisión de cursos y programas. Sin embargo, hay alternativas que buscan solucionar la misma problemática pero no existe alternativa que pueda soportar el uso de competencias ni que pueda comunicarse con un sistema de gestión de evaluación basadas en competencias.

Como se habló en la sección \ref{procesoCurricular} de proceso curricular; una vez finalizado el proceso de diseño y revisión curricular de parte de las oficinas, se procede a publicar la nueva competencia, curso, o programa para que cada universidad tenga la información necesaria para ir cargando la misma en sus correspondientes sistemas. 

En el caso de las universidades comunitarias del estado de California, se utilizan los AMS para gestionar y evaluar las competencias de sus estudiantes. El proceso de registro de las nuevas entidades en los AMS es individual; eso quiere decir que un encargado del AMS debe encargarse de cargar uno por uno las nuevas entidades aprobadas y publicadas por el comité curricular.

En la propuesta de solución del capítulo \ref{capitulo5} hablaremos de como el proyecto final busca solucionar la problemática y unir los procesos de los cuales hablamos en el párrafo anterior.
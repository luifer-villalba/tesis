\section{Reportes y notificaciones para flujos de trabajo}
\begin{table}[H]
\centering
\caption{Historias de usuario para los reportes y notificaciones de versiones de cursos}
\label{epic:9}
\resizebox{\columnwidth}{!}{%
\begin{tabular}{@{}lllll@{}}
\toprule
Historias de usuario                     			& HE & HC & PH & Sprints \\ \midrule
Notificaciones para las partes del flujo de trabajo & 58 & 60 &  5 &  1 \\
Reporte de esquemas de curso 						& 88 & 91 &  8 &  2 \\ \bottomrule
\end{tabular}
}
\end{table}

\subsection{Notificaciones para las partes del flujo de trabajo}
La historia de usuario tiene como descripción lo siguiente \enquote{\textit{Como especialista curricular, me gustaría que mi equipo de diseño y revisión de flujos de trabajo reciban las notificaciones cuando tengan trabajos pendientes (y alertas cuando este retrasado), para que pueda manejar mejor mis procesos curriculares}}.

Como criterios de aceptación se encuentran los siguientes:
\begin{itemize}
	\item Establecer notificaciones cuando las partes del flujo de trabajo son asignadas a los roles de las personas.
	\item Establecer notificaciones de alerta a asignaciones de partes y etapas. Por ejemplo, 5 días después de su asignación.
	\item Mandar notificaciones por mail.
\end{itemize}

Algunas de las tareas identificadas en la planificación de las iteraciones eran los siguientes:
\begin{itemize}
	\item Diseñar e implementar nuevos modelos de datos que permitan soportar el uso de roles para creadores y editores de partes.
	\item Actualizar el sistema de notificaciones del AMS.
	\item Diseñar e implementar la página de configuración de notificaciones.
\end{itemize}

La historia fue finalizada en tres iteraciones con una cantidad de 60 horas cargadas en el sistema.

\subsection{Reporte de esquemas de curso}
La historia de usuario tiene como descripción lo siguiente \enquote{\textit{Como especialista curricular, me gustaría ser capaz de hacer reportes de registro de esquemas de curso, para que pueda de esta manera ser compartidas por los diferentes colaboradores}}.

Como criterios de aceptación se encuentran los siguientes:
\begin{itemize}
	\item Reporte diseñado por el equipo de diseño curricular.
	\item Incluir competencias es opcional para el reporte.
	\item Incluir alineación de competencias es opcional para el reporte.
	\item Formatos en DOC, PDF o HTML (no Excel).
	\item Se puede ejecutar desde la lista de cursos o de la lista de reportes.
	\item Cuando se corre desde la lista de reportes se pueden elegir uno o más cursos.
\end{itemize}

Algunas de las tareas identificadas en la planificación de las iteraciones eran los siguientes:
\begin{itemize}
	\item Diseñar e implementar el reporte.
	\item Implementar los métodos que acceden a la base de datos.
	\item Diseñar e implementar la página de generación de reportes.
\end{itemize}

La historia fue finalizada en tres iteraciones con una cantidad de 111 horas cargadas en el sistema.
\section{Versionamiento de evaluaciones}
La épica fue creada para abarcar todos los cambios que requerían las evaluaciones en sus diferentes formas de versionamiento. Entre ellas, se encuentra el versionamiento encadenado de las evaluaciones debido al flujo terminado de versionamiento de competencias. 

En caso de que existan evaluaciones ya creadas para las secciones de los cursos y en dicho curso se generaba una nueva versión de alguna de sus competencias, el sistema debía versionar las evaluaciones que utilizaba la versión anterior de dicha competencia con la versión actual para el periodo.

Como trabajo futuro se cuenta con el desarrollo de flujos de trabajo para las evaluaciones, ya que el versionamiento que ahora cuenta el sistema es automático y no a través de formularios web.

\begin{table}[H]
\centering
\resizebox{\columnwidth}{!}{%
\begin{tabular}{@{}lllll@{}}
\toprule
Historias de usuario           & HE & HC & PH &  Sprints \\ \midrule
Versionamiento de evaluaciones & 102 & 87 & 8 &  2 \\ \bottomrule
\end{tabular}
}
\caption{Historias de usuario para el versionamiento encadenado de evaluaciones debido al versionamiento de competencias}
\label{epic:5}
\end{table}

\subsection{Versionamiento de evaluaciones} \label{versionamiento_encadenado}
La historia de versionamiento de evaluaciones tiene como propósito permitir el versionamiento automático de evaluaciones existentes que utilizar competencias del sistema. Por ejemplo, en caso de que una evaluación hecha por un profesor tenga una nueva versión en el nuevo periodo de su sección, el sistema versiona la evaluación para ese periodo obteniendo las competencias actuales.

La historia de usuario tiene como descripción: \enquote{\textit{Como usuario del módulo curricular, me gustaría ser capaz de versionar mis evaluaciones, para que se observen los cambios a través del tiempo. Y que la interfaz y los reportes sigan presentando datos para los diseños históricos}}.

Algunas de las tareas fueron las siguientes:
\begin{itemize}
	\item Adaptar versionamiento para el modelo de datos de las evaluaciones.
	\item Migrar datos de los usuarios para que soporten versionamiento de las mismas.
	\item Actualizar el selector de evaluaciones de los profesores para que puedan seleccionar evaluaciones actuales.
	\item Actualizar el widget de profesores que utilizan las evaluaciones como datos.
\end{itemize}

Esta historia de usuario fue realizada en una iteración con un total de 87 horas cargadas en el sistema.

\section{Versionamiento encadenado de evaluaciones debido al versionamiento de competencias}
\begin{table}[H]
\centering
\caption{Historias de usuario para el versionamiento encadenado de evaluaciones debido al versionamiento de competencias}
\label{epic:5}
\resizebox{\columnwidth}{!}{%
\begin{tabular}{@{}llllll@{}}
\toprule
Historias de usuario           & HE & HC & PH & PA & CS \\ \midrule
Versionamiento de evaluaciones & 102 & 87 & 8 &  3 &  2 \\ \bottomrule
\end{tabular}
}
\end{table}

\subsection{Versionamiento de evaluaciones}
La historia de versionamiento de evaluaciones tiene como propósito permitir el versionamiento automático de evaluaciones existentes que utilizar competencias del sistema. Por ejemplo, en caso de que una evaluación hecha por un profesor tenga una nueva versión en el nuevo periodo de su sección, el sistema versiona la evaluación para ese periodo obteniendo las competencias actuales.

La historia de usuario tiene como descripción lo siguiente \enquote{\textit{Como usuario del módulo curricular, me gustaría ser capaz de versionar mis evaluaciones, para que se observen los cambios a través del tiempo. Y que la interfaz y los reportes sigan presentando datos para los diseños históricos}}.

Algunas de las tareas fueron las siguientes:
\begin{itemize}
	\item Adaptar versionamiento para el modelo de datos de las evaluaciones.
	\item Migrar datos de los usuarios para que soporten versionamiento de las mismas.
	\item Actualizar el selector de evaluaciones de los profesores para que puedan seleccionar evaluaciones actuales.
	\item Actualizar el widget de profesores que utilizan las evaluaciones como datos.
\end{itemize}

Esta historia de usuario fue realizada en una iteración con un total de 87 horas cargadas en el sistema.

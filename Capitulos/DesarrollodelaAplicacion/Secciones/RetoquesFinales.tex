\section{Retoques finales}
\begin{table}[H]
\centering
\caption{Historias de usuario para los retoques finales}
\label{epic:11}
\resizebox{\columnwidth}{!}{%
\begin{tabular}{@{}llllll@{}}
Historias de usuario                                                           & HE & HC & PH & PA & CS \\ \hline
Retoques finales para el flujo de trabajo de cursos                            & 86 & 162&  8 &  3 &  1 \\ \hline
\end{tabular}
}
\end{table}

\subsection{Retoques finales para el flujo de trabajo de cursos}
La historia de usuario tiene como descripción lo siguiente \enquote{\textit{Como especialista curricular, me gustaría poder visualizar y editar todos los campos requeridos por el PCAH}}.

Como criterios de aceptación se encuentran los siguientes:
\begin{itemize}
	\item El flujo de trabajo soporta todos los campos del PCAH.
	\item El flujo de trabajo soporta los campos almacenados adicionales.
\end{itemize}

Algunas de las tareas identificadas en la planificación de las iteraciones eran los siguientes:
\begin{itemize}
	\item Diseñar mockups para todas las partes del flujo de trabajo.
	\item Actualizar las pantallas de las partes del flujo de trabajo.
	\item Actualizar los procedimientos de guardado y versionamiento de entidades.
\end{itemize}

La historia fue finalizada en tres iteraciones con una cantidad de 162 horas cargadas en el sistema.
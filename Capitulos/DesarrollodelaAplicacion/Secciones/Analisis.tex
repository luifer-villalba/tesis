\section{Conformación del equipo de trabajo} \label{analisis_herramientas}
La primera etapa del proyecto fue realizar una encuesta a los posibles integrantes del nuevo equipo de desarrollo donde se relevaron las capacidades adquiridas en cuanto a lenguajes de programación y tecnologías utilizadas, como así también de los conocimientos de dominio de la aplicación.

Dicha encuesta tiene como propósito permitir una mejor organización de los miembros de equipos y de esta manera una distribución eficaz de conocimientos y dominio de la aplicación para resolver las diferentes problemáticas que podría afectar al módulo curricular.

Una vez formado lo que sería el equipo de desarrollo se procedió a hacer análisis de las herramientas que podrían resolver la problemática entre ellas las que eran tomadas como requerimientos no funcionales para el módulo de gestión curricular del capítulo \ref{reqnofuncional}.
\section{Soporte de versionamiento en el AMS}
\subsection{Interfaz de alineamiento de códigos TOP/CIP}
La historia de usuario tiene como descripción lo siguiente \enquote{\textit{Como encargado del sistema de gestión de evaluaciones, me gustaría ser capaz de mantener alineados los codigos TOP a los códigos federales CIP y de esta manera no tener que depender de los administradores para manejar o interpretar estos datos}}.

Como criterios de aceptación se encuentran los siguientes:
\begin{itemize}
	\item Diseñar e implementar una página CRUD para códigos TOP/CIP.
	\item Permitir alinear los códigos.
\end{itemize}

Algunas de las tareas identificadas en la planificación de las iteraciones eran los siguientes:
\begin{itemize}
	\item Diseñar mockups para la nueva página.
	\item Diseñar e implementar el modelo de datos que soporte la funcionalidad.
	\item Desarrollar la página y los métodos de guardado.
\end{itemize}

La historia fue finalizada en una iteración con una cantidad de 36 horas cargadas en el sistema.

\subsection{Renombrar/Reorganizar pestañas para una mejor apariencia del módulo curricular}
La historia de usuario tiene como descripción lo siguiente \enquote{\textit{Como presidente del comité curricular, me gustaría ver las pestañas que están enfocadas a mi trabajo para que pueda navegar y manejar mi tiempo en la aplicación}}.

Como criterios de aceptación se encuentran los siguientes:
\begin{itemize}
	\item Crear pestaña curricular.
	\item Identificar y esconder pestañas no relevantes para el rol.
\end{itemize}

Algunas de las tareas identificadas en la planificación de las iteraciones eran los siguientes:
\begin{itemize}
	\item Diseño de pestañas.
	\item Implementación de pestañas y modificaciones de espacio.
\end{itemize}

La historia fue finalizada en una iteración con una cantidad de 44 horas cargadas en el sistema.


\subsection{Lista curricular mejorada para cursos y programas}
La historia de usuario tiene como descripción lo siguiente \enquote{\textit{Como presidente curricular o miembro del plantel de profesores, me gustaría ser capaz de ordenar/filtrar/visualizar mis cursos y programas, para que de esta manera pueda ver que es importante para mi para luego tomar la acción correspondiente desde esta vista y mi trabajo pueda ser realizado de manera eficiente y con buena visibilidad}}.

Como criterios de aceptación se encuentran los siguientes:
\begin{itemize}
	\item Filtrar por estado de flujo de trabajo, fecha de inicio, fecha de fin y otros a ser designados por el equipo de validación.
	\item Ser capaz de tomar acciones desde la lista (mirar reporte de esquema, empezar una revisión, iniciar el flujo, otros).
	\item Columnas configurables para la vista (resultados de curso, semestre inicial, estado de flujo de trabajo, perfomance de competencia, etc.).
\end{itemize}

Algunas de las tareas identificadas en la planificación de las iteraciones eran los siguientes:
\begin{itemize}
	\item Diseñar mockups para la nueva página.
	\item Diseñar e implementar el modelo de datos que soporte la funcionalidad.
	\item Implementar los filtros.
	\item Implementar página para las columnas configurables.
	\item Actualizar lista.
	\item Actualizar el reporte de esquema de cursos para que soporte múltiples cursos.
\end{itemize}

La historia fue finalizada en dos iteraciones con una cantidad de 135 horas cargadas en el sistema.
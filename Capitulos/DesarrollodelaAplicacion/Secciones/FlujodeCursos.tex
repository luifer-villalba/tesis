\section{Flujo de trabajo para el versionamiento de cursos}
\subsection{Course Versioning}
Esta historia de usuario se inició en el sprint 47, se estimó terminar en un sprint, pero debido a la cantidad de partes que suponía cambios se utilizaron dos sprints para terminar la historia. Se estimó que la historia era de unos 13 puntos, se finalizó en el sprint 49 con un total de 96hs cargadas en el JIRA.

Esta historia de usuario tenía como descripción los siguiente “Como coordinador de eLumen, me gustaría poder hacer una versión de mi plan de curso para que pueda realizar un seguimiento de los cambios para cosas como la revisión de programas y los acuerdos de articulación y transferencia en eLumen.”.

Algunas de las tareas realizadas en la historia fueron las siguientes:

\begin{itemize}
	\item Buscar técnicas y herramientas de versionamiento para verificar posibles implementaciones parecidas para implementar.
	\item Luego de buscar algunas técnicas de versionamiento, diseñar una posible solución a la problemática.
	\item Implementar cambios en la base de datos mediante scripts en el proyecto.
	\item Actualizar clases de Java existentes en el proyecto de cursos.
	\item Implementar la solución para el flujo de creación de cursos de Curriculum, conocida como Workflow en inglés. Además, incluir nuevas clases de Java para las mismas.
	\item Actualizar la creación de curso con los cambios aplicados mediante scripts de base de datos.
	\item Adaptar la relación de cursos y competencias para que soporte el versionamiento de los mismos.
	\item Actualizar la lista de competencias por cursos.
\end{itemize}

\subsection{Course Detail: Course Cover Info}
Esta historia de usuario tiene como propósito de diseñar páginas que permitan al usuario completar la información básica de curso que buscan diseñar.

Tiene la siguiente descripción: “Como miembro del comité de Curriculum, me gustaría ser capaz de administrar la página de información básica de cursos, para que no tenga que buscar por documentos a la hora de crear o versionar cursos”.

\begin{itemize}
	\item Diseñar un modelo de datos que soporte el nuevo paso de información de curso.
	\item Adaptar tablas existentes y crear clases nuevas para las nuevas entidades de base de datos.
	\item Actualizar la plantilla de creación de Workflow para que soporte el nuevo paso.
	\item Actualizar el visualizador de Workflow.
	\item Diseño de pruebas automatizadas.
\end{itemize}

La historia fue estimada con 5 puntos de historia y se terminó en un sprint con un total de 53hs de desarrollo.

\subsection{Course Details: Units and Hours}
En esta historia se desarrolló un nuevo paso para el desarrollo de Workflow, en la cual el que inició el mismo va a poder detallar las horas y unidades que requiere el curso o cree que se va a requerir.

La organización proveyó de algunas muestras de cómo debería ser la página y era un criterio de aceptación de parte del ticket que siga el modelo de la misma.

La historia tenía la siguiente descripción: “Como aprobador de Curriculum de eLumen, me gustaría tener una página de horas y métricas para que pueda conseguir información básica sobre mi curso en eLumen”.

La historia a desarrollar se dividió entre miembros del equipo de desarrollo en las siguientes tareas:
\begin{itemize}
	\item Crear scripts en la base de datos para adaptar el modelo de datos para que soporte los nuevos campos de curso.
	\item Crear y/o editar las clases de las entidades de Java que utiliza o utilizará la aplicación.
	\item Actualizar la plantilla de workflows.
	\item Actualizar el visualizador de workflows.
	\item Pruebas de funcionalidad.
\end{itemize}

La historia fue estimada con 3 puntos de historia y se terminó en un sprint con un total de 66hs de desarrollo.

\subsection{Curriculum: Course Specifications}
En esta historia de usuario se desarrolló un nuevo paso para el Workflow, el cual era un paso de tipo formulario en la cual el que inició el Workflow puede agregar objetivos, información acerca de los métodos de evaluación de la materia, algunos equipos requeridos y libros que se necesitará en el curso.

La organización proporcionó de modelos de pantallas para el nuevo paso y era un criterio de aceptación de parte del ticket que siga el mismo formato.

La historia de usuario proporciona la siguiente descripción: “Como especialista de Curriculum de eLumen, me gustaría ser capaz de agregar o editar especificaciones de curso como parte del Workflow de creación y/o versionamiento de mi curso, con el objetivo de no hacerlo en papel.”

Las tareas fueron separadas y desarrolladas por los desarrolladores y eran las siguientes:
\begin{itemize}
	\item Crear scripts de base de datos para que soporte el nuevo formato de cursos.
	\item Crear y/o editar clases de Java para las nuevas entidades de la base de datos.
	\item Actualizar la página de creación y/o edición de plantillas de Workflow para que soporte el nuevo paso.
	\item Actualizar el visualizador de Workflow.
	\item Actualizar los servicios de guardado para creación y versionamiento de cursos y workflows.
	\item Actualizar el servicio de aprobación de Workflow.
	\item Test de la historia.
\end{itemize}

La historia fue terminada en el sprint 50 con 5 puntos de historia de usuario y 40hs de desarrollo cargadas en el sistema

\subsection{Prereq \& Entrance Skills}
Esta historia tiene como objetivo que el usuario pueda colocar una lista de cursos como pre-requisitos, co-requisitos, anti-requisitos y recomendaciones para su nuevo curso. Además de ciertas capacidades que el alumno debe tener como requisito para tomar el curso.

Para entrar un poco en contexto de la historia vamos a definir cuáles son los tipos de requisitos que puede tener un curso:

\begin{itemize}
	\item \textbf{Pre-requisito:} es un tipo de requisito que impide al usuario tomar o cursar un curso sin haber pasado antes del curso que está como pre-requisito.
	\item \textbf{Co-requisito:} es un tipo de requisito impide al usuario tomar o cursar un curso si no toma también el curso que tiene como co-requisito.
	\item \textbf{Anti-requisito:} es un requisito impide al usuario tomar o cursar un curso si ya curso o va a cursar un curso que tiene como anti-requisito.
	\item \textbf{Recomendación:} es una recomendación de parte del sistema que curso tomar para aprovechar mejor la materia o curso. Es opcional.
\end{itemize}

La historia de usuario tiene como descripción: “Como especialista de Curriculum de eLumen, me gustaría ser capaz de introducir requisitos para cursos y capacidades de entrada en la creación o revisión de Workflow, para que podamos seguir durante su desarrollo y aprobación”.

La historia fue dividida en partes para que los desarrolladores puedan trabajar en partes independientes durante el proceso de la misma, y eran las siguientes:
\begin{itemize}
	\item Crear scripts para el nuevo modelo de datos.
	\item Generar o editar clases de entidades para el nuevo modelo de datos.
	\item Actualizar la plantilla de Workflow para que soporte un nuevo paso.
	\item Actualizar el visualizador de Workflow.
	\item Actualizar los servicios de guardado y aprobación.
	\item Pruebas de funcionalidad.
\end{itemize}

La historia de usuario fue cerrada en el sprint 50 con 3 puntos de historia y 44hs de desarrollo cargadas.

\subsection{Course Creation/Approval Review}
La historia de usuario tenía como criterio de aceptación los siguientes puntos:
\begin{itemize}
	\item Las páginas para revisar los Workflows tienen una región de retroalimentación o feedback debajo de cada paso, con la opción de ocultar y mostrar, para que el usuario que está revisando el Workflow en desarrollo pueda dejar comentarios al encargado de la creación o versionamiento del curso.
	\item La UI tiene elementos de status que indican que cierta parte es “new” o nueva, “approved” o aprobada y “rejected” o rechazada.
	\item Los pasos tienen regiones que permiten aceptar o rechazar los campos propuestos por los desarrolladores del curso. Por lo tanto, deben tener elementos de UI que indiquen al usuario que puede aprobar o rechazar cada parte.
\end{itemize}
Además de los criterios de aceptación, había que volver a actualizar el buzón de entrada para que acepten los cambios que tiene la historia de usuario. Debido a que más de una persona puede revisar el Workflow y podría trancar el proceso si es que no se le notifica debidamente que hay nuevos cambios que revisar.

Algunas de las tareas descompuestas de la historia de usuario son las siguientes:
\begin{itemize}
	\item Actualizar el visualizador de Workflow para que pueda soportar la nueva característica de aprobación o rechazo de cada parte.
	\item Actualizar el buzón de entrada de Cursos.
	\item Pruebas de funcionamiento.
\end{itemize}

La historia se cerró en el sprint 50 con 3 puntos de historia y 68hs cargadas de desarrollo

\subsection{Learning Outcomes}
Esta historia tiene como propósito de crear o versionar competencias para el curso a ser creado o versionado. 

La organización proporcionó de modelos de pantallas para el nuevo paso y era un criterio de aceptación de parte del ticket que siga el mismo formato.

La historia de usuario tiene como descripción: “Como especialista de Curriculum de eLumen, me gustaría ser capaz de articular las competencias de mi nuevo curso”.

Como criterio de aceptación de la historia fue la de agregar el Workflow de competencias en el Workflow de cursos. Algunas de las tareas de la historia fueron:
\begin{itemize}
	\item Modificar la base de datos para que soporte el nuevo modelo de datos de las competencias dentro de Workflow de curso.
	\item Modificar o agregar clases de las entidades que van a ser usadas durante la historia.
	\item Actualizar la plantilla de Workflow para que soporte el nuevo paso para la creación o versionamiento de competencias.
	\item Actualizar el visualizador de Workflow para que soporte el nuevo paso de competencias.
	\item Actualizar los servicios de guardado y de versionamiento de cursos y competencias.
	\item Pruebas de nuevas funcionalidades.
\end{itemize}

La historia fue cerrada en el sprint 51 e iniciada en el sprint 50 con un total de 76hs cargadas en el sistema.

\subsection{Curriculum: Assign Classification Codes}
La siguiente historia tiene como propósito la de asignar Classification Codes a los cursos.

Para entrar en contexto, habría que definir primero que son los Taxonomy of Programs (TOP) o taxonomía de programas en español.

La taxonomía de programas o TOP es un sistema numérico de códigos usados a nivel de Estado para recolectar y reportar información en cursos y programas, en diferentes instituciones educativas [sacado de TOP manual] por todo el Estado. 

TOP ha sido diseñado para agregar información acerca de los programas. Sin embargo, un código TOP debe ser asignado a cada curso del sistema. Aunque el TOP no contiene tantas opciones específicas como lo haría un sistema diseñado para cursos, a cada curso se le debe dar el código TOP que se aproxima a describir el contenido del curso.

Algunos usos a los códigos TOP:
\begin{itemize}
	\item En el inventario de programas aprobados y rechazados, para tener información que tipos de cursos y programas son ofrecidas por el Estado.
	\item En bases de datos de administración de información, para recolectar y reportar información en logros estudiantiles (licenciaturas y certificados) en ciertos programas.
	\item En contabilidad vocacional estudiantil, para reportes de compleción de programas y cursos de ciertos programas vocacionales.
\end{itemize}

La descripción de la historia fue la siguiente: “Como miembro del comité curricular, me gustaría ser capaz de asignar a mis cursos de códigos de clasificación como parte de la aprobación de mis workflows para asegurar que estén correctos, como esto es motivo de rechazo en la oficina del canciller del Estado”.

Algunas de las tareas fueron las siguientes:
\begin{itemize}
	\item Diseño del nuevo modelo y creación de scripts de base de datos, donde se debían generar tablas para cada nueva entidad del modelo de datos ajustado para las taxonomías de programas. Además, cargar todos los datos de códigos de cursos existentes para el estado de California.
	\item Creación de clases Java de las nuevas entidades de la base de datos.
	\item Crear páginas CRUD de las nuevas tablas (disciplina, sub-disciplina, campo).
	\item Diseño e implementación de la nueva página de asignación de códigos de clasificación para los cursos en proceso de diseño.
	\item Hacer servicios para cada una de las nuevas páginas.
	\item Pruebas de funcionalidad.
\end{itemize}

La historia de usuario, con 5 puntos de historia, fue terminada con un total de 80 horas en el sprint 51.
\section{Flujo de trabajo para el versionamiento de competencias}

% Please add the following required packages to your document preamble:
% \usepackage{booktabs}
\begin{table}[]
\centering
\caption{Historias de usuario para el flujo de trabajo para el versionamiento de competencias}
\label{epic:3}
\begin{tabular}{@{}llllll@{}}
\toprule
Historias de usuario                & HE                   & HC                   & PH                   & PA                   & CS                   \\ \midrule
Versionamiento de competencias      & 76 & 87 & 8 & 3 & 1 \\ 
Flujo de trabajo simple             & 78 & 78 & 8 & 5 & 1 \\
Aprobar pasos del flujo de trabajo  & 44 & 44 & 5 & 2 & 1 \\
Rechazar pasos del flujo de trabajo & 52 & 53 & 5 & 3 & 1 \\ \bottomrule
\end{tabular}
\end{table}

\subsection{Versionamiento de competencias}
Esta historia de usuario tenía como descripción los siguiente \enquote{\textit{Como encargado del sistema de gestión de competencias, me gustaría ser capaz de versionar competencias con la finalidad de que se puedan redefinir competencias con el paso del tiempo, sin perder datos de corrección de las mismas}}.

Algunas tareas que se definieron en la historia de usuario son las siguientes:
\begin{itemize}
	\item Investigar y diseñar el versionamiento de las competencias. La misma fue desarrollada de manera en que toda competencia versionada apunta al origen y el origen se apunta a sí mismo, de esta manera se puede saber la familia de versiones de una competencia.
	\item Actualizar todos esos lugares de la aplicación que listan las competencias, donde solamente deberían traer las competencias actuales.
	\item Manejar la distribución de competencias a periodos futuros, de manera que una competencia no pueda ser distribuida a periodos en las que no tiene validez.
	\item Diseñar y mantener pruebas automatizadas.
\end{itemize}

Fue desarrollado durante tres iteraciones con un total de 170 horas cargadas en el sistema, debido a la complejidad a la hora de migrar los datos ya existentes de todas las universidades y por la cantidad de servicios que debían ser modificados.

\subsection{Flujo de trabajo simple}
En la siguiente historia de usuario inicia el proceso de creación de flujos de trabajo donde las plantillas de los flujos de trabajo pueden ser creados, editados, y eliminados por el administrador encargado de la aplicación de cada universidad. Para esta historia se debe diseñar y desarrollar las plantillas de manera que el administrador pueda agregar los diferentes pasos del flujo de trabajo si así lo decide en el futuro. Inicialmente se considera un solo paso para la iteración inicial de desarrollo.

Esta historia de usuario tenía como descripción los siguiente \enquote{\textit{Como coordinador del AMS, me gustaría ser capaz de crear flujos de trabajo simples para administrar la aprobación de revisiones de competencias y que se pueda tener un mejor manejo de las creaciones y aprobaciones de las mismas en el campus}}.

La historia tiene los siguientes criterios de aceptación:
\begin{itemize}
	\item Diseñar e implementar plantillas de flujo de trabajos simples para creación y revisión de todos los niveles de competencias.
	\item Diseñar e implementar un flujo de trabajo simple sin aprobación por parte de evaluadores, donde el iniciador del flujo puede revisar y aprobar su propio formulario.
	\item La plantilla de flujo de trabajo simple debe soportar el uso de pasos personalizados.
	\item El que inició el flujo es el único que puede llenar los campos del formulario.
\end{itemize}

El usuario debe ser capaz de agregar pasos personalizados para la plantilla de flujo de trabajo de la institución. Estos pasos personalizados son pasos que puede diseñar el usuario, donde puede colocar una pregunta como título y por cada título tiene un campo que puede llenar el usuario. Por lo general, un paso personalizado puede tener una o más preguntas definida por el usuario.

En los mockups entregados para el desarrollo se contemplan trabajos futuros donde cada paso tiene que ser aprobado por un rol del AMS, donde cualquier usuario con dicho rol puede aprobar o rechazar el flujo de trabajo con solo rechazar uno de los pasos. 

Además, se da inicio al desarrollo de plantillas de flujos de trabajo con las competencias, donde se podía asignar un tipo de flujo para cada plantilla ya sea de creación o versionamiento de los diferentes niveles de competencias.

Como las plantillas era una funcionalidad conocida y utilizada en otra parte de la aplicación, se imitó el comportamiento de la misma utilizando las mismas tablas para el almacenamiento de los datos en la base de datos relacional como requerimiento no funcional de la organización. Se diseñaron las nuevas pantallas con la definición de las plantillas de flujo de trabajo y también la pantalla para listar las mismas. En la misma el usuario administrador puede crear, editar si aún no ha sido usada, eliminar, y clonar plantillas.

La historia de usuario fue desarrollada durante una iteración con un total de 90 horas cargadas en el sistema.

\subsection{Aprobación de pasos completados de flujos de trabajo}

Luego de la historia en la que se diseñaron las plantillas y fue desarrollado un flujo de trabajo simple inicial para creación o versionamiento de competencias, el siguiente paso es que un usuario designado desde la plantilla pueda iniciar y otro pueda aprobar el proceso de creación o revisión de competencias de cualquier nivel.

Esta historia de usuario tenía como descripción los siguiente \enquote{\textit{Como evaluador de un flujo de trabajo, me gustaría un simple proceso paso por paso en el que pueda revisar y/o aprobar competencias de manera sencilla e intuitiva}}.

La historia de usuario tiene los siguientes criterios de aceptación:
\begin{itemize}
	\item Soporte de asignaciones de tareas de creación y revisión por roles del AMS en las plantillas de flujos de trabajo.
	\item Diseño e implementación de vista de revisión para el flujo de trabajo.
\end{itemize}

Cada paso del flujo de trabajo debe estar terminado para que pase a la etapa de revisión por parte de los encargados. Luego de enviar el formulario, cada rol debe hacer su revisión para que el sistema pueda agregar la nueva competencia.

Como en las reuniones de demostración de cada sprint se notaban ciertos aspectos de las historias de usuario que no llenaban las expectativas de los clientes, los desarrolladores decidieron diseñar maquetas de pantallas que mostraban el posible diseño de la página. Luego de recibir feedback de parte de los clientes, se empezaba a desarrollar las nuevas pantallas. Finalizando la historia de usuario con pruebas automatizadas.

La historia de usuario fue desarrollada durante una iteración en un periodo de tiempo de 44 horas cargadas en el sistema.

\subsection{Rechazar pasos completados del flujo de trabajo}
Esta historia de usuario tenía como descripción los siguiente \enquote{\textit{Como evaluador de flujos de trabajo, me gustaría ser capaz de rechazar partes de los mismos y poder dar feedback a partes que no cumplen con nuestros estándares}}.

\begin{itemize}
	\item Diseño e implementación de funcionalidad de rechazo de pasos en los flujos de trabajo.
	\item Diseño e implementación de funcionalidad de retroalimentación de parte de los evaluadores y encargados.
\end{itemize}

Esta funcionalidad tiene como propósito permitir a la persona que hace la revisión de los pasos rechazar y dejar feedback para que se puedan hacer los cambios correspondientes. Cuando se rechaza un paso, se rechaza el flujo de trabajo, y por lo tanto vuelven a estar activos los campos para que se hagan los cambios correspondientes.

El trabajo se inició la actualización del modelo de base de datos actual, luego de crear las clases correspondientes en el código para su utilización. Luego, se actualizaron las páginas donde el usuario puede aprobar los steps para que soporte rechazar pasos y poder así dejar algunos comentarios. 

La historia de usuario fue desarrollada durante una iteración en un periodo de tiempo de 53 horas cargadas en el sistema.

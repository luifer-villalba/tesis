\section{Buzón de entrada para evaluadores y colaboradores del flujo de trabajo}
\subsection{Workflow Queue Visibility or Inbox}
La siguiente historia tiene como propósito mostrar a cada usuario la lista de workflows pendientes que requiere de su aporte. Además de adaptar el nuevo buzón de entrada para otros rasgos de la aplicación como son las evaluaciones, los planes de acción y preguntas de parte del usuario a profesores.

La historia tiene como descripción: “Como aprobador de eLumen, me gustaría una vista unificada de los workflows que tengo que revisar – además de mis evaluaciones, planes de acción y mis preguntas a profesores – para que no vaya cazando workflows por la aplicación”

Las tareas de la historia de usuario eran las de crear la página que listen los workflows inicialmente, luego de ese hacer las pruebas correspondientes. Luego de que funcione la lista de workflows, agregar los planes de acción y RFI en la lista a la misma lista y volver a hacer las pruebas de funcionamiento. 

Se estimó con 5 puntos de historia y tuvo una duración de 2 sprints debido a inconvenientes en el camino con un total de 56hs de desarrollo.

\subsection{Notificaciones con soporte a etapas}
La historia de usuario tiene como descripción lo siguiente \enquote{Como presidente curricular, me gustaría que el equipo de diseño y revisión curricular reciban notificaciones cuando tengan alertas de deuda de trabajo (y alertas cuando pase el tiempo), para que se puedan manejar mejor de esa manera los procesos curriculares.}.

Como criterios de aceptación se encuentran los siguientes:
\begin{itemize}
	\item Establecer notificaciones cuando las partes del flujo de trabajo son asignadas a los roles de las personas.
	\item Establecer notificaciones de alerta a asignaciones de partes y etapas. Por ejemplo, 5 días después de su asignación.
	\item Mandar notificaciones por mail.
\end{itemize}

Algunas de las tareas identificadas en la planificación de las iteraciones eran los siguientes:
\begin{itemize}
	\item Diseñar e implementar nuevos modelos de datos que permitan soportar el uso de roles para creadores y editores de partes.
	\item Actualizar el sistema de notificaciones del AMS.
	\item Diseñar e implementar la página de configuración de notificaciones.
\end{itemize}

La historia fue finalizada en tres iteraciones con una cantidad de 60 horas cargadas en el sistema.
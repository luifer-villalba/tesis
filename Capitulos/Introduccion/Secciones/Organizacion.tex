\section{Organización del libro}
El documento se encuentra estructurado en 7 capítulos. 

En el capítulo \ref{capitulo2}, \enquote{Marco teórico}, el cual presenta el conjunto de escritos académicos tanto técnicos como educativos que constituyen los fundamentos utilizados en este trabajo de tesis.

En el capítulo \ref{capitulo3}, \enquote{Estado del arte}, realizamos un relevamiento y análisis de herramientas de informática educativa que intentan resolver problemas relacionados de forma cercana a los problemas planteados en éste trabajo de tesis.

En el capítulo \ref{capitulo4}, \enquote{Propuesta de solución}, se presenta en detalle la arquitectura del sistema planteada como solución, con un correspondiente estudio para la determinación de las herramientas para cada paso. Se expone además el flujo de datos y se presenta un diseño del mismo, donde queda gráficamente los pasos a seguir. Se incluyen también las técnicas de validación y evaluación que se deben utilizar.

En el capítulo \ref{capitulo5}, \enquote{Desarrollo de la aplicación}, se presentará el proceso de desarrollo del módulo de gestión de programas orientado a competencias. Listando las épicas y brindando una breve explicación de cada una de ellas.

En el capítulo \ref{capitulo6}, \enquote{Validación del desarrollo}, se encuentran los análisis y validaciones de resultados obtenidos. Además, se analiza el flujo de la validación de las historias de usuario como proceso sistemático, consistente y controlado de parte de los desarrolladores y de los expertos en educación encargados de validar el software desarrollado.

En el capítulo \ref{capitulo7}, \enquote{Conclusión y aportes}, se encuentran las conclusiones obtenidas, acompañado de un breve resumen de toda la investigación. Se exponen los aportes a la problemática, así como también se proponen posibles temas de investigación y trabajos futuros o complementarios, que puedan continuar a partir del presente trabajo.

\section{Dominio de la problemática}
El proyecto objetivo a desarrollarse utiliza como base una aplicación web AMS\footnote{de sus siglas en inglés, Assessment Management System, que significa en español sistema de gestión de evaluaciones.} que integrará un módulo de gestión curricular a la misma. Los AMS son utilizados por las universidades en Estados Unidos para evaluar las competencias adquiridas de los estudiantes durante el proceso de su carrera o grado. También se busca la ubicuidad en dispositivos inteligentes mediante la nube. Los dispositivos inteligentes que van a poder utilizar la aplicación se definirán durante el proceso de desarrollo de la misma.

En las aplicaciones dirigidas al ambiente educativo, el hecho de utilizar nuevas tecnologías no asegura una mejor UX\footnote{de sus siglas en inglés, User eXperience, que significa en español experiencia de usuario.}, es por eso que introduciremos durante el desarrollo de la aplicación el concepto de HCI\citep{lazar_research_2010}, donde ésta se encarga de establecer los patrones de diseño e interacción a seguir a la hora de construir cada uno de los componentes de la aplicación a ser desarrollada.

Un requisito no funcional que forma parte de la infraestructura del caso de estudio es la utilización de la arquitectura SaaS\footnote{de sus siglas en inglés, Software as a Service, que significa en español software como servicio.}, ya que provee servicios bajo un modelo de pagos de suscripción por las diferentes características que la misma ofrece.

Dentro de la metodología Ágil, se definirán épicas a desarrollarse durante el periodo de diseño de la aplicación y las historias de usuario que están contenidas en las mismas, para que el equipo de desarrollo pueda entregar valor de negocio del módulo integrado en iteraciones cortas de dos semanas.
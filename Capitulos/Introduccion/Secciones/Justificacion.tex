\section{Justificación}
En la actualidad, la tecnología forma parte de nuestra vida cotidiana donde el software ha avanzado con el paso del tiempo. La tendencia apunta a que las aplicaciones puedan ser accedidas en cualquier momento desde cualquier lugar, ya sea en un viaje de negocios o haciendo compras desde cualquier dispositivo con acceso a internet. 

Últimamente, de manera a facilitar el acceso a la aplicación por parte de los usuarios, las aplicaciones web se han vuelto populares porque estas se ejecutan en un navegador y no es necesario descargar ningún tipo de software adicional debido a que el peso principal de la aplicación se ejecuta remotamente\citep{net_app_architecture}.

Al mismo tiempo, en la educación han surgido avances tecnológicos aplicables donde se aprovecha la misma para potenciar el aprendizaje adquirido en los estudiantes. Con esto se busca idear nuevas técnicas de evaluación para impulsar un aprendizaje significativo en los alumnos, pero en muchos casos nos encontramos que la forma de evaluar los cursos o actividades no reflejan los conocimientos o capacidades de los alumnos. Para llenar este vacío surge la evaluación basada en competencias, la cual se enfoca en aquellas adquiridas por un estudiante en el proceso de un programa educativo\citep{kuh_knowing_2014}.  

Así, las aplicaciones de evaluación académica basadas en competencias han adquirido mucha importancia en los últimos años\citep{kuh_knowing_2014}. En dichas aplicaciones se buscan conocer las fortalezas y debilidades del estudiante de una manera modular, en comparación a los métodos cuantitativos de evaluación. Dichos aprendizajes y competencias son expresados por segmentos de estudios o actividades, mediante resultados esperados medibles a nivel institucional, de programa, grado, o de curso; expresados en calificaciones\citep{kuh_using_2015}.

Para las personas ajenas al entorno educativo, en específico aquellas que no participan directamente en el diseño de planes y programas, puede resultar un tanto complejo comprender el proceso del diseño curricular y reconocer la importancia de involucrar a todas aquellas autoridades que forman parte del mismo. Por esta razón, definir las bases teóricas que servirán de referencia para el diseño curricular, describir sus conceptos básicos, y distinguir sus elementos, es fundamental para el desarrollo del curso\citep{boyle_curriculum_2016}.

En el ámbito de las aplicaciones académicas basadas en competencias, si bien existen aplicaciones que abarquen el diseño y la emisión de planes de estudio por parte de los profesores o encargados de las universidades, además de su revisión y posterior aprobación por el comité curricular, no se ha encontrado durante el proceso de investigación una aplicación que integre todos estos a un sistema de gestión de evaluaciones basadas en competencias.
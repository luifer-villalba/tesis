\section{Diseño de la investigación}
El presente trabajo ilustra cómo caso de estudio la aplicación de la metodología ágil en la gestión de un proyecto de desarrollo de un módulo, integrado a un sistema de gestión y evaluación de competencias de una organización ubicada en los Estados Unidos. El trabajo a realizar fue propuesto por la organización que brinda el sistema de gestión de competencias, donde el mismo sirve como base al módulo de gestión curricular a desarrollarse.

El caso refleja de manera práctica se ha desenvuelto un proceso de desarrollo ágil en un diálogo con usuarios ubicados en diferentes localidades como parte del sistema de trabajo. En este proyecto los miembros ubicados en forma remota se encargan del diseño de las tareas llamadas historias de usuario y de la validación de las características entregadas.

Debido a que los requisitos de la aplicación a desarrollarse se irán esclareciendo acorde se vayan completando cada valor de negocio, se optó por la metodología ágil como técnica de gestión de desarrollo de software, puesto que es el enfoque más adecuado para poder encarar esta problemática.

La observación participante de parte del investigador es un paso inicial para el desarrollo del sistema en un nuevo equipo, donde se identifican y guían las relaciones con los informantes, lo ayuda a observar de manera y embebida la organización y dinámica del equipo y las prioridades de desarrollo. También, lo permite integrarse con los demás miembros del mismo y de esa manera le facilita el proceso de investigación, además de proveerle una cantidad de interrogantes a ser dilucidadas con los participantes \citep{erlandson_doing_1993}.

El primer paso es adaptarse a los cambios y a la metodología de trabajo del equipo de desarrollo. Dicho equipo se constituirá luego de a una encuesta previa de capacidades de todos los desarrolladores de la empresa, donde cada uno colocará sus conocimientos en herramientas o en partes del sistema de gestión de competencias, para así poder aprovechar las virtudes de cada persona. Acto seguido se procede al análisis de las herramientas a utilizarse para corroborar que cumplen con las exigencias del sistema a desarrollarse.

Entre algunas técnicas de recolección de información se utilizan las encuestas \citep{robson_real_2011}. Para este sistema un equipo en Estados Unidos utilizó la misma para proporcionar una visión general de los conocimientos generales del equipo de desarrollo, más información al respecto en la sección \ref{analisis_herramientas}.

Los casos de estudio con enfoque HCI\footnote{de sus siglas en inglés, Human-Computer Interaction, que significa en español interacción humano-computador.} tienen como meta la comprensión de problemas o situaciones mediante la interacción del ser humano con la computadora. Además, se busca una documentación descriptiva del sistema y de su proceso de desarrollo, que apunta a la evolución del modelo propuesto durante el diseño de la misma, finalmente, se brinda y analiza evidencia de que la herramienta haya sido utilizada de manera exitosa mediante demostraciones a los usuarios o validaciones por parte del mismo \citep{lazar_research_2010}. 

Por lo tanto, el caso de estudio propuesto es uno con enfoque HCI que utiliza la observación participante como método principal de recolección de información.
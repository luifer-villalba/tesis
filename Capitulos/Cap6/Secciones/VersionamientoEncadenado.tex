\section{Versionamiento encadenado de evaluaciones debido al versionamiento de competencias}
\subsection{Versioning for Assessments}
La historia de versionamiento de evaluaciones tiene como propósito permitir el versionamiento automático de evaluaciones existentes que utilizar competencias del sistema. Por ejemplo, en caso de que una evaluación hecha por un profesor tenga una nueva versión en el nuevo periodo de su sección, el sistema versiona la evaluación para ese periodo obteniendo las competencias actuales.

Esta historia de usuario se inició en el sprint 48 con un total de 13 puntos de historia, se finalizó en el sprint con un total de 87hs cargadas en el JIRA.

Esta historia de usuario tenía como descripción los siguiente “Como usuario de eLumen, me gustaría ser capaz de versionar mis evaluaciones, para que se observen los cambios a través del tiempo (y que la interfaz y los reportes sigan presentando datos para los diseños históricos)”.

Algunas de las tareas de la historia fueron las siguientes:
\begin{itemize}
	\item Adaptar versionamiento para el modelo de datos de las evaluaciones.
	\item Actualizar la biblioteca de evaluaciones de los usuarios para que soporte versionamiento de las mismas.
	\item Actualizar el selector de evaluaciones de los profesores, que puedan seleccionar evaluaciones actuales.
	\item Actualizar el widget de profesores que utilizan las evaluaciones como datos.
\end{itemize}

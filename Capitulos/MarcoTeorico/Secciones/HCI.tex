\section{Interacción humano-computador}
En HCI definen la funcionalidad y la usabilidad de los sistemas que se desarrollan, donde la funcionalidad de un sistema es definida por un conjunto de acciones o servicios que son proveídas a los usuarios, sin embargo, el valor de la funcionalidad es verificada cuando es eficientemente utilizada por el usuario\citep{shneiderman_designing_2010}. La usabilidad de un sistema con cierta funcionalidad es el rango y grado por el cual el mismo puede ser utilizada de manera eficiente y adecuada para cumplir ciertas metas para ciertos usuarios. La eficiencia de un sistema es alcanzada cuando se cumple un balance entre la usabilidad y la funcionalidad\citep{nielsen_usability_2010}.

HCI es un diseño que debe producir un ajuste entre el usuario, la máquina y los servicios requeridos con el fin de lograr un balance óptimo entre la calidad y la eficiencia de los servicios.

La definición de la estrategia de UI\footnote{de sus siglas en inglés, User Interface, que significa en español experiencia de usuario.} es importante para una mejor usabilidad del sistema, donde este proceso debería comenzar antes que el diseño y desarrollo de las aplicaciones. Es la visión de una solución que necesite ser verificado con potenciales usuarios que prueben que necesite el mercado\citep{levy_ux_2015}.
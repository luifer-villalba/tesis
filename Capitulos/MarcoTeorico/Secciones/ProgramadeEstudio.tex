\section{Programas de estudio}
En términos generales, se puede definir un programa de estudios como una herramienta educativa que regula y ordena el proceso de enseñanza-aprendizaje a desarrollar en una unidad de aprendizaje determinada, orientando las actividades que profesor y alumno han de llevar a cabo para el logro de los objetivos planteados en dicha unidad, en relación con los objetivos del plan de estudios, de tal manera que el egresado concluya su carrera con el perfil deseado. En pocas palabras, es un esquema organizado de los contenidos situados dentro de una determinada unidad de aprendizaje\citep{lalor_ensuring_2017}.

El termino “unidad de aprendizaje” sustituye al de “asignatura” o “materia” que evocan los tradicionales cursos unidisciplinarios, generalmente teóricos y sobrecargados de información. Un programa resume las características de la unidad de aprendizaje, su contenido mínimo obligatorio, y sus objetivos, principalmente.

En la primera etapa de diseño curricular, se requiere la elaboración de la propuesta de los programas para su aprobación de parte de las autoridades con la colaboración de las academias y, de ser necesario, con la asesoría de externos de la unidad académica.

\subsection{Proceso curricular}
\section{Competencias académicas y su evaluación}
La creciente profesionalización trajo al campo educativo elementos evaluativos tales como calidad, equidad, competitividad, eficiencia, y eficacia; junto con ellos surgieron las competencias, que pasaron a jugar papel importante en el contexto educativo. En la formación de profesionales, resalta la necesidad de reflexionar sobre los aprendizajes que se ofrecen en las instituciones educativas, las cuales deben servir al estudiante para ser útil a la sociedad, que es su entorno inmediato \citep{kuh_using_2015}. En otras palabras, la competencia es la capacidad de un buen desempeño en contextos complejos y auténticos. Se basa en la integración y activación de conocimientos, habilidades, destrezas, actitudes y valores.

De esa necesidad va surgiendo la idea de las evaluaciones orientadas a competencias. Cabe resaltar cómo se desenvuelve el aprendizaje basado en competencias usando aplicaciones como herramientas para la evaluación de estudiantes, mediante el análisis de los aportes que introduce la tecnología en este campo, que modifican significativamente las prácticas tradicionales\citep{carriveau_connecting_2016}.

Uno de los factores de motivación relevantes para el aprendizaje es la evaluación. Cada actividad ofrece a los estudiantes la oportunidad de conocer cuáles son sus resultados de aprendizaje en lo que se refiere al \enquote{qué} se ha aprendido y al \enquote{cómo} habría podido hacerse. Cualquier proceso de evaluación debería ser diseñado teniendo en cuenta este principio básico.

En un sistema de gestión de evaluaciones basado en competencias, los encargados hacen evaluaciones según las evidencias obtenidas de diversas actividades de aprendizaje, que definen si un estudiante alcanza o no los requisitos recogidos por un conjunto de indicadores en un determinado grado. Una evaluación por competencias asume que pueden establecerse indicadores posibles de alcanzar por los estudiantes, que diferentes actividades de evaluación pueden reflejar los mismos indicadores\citep{barrio_minton_evaluating_2016}.

La evaluación por competencias ofrece nuevas oportunidades a los estudiantes al generar entornos significativos de aprendizaje que acercan sus experiencias académicas al mundo profesional, y donde pueden desarrollar una serie de capacidades integradas y orientadas a la acción, con el objetivo de ser capaces de resolver problemas prácticos o enfrentarse a situaciones cotidianas \citep{carriveau_connecting_2016}.

Hoy día existen herramientas que ayudan al alumno a potenciar su aprendizaje y algunas de ellas son los sistemas de gestión de aprendizajes y los sistemas de gestión de evaluaciones basadas en competencias, cuyas diferencias se mostrarán a posteriori.
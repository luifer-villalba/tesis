\section{Diferencias entre LMS y AMS}
Por lo general, en las universidades norteamericanas utilizan los AMS y LMS para mejorar el proceso de evaluación de sus estudiantes. Sin embargo, para las personas que no están familiarizadas con estos sistemas pueden quedar ambiguas las diferencias entre estos sistemas de aprendizaje. Estas diferencias se encuentran sumarizadas y tabuladas en la tabla \ref{comparacion_ams_lms}.

\begin{table}[H]
\centering
\caption{Comparación de características entre plataformas AMS y LMS.}
  \label{comparacion_ams_lms}
  \resizebox{\columnwidth}{!}{%
	\begin{tabular}{lllcc}
		\toprule
		\multicolumn{3}{l}{Características}                                                                     & AMS       & LMS       \\
		\midrule
		\multicolumn{3}{l}{Soporte de evaluaciones a los estudiantes.}                                          & $\checkmark$         & $\checkmark$         \\
		\multicolumn{3}{l}{Soporte de evaluación colectiva.}                                                    & $\checkmark$         &           \\
		\multicolumn{3}{l}{Diferentes tipos de rúbricas para evaluaciones.}                                     & $\checkmark$         &           \\
		\multicolumn{3}{l}{Permite acceder a cursos realizados por terceros (aula virtual para el estudiante).} &           & $\checkmark$         \\
		\multicolumn{3}{l}{Permite la evaluación de desempeño estudiantil.}                                     & $\checkmark$         &           \\
		\multicolumn{3}{l}{Permite generar reportes de cursos, progresos, etc.}                                 & $\checkmark$         & $\checkmark$         \\
		\multicolumn{3}{l}{Soporte de presupuestos (solicitud de profesores, coordinadores, etc.).}             & $\checkmark$         &           \\
		\multicolumn{3}{l}{Soporta competencias de aprendizaje del estudiante o Student Learning Outcomes.}     & $\checkmark$         &           \\
		\multicolumn{3}{l}{Soporte de alineación de competencias.}                                              & $\checkmark$         &           \\
		\multicolumn{3}{l}{E-portfolio y evaluación de proyectos.}                                              & $\checkmark$         & $\checkmark$         \\
		\multicolumn{3}{l}{Soporte de comunicación entre estudiantes y profesores (foro).}                      &           & $\checkmark$         \\
		\multicolumn{3}{l}{Soporte de evaluación sumativa y formativa.}                                         & $\checkmark$         &           \\
		\multicolumn{3}{l}{Software distribuido y desarrollado libremente.}                                     &           & $\checkmark$         \\
		\bottomrule
	\end{tabular}
	}
\end{table}

Los LMS permiten al estudiante a una capacitación flexible a distancia sin limitaciones de horarios con costos reducidos. Además, como es una plataforma intuitiva, permite a las personas con nivel de conocimiento básico en informática un aprendizaje constante y actualizado a través de la interacción entre alumnos y profesores. En cambio, los AMS son más bien utilizados para evaluaciones de las competencias de los alumnos que como vínculo entre el alumno y el profesor mediante un aula virtual, y permiten la mejora continua del aprendizaje estudiantil.

Los LMS permiten la integración de competencias mediante herramientas externas, pero de una manera superficial en comparación a los AMS.
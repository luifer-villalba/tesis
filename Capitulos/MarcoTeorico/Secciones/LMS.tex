\section{Sistemas de gestión de aprendizajes}
Las plataformas LMS\footnote{de sus siglas en inglés, Learning Management System, que significa en español sistema de gestión de aprendizajes.} son espacios virtuales de aprendizaje orientados a facilitar la experiencia de aprendizaje a distancia, donde permite una mejor interacción de profesores y alumnos. También se pueden hacer evaluaciones, intercambiar archivos y participar en foros y chats, además de otras herramientas de interacción estudiante a profesor.

La centralización y automatización de la gestión del aprendizaje es una de las principales características de los LMS. La plataforma puede ser adaptada tanto a los planes de estudio de la institución como a los contenidos y estilo pedagógico de la misma.

El usuario se convierte en el protagonista de su propio aprendizaje a través del autoservicio y los servicios guiados por los tutores o profesores mediante la herramienta. Además, permite utilizar los cursos desarrollados por terceros, personalizando el contenido y reutilizando el conocimiento adquirido. Además, posee prestaciones y características que hacen que cada plataforma sea adecuada según los requerimientos y necesidades de los usuarios.

Los LMS que almacenan información de programas académicos no involucran el aspecto de resultados pedagógicos en sus estructuras de datos o procesos de aprobación. Esta aprobación se realiza a través de flujos de trabajo\citep{aalst_workflow_2004}.
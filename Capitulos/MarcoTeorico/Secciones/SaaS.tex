\section{Software as a service}
Software as a service, más conocida como SaaS, es un paradigma de entrega de software donde la misma se encuentra alojada por lo general en la nube y se entrega como servicio a través de Internet a un gran número de usuarios a través de un modelo de suscripción. Se trata de un modelo de entrega de negocio en el que tanto la aplicación y el alojamiento son gestionados y compartidos con varias empresas, que alquilan y utilizan los servicios de aplicaciones de forma centralizada\citep{gupta_software_2014}.

El software como servicio es equivalente a servicios de proveedores externos que manejan todo mantenimiento, personalización, actualización y cobro para los servicios que su cliente utiliza de manera mensual o anual. El proveedor se encarga de ofrecer el software basado en un conjunto de códigos y datos definidos junto a las diferentes configuraciones para los diferentes clientes. Los suscriptores al servicio acceden a la aplicación con la sensación de que son los únicos usuarios de la aplicación. Sin embargo, los cambios de configuraciones como cambios de datos, flujo de trabajo, interfaz y el flujo de negocio son realizados de manera masiva y transparente para ellos\citep{kumar_cloud_2012}\citep{kang_web_2012}.

\subsection{Características}
Entre las características relevantes de las aplicaciones SaaS, y las que las remarcan como aplicaciones bien construidas, son construidas a la medida, escalables y soportan multitenancy\footnote{En español se conoce como multiples clientes inquilinos, se refiere cuando varios clientes pueden utilizar la misma instancia.}. No todas las aplicaciones SaaS comparten todas las características, pero las más comunes son las siguientes: [1][3][4][5]

\subsubsection{Configurabilidad}
Las aplicaciones con esta característica poseen el mismo código base y provee a instancias con múltiples opciones de configuración tal que cada cliente pueda tener sus propias configuraciones de software únicas y pueda tener la sensación de que es el único usuario en utilizar la aplicación. Esta es la clave del éxito para las aplicaciones SaaS. Por ejemplo, cada cliente puede tener configurado su sitio para que muestren fondos de pantallas o logos en las páginas de inicio de sesión o páginas principales que ellos especifican. Esta característica también puede ser llamada personalización de la aplicación.

\subsubsection{Multitenancy}
Con esta característica, una sola instancia de la aplicación corriendo puede servir a una cantidad de clientes. Los diferentes modelos de datos que están disponibles para soportar SaaS son las bases de datos aisladas, arquitectura de bases de datos aisladas compartidas y la construcción de datos. Utilizando la arquitectura multitenant los proveedores de aplicaciones SaaS pueden innovar de forma sencilla y ahorrar tiempo valioso gastado en mantener varias versiones de código deprecado y/o desactualizado.

\subsubsection{Escalabilidad}
Esta característica es la más complicada de agregar a una aplicación SaaS debido a su elevado costo. La escalabilidad es soportada por la virtualización, pero teniendo en cuenta el costo y el problema de complejidad muchas veces el desarrollador de la aplicación no se complica con esta característica.

\subsection{Ventajas}
Además, aparte de estas características, algunas ventajas que poseen las aplicaciones con arquitectura SaaS son las siguientes: [4][6][11]
\begin{itemize}
	\item Las aplicaciones SaaS pueden ser utilizadas por los usuarios por medio de sus navegadores Web. Esto ahorra en costos operacionales para el usuario junto con los requerimientos de hardware mínimo, por lo tanto, reduce el costo que el usuario necesita gastar en hardware.  Además de los costos de mantenimiento, costos de licencias del software también son minimizados.
	\item Mejor utilización de recursos, debido a que los recursos requeridos por las aplicaciones con arquitectura SaaS son mínimos.
	\item El avance de la tecnología Web permite que los proveedores de SaaS se ubiquen en el extranjero y también ofrezcan servicios de alta calidad. De esta manera, permite a los usuarios de la aplicación ahorrar en infraestructura.
	\item Usualmente, las soluciones SaaS residen en entornos en la nube donde son escalables y poseen integración con otras ventajas SaaS. En comparación al modelo tradicional, los usuarios no tienen que comprar otro servidor o software.
\end{itemize}
\section{Sistemas de gestión de evaluación basados en competencias}
Un AMS es un sistema, generalmente basado en tecnologías web, que permite a la institución la recolección, el manejo, y reporte de datos relacionados a las evaluaciones del estudiante, por lo general basadas en competencias. Los AMS permiten a la institución y a los educadores listar sus competencias, guardar, y mantener datos para cada una, facilitar conexiones a competencias similares de la institución, y generar reportes\citep{cartwright2009student}.

Estas aplicaciones permiten que las competencias sean enlazadas a nivel institucional, departamental, de programas, y de división. Esta estructura permite examinar la competencia
acorde al nivel que pertenece. Una representación común de este tipo de enlace es el mapa curricular\citep{oakleaf_choosing_2013}. Algunos de estos sistemas de manejo de evaluaciones generan sus propios mapas curriculares.

Varios sistemas comerciales existen; incluyendo Blackboard Learn, Campus Lab, eLumen, LiveText, TaskStream, TracDat/Webfolio, Waypoint Outcomes y WEAVEOnline. También existen otras desarrolladas por las propias instituciones para manejar sus datos de evaluaciones.

Mientras que cada AMS tiene un conjunto diferente de capacidades, todas manejan, mantienen, y permiten generar reportes de los datos de las evaluaciones. Generalmente, teniendo como ejemplo los distintos sistemas, los AMS tienen una estructura jerárquica basada en unidades organizacionales (programas, departamentos, escuelas, colegios o la misma institución), por lo tanto, las metas y/o las competencias también se ven adaptadas a esta estructura.

\subsection{Capacidad de evaluación}
La característica más importante de todo AMS es la capacidad de soportar evaluaciones de diferentes tipos\citep{oakleaf_choosing_2013}. Por ejemplo, algunos AMS se centran en soportar evaluaciones acumulativas; mientras que otros permiten el seguimiento de las evaluaciones formativas. Además, las capacidades de evaluación apoyadas por una AMS pueden residir en una escala de la unidad.

Algunas de las mencionadas anteriormente permiten evaluaciones a nivel de estudiante. Un número cada vez mayor de las AMS apoyan la documentación, desarrollo o aplicación de criterios de evaluación específicos, más comúnmente a las rúbricas que se aplican a los productos creados por los estudiantes.

Como una faceta adicional, muchas de estas herramientas permiten evaluaciones para vincularse con las normas educativas y profesionales, de modo que la información de evaluación de múltiples unidades puede ser adherido como datos de reportes.

\subsection{Alineación de capacidades}
Una importante característica de cualquier AMS es la habilidad de enlazar competencias entre sí.

Primeramente, algunos AMS permiten que las competencias sean enlazadas a nivel de institución, departamento, o programa. Esta estructura permite examinar la competencia acorde al nivel que pertenece. Una representación común de este tipo de enlace es el mapa curricular. Algunos de estos sistemas de manejo de evaluaciones generan sus propios mapas curriculares.

Además, provee soporte a iniciativas de mejora continua educacionales de instituciones y competencias de aprendizaje que necesitan las evaluaciones. Un AMS puede ser levantado y utilizado para mantener y alcanzar altos estándares de calidad y cumpliendo los requisitos de acreditación\citep{kuh_using_2015}.

Esta evaluación es un proceso complejo que requiere la contribución y la retroalimentación de todo el personal, profesores y alumnado de un centro de educación superior. El sistema AMS facilita la esquematización, la recopilación de pruebas, documentación y presentación de las contribuciones que cada uno de los programas académicos de la institución y los servicios de apoyo hace la consecución de los objetivos de calidad y eficacia institucionales.

Un ciclo de evaluación completo incluye la coordinación, la planificación, la medición, la reflexión y la toma de acción del proceso de evaluación de la institución enteras, programas o cursos.
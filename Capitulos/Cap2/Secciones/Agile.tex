\section{Metodología ágil de desarrollo de software}
La metodología Ágil envuelve un enfoque para la toma de decisiones en los proyectos de software, que se refiere a métodos de ingeniería del software basados en el desarrollo iterativo e incremental, donde los requisitos y soluciones evolucionan con el tiempo según la necesidad del proyecto. Los métodos tradicionales, como Waterfall, pretenden ser capaces de modelar completamente el dominio del problema de entrada y luego esperar que se produzcan pequeños cambios (o inclusive ninguno)\citep{davis_agile_2015}. Los métodos ágiles asumen que el cambio es inevitable, por lo que abordan el desarrollo de software de tal manera a facilitar la adaptación de los nuevos requisitos mientras vayan surgiendo.

Las metodologías ágiles, en comparación a otras metodologías de desarrollo, ofrecen un modelo de diseño flexible que fomenta al desarrollo evolutivo. Los desarrolladores trabajan en pequeños módulos cada vez y la retroalimentación proveída por el cliente ocurre simultáneamente en el desarrollo. Además, la metodología puede ser bastante útil en situaciones donde los objetivos finales del proyecto no están claramente definidos donde los requisitos del cliente se clarificarán gradualmente a medida que el proyecto avance.

El uso de la metodología ágil como método de entregas de las características del módulo entra como requisito no funcional.

\subsection{Historias de usuario}
Las historias de usuario conforman la parte central de muchas metodologías de desarrollo ágil, tales como XP\footnote{de sus siglas en inglés, eXtreme Programming, que significa en español programación extrema}, Scrum, entre otras. Estas definen lo que se debe construir en el proyecto de software, tienen una prioridad asociada definida por el cliente de manera a indicar cuales son las más importantes para el resultado final. Son divididas en tareas y su tiempo es estimado por los desarrolladores.

Por lo general, se espera que una estimación de tiempo de cada historia de usuario se sitúe entre horas y el tiempo máximo de iteración. Estimaciones superiores a este tiempo máximo son indicativas de que la historia es muy compleja y debe ser dividida en varias historias.

Una historia de usuario es una representación de un requisito escrito en una o dos frases utilizando el lenguaje común del usuario\citep{davis_agile_2015}. Ellas son utilizadas para la especificación de requisitos acompañadas de las discusiones con aquellos y las pruebas de validación.

Cada historia de usuario debe ser limitada. La metodología estipula que las mismas deben ser escritas por los clientes. Son una forma rápida de administrar los requisitos sin tener que elaborar gran cantidad de documentos formales y sin requerir de mucho tiempo para administrarlos.

Las historias de usuario deben ser:
\begin{itemize}
    \item Independientes unas de otras. De ser necesario, combinar las historias dependientes o buscar otra forma de dividir las historias de manera que resulten independientes.
    \item Negociables. La historia en sí misma no es lo suficientemente explicita para considerarse un contrato, la discusión con los usuarios debe permitir esclarecerse y éste debe dejarse explicito bajo la forma de pruebas de validación.
    \item Valoradas por los clientes o usuarios. Los intereses de los clientes y de los usuarios no siempre coinciden, pero en todo caso, cada historia debe ser más importante para los clientes que para el desarrollador.
    \item Pequeñas. Las historias grandes son difíciles de estimar e imponen restricciones sobre la planificación de un desarrollo iterativo. Generalmente se recomienda la consolidación de historias muy cortas en una sola historia.
    \item Verificables. Las historias de usuario cubren requerimientos funcionales, por lo generalmente son verificables. Cuando sea posible, la verificación debe automatizarse, de manera que pueda ser verificada en cada entrega del proyecto.
\end{itemize}

Las iniciales de estas características, con sus nombres en inglés, forman la palabra INVEST, que significa “inversión”. Es porque toda historia de usuario, si se construye bien, es una inversión.

Al momento de implementar las historias, los desarrolladores deben tener la posibilidad de discutirlas con los clientes. El estilo sucinto de las historias podría dificultar su interpretación, podría requerir conocimientos de base sobre el modelo, o podría haber cambiado desde que fue escrita.

Cada historia de usuario debe tener en algún momento pruebas de validación asociadas, lo que permitirá al desarrollador, y más tarde al cliente, verificar si la historia ha sido completada. Como no se dispone de una formulación de requisitos precisa, la ausencia de pruebas de validación concertadas abre la posibilidad de discusiones largas y no constructivas al momento de la entrega del producto.

\subsection{Épicas}
Una épica es esencialmente una historia de usuario de un tamaño mucho mayor, siempre superior al tiempo de iteración máximo, y tiene como propósito el de asociar historias de usuario individuales relacionadas con un propósito de más alto nivel que cumplir. La misma es, por lo general, muy grande para que un equipo del proyecto pueda trabajar directamente sin partir en diversas historias de usuario\citep{cobb2015project}.

El uso de las épicas en proyectos de gran tamaño ayuda a organizar tareas complejas en un tipo de estructura para que la interrelación de historias de usuario esté bien entendida. Por lo tanto, el diseño de épicas en el proceso de desarrollo es fundamental antes de comenzar el proyecto.
% Chapter 8

\chapter{Conclusión y aportes} \label{capitulo8} 

\section{Conclusiones generales}
En el proyecto final se tomó como caso de estudio con enfoque en la interacción humano-computador y con observación participante el diseño e implementación de un módulo de gestión curricular para un sistema de gestión de evaluaciones basadas en competencias académicas, la cual tiene sus cimientos en el mercado y también dispone de clientes utilizando la misma. 

Se diseñó la manera de integrar y estructurar procesos separados de validación de competencias, cursos, y programas para el estado de California en un sistema de gestión de evaluaciones basadas en competencias. Dicho proceso era un proceso que se hacía en papel y tenía sus falencias debido a que el proceso requería mucho tiempo en revisar y aprobar el formulario como se muestra en la figura \ref{course_creation_flow}, y la complejidad del flujo aumentaba cuando habían más personas colaboradoras o evaluadoras en el proceso.

Debido a que los requerimientos eran cambiantes y el equipo que diseñaba no disponía de un panorama completo de las funcionalidades del módulo curricular, la elección de la metodología ágil para el desarrollo del proyecto fué acertada debido a que la misma permitía el desarrollo iterativo e incremental del software con validaciones del cliente como proceso de desarrollo, que en este caso el equipo de validación tomaba el rol de cliente ya que ellos tenían comunicación directa con las diferentes instituciones académicas que utilizan la aplicación.

Al diseñar los flujos de trabajo para el diseño y revisión de formularios de competencias, cursos, y programas permitió a los profesores encargados de los cursos y evaluadores de dichos formularios seguir el proceso de una manera intuitiva buscando la mejor experiencia de usuario y con menos cuellos de botella debido a que cada paso tenía notificaciones que el usuario podía acceder en su buzón de entrada en caso de que tuviera trabajo pendiente.

En el desarrollo del módulo se utilizaron muchas de las tecnologías y herramientas que disponía el sistema como requerimiento no funcional de parte de la organización. El uso de estas tablas en común para la funcionalidad de plantillas de flujos de trabajo fué una decisión errónea, debido a que agregaba complejidad a tablas y además las pruebas también habían que hacerse a los módulos que utilizaban esas tablas en el AMS porque podría ocasionar errores en la aplicación. 

Sin embargo, al utilizar MySQL para guardar el flujo de trabajo y todos sus datos temporales no fué la mejor decisión debido a que la funcionalidad tiene cambios constantes en cuanto a datos que tendrían que guardarse y migrar los datos y columnas de los usuarios aumentaba la complejidad de la historia de usuario.

Debido a la falta de retroalimentación de parte de los usuarios finales no se pudieron enderezar las historias para que vayan por el camino correcto y eso conlleva a cambios futuros muchos más grandes.

A pesar de todas las falencias de desarrollo, el módulo tuvo resultados positivos por parte de los usuarios finales ya que era una herramienta que automatizaba trabajos de validación curricular para las instituciones. Además, al tener comentarios acerca de que habría que mejorar en la aplicación y con la utilización de la metodología ágil se permitió que se creen de nuevas historias de usuario para algunos retoques futuros del módulo curricular. 

\section{Aportes del proyecto}
Se puede afirmar que la llegada del módulo curricular representa la tecnología al servicio de la educación, en específico para automatizar el proceso de diseño curricular. 

Una buena parte de su importancia, radica en las amplias posibilidades que ofrece, entre las ya mencionadas y más sobresaliente se centra en la capacidad otorgada por estos sistemas para gestionar el diseño y validación de material curricular en el estado de California, y la comunicación que ofrece entre los encargados del diseño de los formularios y los evaluadores de los mismos.

Además, podemos citar otros aportes:
\begin{itemize}
	\item El módulo desarrollado reemplaza la preparación manual de formularios, los procesos de revisión y aprobación curricular de colegios ya sea para competencias, cursos, y programas.
	\item Al implementar el módulo en un ambiente que puede ser accedido desde cualquier navegador, permite a las personas trabajar en conjunto sin necesidad de agendar reuniones para desarrollar el formulario o las revisiones.
	\item Permite almacenar los datos nuevos, históricos, propuestos y activos de competencias, cursos, y programas.
	\item Provee notificaciones automatizadas cuando hay cambios de estado en los flujos de trabajo.
	\item Disminución del tiempo promedio de formulación y revisión de cada flujo de diseño e implementación de flujos de trabajo.
	\item Se facilitó llenar formularios utilizando información ya existente en el AMS.
\end{itemize}


\section{Proyectos futuros}
En consenso con los usuarios finales de la aplicación y los que diseñan las historias de usuarios se observaron ciertas características que podrían dar una mayor utilidad al proyecto, donde citaremos algunos de los trabajos futuros ya creadas como épicas del proyecto:
\begin{itemize}
	\item \textbf{Importador de cursos:} muchos de los cursos que ya fueron agregados al sistema de gestión curricular del estado de California existen y como trabajo futuro para el módulo curricular es la forma de importar todos estos cursos al AMS sin necesidad de hacer todo el flujo de trabajo para las competencias, cursos y programas válidos actualmente.
	\item \textbf{Migración de motor de base de datos de los flujos de trabajo:} como ya comentamos, la decisión de tecnología en cuanto al motor de base de datos fué una decisión errónea, debido a que el estándar de California para diseño y revisión de cursos y programas puede variar con el tiempo y la utilización de MySQL como motor de base de datos dificulta la migración de los datos de los clientes, por lo que se considera un motor de base de datos no relacional basada en documentos como \enquote{MongoDB}.
	\item \textbf{Acceso de información a través de API pública:} es una práctica que exige el estado de California que todos los datos en cuanto a cursos y programas de las universidades puedan ser accedidas desde una API\footnote{de sus siglas en inglés, Application Programming Interface, que sirve como un conjunto de reglas para que las aplicaciones puedan comunicarse entre ellas.} pública. Por lo tanto, se debe diseñar e implementar una interfaz pública que permita comunicarse y acceder a los datos del módulo curricular.
	\item \textbf{Flujo de trabajo para evaluaciones:} el mismo flujo de trabajo que se diseño e implementó para las competencias, cursos, y programas se debe repetir para las evaluaciones ya existentes en el sistema. Dicho trabajo conlleva también migraciones de todos los datos que el AMS posee de los clientes.
	\item \textbf{Catálogo de cursos:} los CMS del estado tienen la opción de mostrar sus cursos válidos de manera pública para que las universidades puedan cargar en sus sistemas académicos como se aprecia en la figura \ref{after_creation}. Por lo tanto, como trabajo futuro para el módulo curricular se busca catalogar los cursos de manera pública desde una interfaz web.
\end{itemize}
% Todo lo que considero como ejemplo proviene de tesis de ejemplo para el grado de ingeniero del MIT.

%%%%%%%%%%%%%%%%%%%%%%%%%%%%%%%%%%%%%%%%%
% Masters/Doctoral Thesis
% LaTeX Template
% Version 2.3 (25/3/16)
%
% This template has been downloaded from:
% http://www.LaTeXTemplates.com
%
% Version 2.x major modifications by:
% Vel (vel@latextemplates.com)
%
% This template is based on a template by:
% Steve Gunn (http://users.ecs.soton.ac.uk/srg/softwaretools/document/templates/)
% Sunil Patel (http://www.sunilpatel.co.uk/thesis-template/)
%
% Template license:
% CC BY-NC-SA 3.0 (http://creativecommons.org/licenses/by-nc-sa/3.0/)
%
%%%%%%%%%%%%%%%%%%%%%%%%%%%%%%%%%%%%%%%%%

%----------------------------------------------------------------------------------------
%	PACKAGES AND OTHER DOCUMENT CONFIGURATIONS
%----------------------------------------------------------------------------------------

\documentclass[
11pt, % The default document font size, options: 10pt, 11pt, 12pt
% oneside, % Two side (alternating margins) for binding by default, uncomment to switch to one side
%chapterinoneline,% Have the chapter title next to the number in one single line
english, % ngerman for German
spanish,
singlespacing, % Single line spacing, alternatives: onehalfspacing or doublespacing
%draft, % Uncomment to enable draft mode (no pictures, no links, overfull hboxes indicated)
%nolistspacing, % If the document is onehalfspacing or doublespacing, uncomment this to set spacing in lists to single
liststotoc, % Uncomment to add the list of figures/tables/etc to the table of contents
%toctotoc, % Uncomment to add the main table of contents to the table of contents
parskip, % Uncomment to add space between paragraphs
%nohyperref, % Uncomment to not load the hyperref package
headsepline, % Uncomment to get a line under the header
]{MastersDoctoralThesis} % The class file specifying the document structure

\usepackage{mathtools}% http://ctan.org/pkg/mathtools
\usepackage[utf8]{inputenc} % Required for inputting international characters
\usepackage[T1]{fontenc} % Output font encoding for international characters
\usepackage{palatino} % Use the Palatino font by default
\usepackage{float} % Para ubicar imagenes en donde corresponden
\usepackage{flexisym}
\usepackage[table]{xcolor}% http://ctan.org/pkg/xcolor
\usepackage{booktabs}

% \usepackage{indentfirst}
% \setlength\parskip{.5\baselineskip plus .1\baselineskip  minus .1\baselineskip}
% \setlength{\parindent}{1em}

\setlength{\parindent}{0.5cm}

\usepackage[ddmmyyyy]{datetime}

\usepackage{titlesec}
\setcounter{secnumdepth}{4}
\titleformat{\paragraph}{\normalfont\normalsize\bfseries}{\theparagraph}{1em}{}
\titlespacing*{\paragraph}{0pt}{3.25ex plus 1ex minus .2ex}{1.5ex plus .2ex}

\usepackage{caption}
\captionsetup[table]{name=Tabla}
\captionsetup[figure]{name=Figura}

\usepackage[backend=bibtex,style=numeric,maxcitenames=1,natbib=true,sorting=none]{biblatex}
\addbibresource{references.bib} % The filename of the bibliography
\usepackage[autostyle=true]{csquotes} % Required to generate language-dependent quotes in the bibliography

\DefineBibliographyStrings{spanish}{%
  andothers = {et\addabbrvspace al\adddot}
}

\usepackage{amsmath}
\usepackage{amssymb}
\usepackage{graphicx}

% Paquetes para las tablas
\usepackage{tablefootnote}

% Paquete para Gantt
\usepackage{pgfgantt}


% BULLET a DASH
\usepackage{enumitem}
\setlist[itemize]{label=$-$}

%----------------------------------------------------------------------------------------
%	MARGIN SETTINGS
%----------------------------------------------------------------------------------------

\geometry{
	paper=a4paper, % Change to letterpaper for US letter
	inner=2.5cm, % Inner margin
	outer=3.8cm, % Outer margin
	bindingoffset=2cm, % Binding offset
	top=1.5cm, % Top margin
	bottom=1.5cm, % Bottom margin
	%showframe,% show how the type block is set on the page
}

%----------------------------------------------------------------------------------------
%	THESIS INFORMATION
%----------------------------------------------------------------------------------------

\thesistitle{Desarrollo de un sistema de gestión de programas de estudio orientado a resultados pedagógicos} % Your thesis title, this is used in the title and abstract, print it elsewhere with \ttitle
\supervisor{Ing. Sebastián \textsc{Ortíz} \\ Ing. Wilfrido \textsc{Inchaustti}} % Your supervisor's name, this is used in the title page, print it elsewhere with \supname
\examiner{} % Your examiner's name, this is not currently used anywhere in the template, print it elsewhere with \examname
\degree{Ingeniería Informática} % Your degree name, this is used in the title page and abstract, print it elsewhere with \degreename
\author{Luis F. \textsc{Villalba V.}} % Your name, this is used in the title page and abstract, print it elsewhere with \authorname
\addresses{} % Your address, this is not currently used anywhere in the template, print it elsewhere with \addressname

\subject{Desarrollo ágil} % Your subject area, this is not currently used anywhere in the template, print it elsewhere with \subjectname
\keywords{agile, development, ams, cms, competencias, diseño, curriculum, desarrollo, observación, participante, caso, estudio} % Keywords for your thesis, this is not currently used anywhere in the template, print it elsewhere with \keywordnames
\university{\href{http://www.universidadcatolica.edu.py/}{Universidad Católica "Nuestra Señora de la Asunción"}} % Your university's name and URL, this is used in the title page and abstract, print it elsewhere with \univname
\department{\href{http://www.dei.uc.edu.py/}{Departamento de Electrónica e Informática}} % Your department's name and URL, this is used in the title page and abstract, print it elsewhere with \deptname
\faculty{\href{http://www.cyt.uc.edu.py/}{Facultad de Ciencias y Tecnología}} % Your faculty's name and URL, this is used in the title page and abstract, print it elsewhere with \facname

\hypersetup{pdftitle=\ttitle} % Set the PDF's title to your title
\hypersetup{pdfauthor=\authorname} % Set the PDF's author to your name
\hypersetup{pdfkeywords=\keywordnames} % Set the PDF's keywords to your keywords
\hypersetup{
    colorlinks = blue,
    filecolor = magenta,
    linkcolor = blue,		% Links del indice, etc
    urlcolor = blue,		% Links de texto con URL asociado
    citecolor = blue,		% Links de texto citado
}

\begin{document}

\frontmatter % Use roman page numbering style (i, ii, iii, iv...) for the pre-content pages
\pagestyle{plain} % Default to the plain heading style until the thesis style is called for the body content

%----------------------------------------------------------------------------------------
%	COVER PAGE
%----------------------------------------------------------------------------------------

\begin{titlepage}
\begin{center}

\begin{figure}[H]
\centering
\includegraphics{Figuras/uca}
\decoRule
%\caption[Logo de la UCA]{UCA.}
\end{figure}

{\scshape\LARGE \univname\par}\vspace{1.5cm} % University name
\textsc{\Large Proyecto final de carrera}\\[0.5cm] % Thesis type

\HRule \\[0.4cm] % Horizontal line
{\huge \bfseries \ttitle\par}\vspace{0.4cm} % Thesis title
\HRule \\[1cm] % Horizontal line

\Large \emph{Autor:}\\
\Large \href{}{\authorname} \\[5mm] % Author name - remove the \href bracket to remove the link
\Large \emph{Tutores:}\\
\Large \href{}{\supname} \\[1cm]

\Large \facname\\ % Research group name and department name
\Large \deptname \\[5mm]

{\large\the\year}\\[1cm] % Date

\end{center}
\end{titlepage}


%----------------------------------------------------------------------------------------
%	TITLE PAGE
%----------------------------------------------------------------------------------------

\begin{titlepage}
\begin{center}
\vspace*{3cm}

{\huge \ttitle\par}\vspace{1cm} % Thesis title
\HRule \\[2cm] % Horizontal line

\begin{minipage}[t]{0.4\textwidth}
\begin{flushleft} \large
\emph{Autor:}\\
\href{}{\authorname} % Author name - remove the \href bracket to remove the link
\end{flushleft}
\end{minipage}
\begin{minipage}[t]{0.4\textwidth}
\begin{flushright} \large
\emph{Tutores:} \\
\href{}{\supname} % Supervisor name - remove the \href bracket to remove the link
\end{flushright}
\end{minipage}\\[2cm]

\large \textit{Una tesis presentada en cumplimiento de los requisitos\\ para el título de \degreename \ en la}\\[1cm]

\univname\\ % University name
\facname\\ % Research group name and department name
\deptname \\[2cm]

{\large\the\year}\\[1cm] % Date

\vfill

\end{center}
\end{titlepage}


%----------------------------------------------------------------------------------------
%	DECLARATION PAGE
%----------------------------------------------------------------------------------------

\renewcommand{\authorshipname}{Declaración de autoría}

\begin{declaration}
\addchaptertocentry{\authorshipname}

\noindent Yo, \authorname, declaro que la tesis titulada, \enquote{\ttitle} y todo el trabajo presentado son de mi propiedad. Además, confirmo que:

\begin{itemize}
\item Este trabajo fue llevado a cabo bajo las normas y regulaciones del Reglamento De Proyecto Final de Carrera (09/2015) del Departamento de Ingeniería Electrónica e Informática.
\item Este trabajo fue realizado en su totalidad con fines investigativos y para obtener el título antes mencionado en esta universidad.
\item Cuando cualquier parte de esta tesis haya sido previamente sometida a un título en esta universidad o cualquier otra institución, esto se ha manifestado claramente.
\item Donde he consultado trabajo publicado por otros autores, eso ha sido siempre claramente atribuido.
\item Donde he citado trabajo de otros autores, la fuente siempre se ha dado. Con la excepción de dichas citas, esta tesis es enteramente mi propio trabajo.
\item He reconocido todas las fuentes principales de ayuda.
\item Donde la tesis está basada en mi trabajo en conjunto con otros autores, he expresado claramente que fue hecho por ellos y en que he contribuido.\\
\end{itemize}

\noindent Firma:\\
\rule[0.5em]{25em}{0.5pt} % This prints a line for the signature

\noindent Fecha:\\
\rule[0.5em]{25em}{0.5pt} % This prints a line to write the date
\end{declaration}

\cleardoublepage

%----------------------------------------------------------------------------------------
%	QUOTATION PAGE	solo si es necesario
%----------------------------------------------------------------------------------------

%\vspace*{0.2\textheight}

%\noindent\enquote{\itshape Thanks to my solid academic training, today I can write hundreds of words on virtually any topic without possessing a shred of information, which is how I got a good job in journalism.}\bigbreak

%\hfill Dave Barry

%----------------------------------------------------------------------------------------
%	ABSTRACT PAGE
%----------------------------------------------------------------------------------------

\renewcommand{\abstractname}{Abstract}

\begin{abstract}
\addchaptertocentry{\abstractname} % Add the abstract to the table of contents
En el proyecto final se toma como caso de estudio con enfoque en interacción humano-computador y en observación participante al diseño e implementación de un módulo de gestión curricular para un sistema de gestión de evaluaciones basadas en competencias académicas, la cual tiene sus cimientos en el mercado y también dispone de clientes utilizando la misma. 

Se diseñó la manera de integrar y estructurar procesos separados de validación de competencias, cursos, y programas para el estado de California en un sistema de gestión de evaluaciones basadas en competencias. Dicho proceso era un proceso que se hacía en papel y tenía sus falencias debido a que el proceso requería mucho tiempo en revisar y aprobar el formulario, y la complejidad del flujo aumentaba cuando habían más personas colaboradoras o evaluadoras en el proceso.

Debido a que los requerimientos eran cambiantes y el equipo que diseña no dispone de un panorama completo de las funcionalidades del módulo curricular, la elección de la metodología ágil para el desarrollo del proyecto fue utilizada debido a que la misma permite el desarrollo iterativo e incremental del software con validaciones del cliente como proceso de desarrollo, que en este caso el equipo de validación tomará el rol de cliente debido al conocimiento y experiencia en didáctica de sus miembros.

% Al diseñar los flujos de trabajo para el diseño y revisión de formularios de competencias, cursos, y programas permitió a los profesores encargados de los mismos y evaluadores de dichos formularios, seguir el proceso de una manera intuitiva buscando la mejor experiencia de usuario y con menos cuellos de botella. Como ayuda, que cada paso genera un mensaje que el usuario puede acceder en su buzón de entrada en caso de que tuviera trabajo pendiente.

% En el desarrollo del módulo se utilizaron muchas de las tecnologías y herramientas que disponía el sistema como requerimiento no funcional de parte de la organización. El uso de estas tablas en común para la funcionalidad de plantillas de flujos de trabajo fue una decisión errónea, debido a que agregaba complejidad a las mismas y además las pruebas de componentes se convertían en pruebas de regresión debido a la complejidad de la estructura. 

% Sin embargo, al utilizar MySQL para guardar el flujo de trabajo y todos sus datos temporales no fue la mejor decisión debido a que la funcionalidad y el estándar tienen cambios constantes en cuanto a datos que tendrían que guardarse, y migrar los datos y columnas de los usuarios aumenta siempre la complejidad de la historia de usuario.

% Debido a la capa adicional de comunicaciones con el usuario final personificada por el equipo de Estados Unidos (que en nuestro caso actúa como cliente), la realimentación de valor real o valor aún necesario provisto al usuario final es lenta e implica grandes cambios luego de varios sprints.

% A pesar de todas las falencias de desarrollo, el módulo tuvo resultados positivos por parte de los usuarios finales ya que es una herramienta que automatiza trabajos de validación curricular para las instituciones. Además, al tener comentarios acerca de qué habría que mejorar en la aplicación y con la utilización de la metodología ágil se permitió que se creen nuevas historias de usuario para algunos retoques futuros en el módulo curricular. 


\end{abstract}

%----------------------------------------------------------------------------------------
%	ACKNOWLEDGEMENTS
%----------------------------------------------------------------------------------------

\renewcommand{\acknowledgementname}{Agradecimientos}

\begin{acknowledgements}
\addchaptertocentry{\acknowledgementname} % Add the acknowledgements to the table of contents

%  Se escribe en primera persona. Se agradece a la universidad por impartir los conocimientos, al tutor, al co-tutor, a la mesa de tesis, a todas las personas que influyeron o colaboraron de alguna forma en el proyeto, Paz, padres, hermanos, familia, amigos de la facultad, compañeros de trabajo, compañeros de colegio, trankas, etc.
The acknowledgments and the people to thank go here, don't forget to include your project advisor\ldots

\end{acknowledgements}

%----------------------------------------------------------------------------------------
%	LIST OF CONTENTS/FIGURES/TABLES PAGES
%----------------------------------------------------------------------------------------

\renewcommand{\contentsname}{Índice General}
\tableofcontents % Prints the main table of contents

\renewcommand{\listfigurename}{Lista de Figuras}
\listoffigures % Prints the list of figures

\renewcommand{\listtablename}{Lista de Tablas}
\listoftables % Prints the list of tables

%----------------------------------------------------------------------------------------
%	ABBREVIATIONS
%----------------------------------------------------------------------------------------

\begin{abbreviations}{ll} % Include a list of abbreviations (a table of two columns)

\textbf{AMS} & \textbf{A}ssessment \textbf{M}anagement \textbf{S}ystem\\
\textbf{CMS} & \textbf{C}urriculum \textbf{M}anagement \textbf{S}ystem \\
\textbf{UX} & \textbf{U}ser e\textbf{X}perience \\
\textbf{LMS} & \textbf{L}earning \textbf{M}anagement \textbf{S}ystem \\
\textbf{SaaS} & \textbf{S}oftware \textbf{a}s \textbf{a} \textbf{S}ervice \\
\textbf{SLO} & \textbf{S}tudent \textbf{L}earning \textbf{O}utcomes \\
\textbf{PCAH} & \textbf{P}rogram and \textbf{C}ourse \textbf{A}pproval \textbf{H}andbook \\
\textbf{INVEST} & \textbf{I}ndependent \textbf{N}egotiable \textbf{V}aluable \textbf{E}stimatable \textbf{S}mall \textbf{T}estable \\
\textbf{HCI} & \textbf{H}uman \textbf{C}omputer \textbf{I}nteraction \\
\textbf{UI} & \textbf{U}ser \textbf{I}nteraction \\
\textbf{DECA} & \textbf{DE}cision a\textbf{CA}demic \\
\textbf{AWS} & \textbf{A}mazon \textbf{W}eb \textbf{S}ervices \\
\textbf{VCS} & \textbf{V}ersion \textbf{C}ontrol \textbf{S}ystem \\
\textbf{MVC} & \textbf{M}odel \textbf{V}iew \textbf{C}ontroller \\
\textbf{UI} & \textbf{U}ser \textbf{I}nteraction \\
\textbf{POJO} & \textbf{P}lain \textbf{O}ld \textbf{J}ava \textbf{O}bject \\

\end{abbreviations}

%----------------------------------------------------------------------------------------
%	PHYSICAL CONSTANTS/OTHER DEFINITIONS
%----------------------------------------------------------------------------------------

%\begin{constants}{lr@{${}={}$}l} % The list of physical constants is a three column table

% The \SI{}{} command is provided by the siunitx package, see its documentation for instructions on how to use it

	%Speed of Light & $c_{0}$ & \SI{2.99792458e8}{\meter\per\second} (exact)\\
%Constant Name & $Symbol$ & $Constant Value$ with units\\

%\end{constants}

%----------------------------------------------------------------------------------------
%	SYMBOLS
%----------------------------------------------------------------------------------------

%\begin{symbols}{lll} % Include a list of Symbols (a three column table)

%$a$ & distance & \si{\meter} \\
%$P$ & power & \si{\watt} (\si{\joule\per\second}) \\
%Symbol & Name & Unit \\

%\addlinespace % Gap to separate the Roman symbols from the Greek

%$\omega$ & angular frequency & \si{\radian} \\

%\end{symbols}

%----------------------------------------------------------------------------------------
%	DEDICATION   solo si es necesario
%----------------------------------------------------------------------------------------

%\dedicatory{ Dedicado a\ldots}
% For/Dedicated to/To my

%----------------------------------------------------------------------------------------
%	THESIS CONTENT - CHAPTERS
%----------------------------------------------------------------------------------------

\mainmatter % Begin numeric (1,2,3...) page numbering

\pagestyle{thesis} % Return the page headers back to the "thesis" style

% Include the chapters of the thesis as separate files from the Chapters folder
\renewcommand{\chaptername}{Capítulo}
\newcommand{\ra}[1]{\renewcommand{\arraystretch}{#1}}

% INTRODUCCION
\chapter{Introducción}
\section{Justificación}
En la actualidad, la tecnología forma parte de nuestra vida cotidiana donde el software ha avanzado con el paso del tiempo. La tendencia apunta a que las aplicaciones puedan ser accedidas en cualquier momento desde cualquier lugar, ya sea en un viaje de negocios o haciendo compras desde cualquier dispositivo con acceso a internet. 

Últimamente, de manera a facilitar el acceso a la aplicación por parte de los usuarios, las aplicaciones web se han vuelto populares porque estas se ejecutan en un navegador y no es necesario descargar ningún tipo de software adicional debido a que el peso principal de la aplicación se ejecuta remotamente\citep{net_app_architecture}.

Al mismo tiempo, en la educación han surgido avances tecnológicos aplicables donde se aprovecha la misma para potenciar el aprendizaje adquirido en los estudiantes. Con esto se busca idear nuevas técnicas de evaluación para impulsar un aprendizaje significativo en los alumnos, pero en muchos casos nos encontramos que la forma de evaluar los cursos o actividades no reflejan los conocimientos o capacidades de los alumnos. Para llenar este vacío surge la evaluación basada en competencias, la cual se enfoca en aquellas adquiridas por un estudiante en el proceso de un programa educativo\citep{kuh_knowing_2014}.  

Así, las aplicaciones de evaluación académica basadas en competencias han adquirido mucha importancia en los últimos años\citep{kuh_knowing_2014}. En dichas aplicaciones se buscan conocer las fortalezas y debilidades del estudiante de una manera modular, en comparación a los métodos cuantitativos de evaluación. Dichos aprendizajes y competencias son expresados por segmentos de estudios o actividades, mediante resultados esperados medibles a nivel institucional, de programa, grado, o de curso; expresados en calificaciones\citep{kuh_using_2015}.

Para las personas ajenas al entorno educativo, en específico aquellas que no participan directamente en el diseño de planes y programas, puede resultar un tanto complejo comprender el proceso del diseño curricular y reconocer la importancia de involucrar a todas aquellas autoridades que forman parte del mismo. Por esta razón, definir las bases teóricas que servirán de referencia para el diseño curricular, describir sus conceptos básicos, y distinguir sus elementos, es fundamental para el desarrollo del curso\citep{boyle_curriculum_2016}.

En el ámbito de las aplicaciones académicas basadas en competencias, si bien existen aplicaciones que abarquen el diseño y la emisión de planes de estudio por parte de los profesores o encargados de las universidades, además de su revisión y posterior aprobación por el comité curricular, no se ha encontrado durante el proceso de investigación una aplicación que integre todos estos a un sistema de gestión de evaluaciones basadas en competencias.
\section{Diseño de la investigación}
El presente trabajo ilustra cómo caso de estudio la aplicación de la metodología ágil en la gestión de un proyecto de desarrollo de un módulo, integrado a un sistema de gestión y evaluación de competencias de una organización ubicada en los Estados Unidos. El trabajo a realizar fue propuesto por la organización que brinda el sistema de gestión de competencias, donde el mismo sirve como base al módulo de gestión curricular a desarrollarse.

El caso refleja de manera práctica se ha desenvuelto un proceso de desarrollo ágil en un diálogo con usuarios ubicados en diferentes localidades como parte del sistema de trabajo. En este proyecto los miembros ubicados en forma remota se encargan del diseño de las tareas llamadas historias de usuario y de la validación de las características entregadas.

Debido a que los requisitos de la aplicación a desarrollarse se irán esclareciendo acorde se vayan completando cada valor de negocio, se optó por la metodología ágil como técnica de gestión de desarrollo de software, puesto que es el enfoque más adecuado para poder encarar esta problemática.

La observación participante de parte del investigador es un paso inicial para el desarrollo del sistema en un nuevo equipo, donde se identifican y guían las relaciones con los informantes, lo ayuda a observar de manera y embebida la organización y dinámica del equipo y las prioridades de desarrollo. También, lo permite integrarse con los demás miembros del mismo y de esa manera le facilita el proceso de investigación, además de proveerle una cantidad de interrogantes a ser dilucidadas con los participantes \citep{erlandson_doing_1993}.

El primer paso es adaptarse a los cambios y a la metodología de trabajo del equipo de desarrollo. Dicho equipo se constituirá luego de a una encuesta previa de capacidades de todos los desarrolladores de la empresa, donde cada uno colocará sus conocimientos en herramientas o en partes del sistema de gestión de competencias, para así poder aprovechar las virtudes de cada persona. Acto seguido se procede al análisis de las herramientas a utilizarse para corroborar que cumplen con las exigencias del sistema a desarrollarse.

Entre algunas técnicas de recolección de información se utilizan las encuestas \citep{robson_real_2011}. Para este sistema un equipo en Estados Unidos utilizó la misma para proporcionar una visión general de los conocimientos generales del equipo de desarrollo, más información al respecto en la sección \ref{analisis_herramientas}.

Los casos de estudio con enfoque HCI\footnote{de sus siglas en inglés, Human-Computer Interaction, que significa en español interacción humano-computador.} tienen como meta la comprensión de problemas o situaciones mediante la interacción del ser humano con la computadora. Además, se busca una documentación descriptiva del sistema y de su proceso de desarrollo, que apunta a la evolución del modelo propuesto durante el diseño de la misma, finalmente, se brinda y analiza evidencia de que la herramienta haya sido utilizada de manera exitosa mediante demostraciones a los usuarios o validaciones por parte del mismo \citep{lazar_research_2010}. 

Por lo tanto, el caso de estudio propuesto es uno con enfoque HCI que utiliza la observación participante como método principal de recolección de información.
\section{Dominio de la problemática}
El proyecto objetivo a desarrollarse utiliza como base una aplicación web AMS\footnote{de sus siglas en inglés, Assessment Management System, que significa en español sistema de gestión de evaluaciones.} que integrará un módulo de gestión curricular a la misma. Los AMS son utilizados por las universidades en Estados Unidos para evaluar las competencias adquiridas de los estudiantes durante el proceso de su carrera o grado. También se busca la disponibilidad en dispositivos móviles. Los dispositivos inteligentes que van a poder utilizar la aplicación se definirán durante el proceso de desarrollo de la misma.

En las aplicaciones dirigidas al ambiente educativo, el hecho de utilizar nuevas tecnologías no asegura una mejor UX\footnote{de sus siglas en inglés, User eXperience, que significa en español experiencia de usuario.}, es por eso que utilizaremos durante el desarrollo de la aplicación técnicas de HCI\citep{lazar_research_2010}, encargadas de utilizar los patrones de diseño e interacción a seguir a la hora de construir cada uno de los componentes de la aplicación a ser desarrollada, y posteriormente validarlos.

Un requisito no funcional que forma parte de la infraestructura del caso de estudio es la utilización de la arquitectura SaaS\footnote{de sus siglas en inglés, Software as a Service, que significa en español software como servicio.}, ya que provee servicios bajo un modelo de pagos de suscripción por las diferentes características que la misma ofrece.

Durante el proceso de desarrollo, se definirán épicas a desarrollarse durante el periodo de diseño de la aplicación y las historias de usuario que están contenidas en las mismas, para que el equipo de desarrollo pueda entregar valor de negocio del módulo integrado en iteraciones cortas de dos semanas.


% OBJETIVOS
\chapter{Objetivos}

\section{Objetivo general}
Diseñar una aplicación que permita integrar y estructurar tareas separadas de un sistema académico para una gestión de programas educativos orientados a resultados pedagógicos, que provea soporte a flujos de trabajo para sus diferentes etapas de aprobación.


\section{Objetivos específicos}
  \begin{itemize}
    \item Realizar un relevamiento de los requerimientos, diseño e implementación de un sistema de gestión de programas orientado a competencias.
    \item Realizar el proyecto en un marco de programación ágil, estimando y desarrollando la aplicación de acuerdo a las directrices brindadas por esta metodología.
    \item Validar la herramienta desarrollada con expertos del dominio principalmente y de manera preliminar con experiencias limitadas con los usuarios finales.
  \end{itemize}

% MARCO TEORICO
\chapter{Marco teórico} % Main chapter title
En este capítulo se presenta el conjunto de escritos académicos tanto técnicos como educativos que constituyen los fundamentos utilizados en este trabajo de tesis. Entre ellas, las competencias, las diferentes herramientas utilizadas por las instituciones educativas con sus diferencias, y definiciones de las diferentes tecnologías utilizadas para el desarrollo del módulo curricular.

\label{capitulo2} % Change X to a consecutive number; for referencing this chapter elsewhere, use \ref{capitulo2}
\section{Competencias académicas y su evaluación}
La creciente profesionalización trajo al campo educativo elementos evaluativos tales como calidad, equidad, competitividad, eficiencia, y eficacia; junto con ellos surgieron las competencias, que pasaron a jugar papel importante en el contexto educativo. En la formación de profesionales, resalta la necesidad de reflexionar sobre los aprendizajes que se ofrecen en las instituciones educativas, las cuales deben servir al estudiante para ser útil a la sociedad, que es su entorno inmediato \citep{kuh_using_2015}. En otras palabras, la competencia es la capacidad de un buen desempeño en contextos complejos y auténticos. Se basa en la integración y activación de conocimientos, habilidades, destrezas, actitudes y valores.

De esa necesidad va surgiendo la idea de las evaluaciones orientadas a competencias. Cabe resaltar cómo se desenvuelve el aprendizaje basado en competencias usando aplicaciones como herramientas para la evaluación de estudiantes, mediante el análisis de los aportes que introduce la tecnología en este campo, que modifican significativamente las prácticas tradicionales\citep{carriveau_connecting_2016}.

Uno de los factores de motivación relevantes para el aprendizaje es la evaluación. Cada actividad ofrece a los estudiantes la oportunidad de conocer cuáles son sus resultados de aprendizaje en lo que se refiere al \enquote{qué} se ha aprendido y al \enquote{cómo} habría podido hacerse. Cualquier proceso de evaluación debería ser diseñado teniendo en cuenta este principio básico.

En un sistema de gestión de evaluaciones basado en competencias, los encargados hacen evaluaciones según las evidencias obtenidas de diversas actividades de aprendizaje, que definen si un estudiante alcanza o no los requisitos recogidos por un conjunto de indicadores en un determinado grado. Una evaluación por competencias asume que pueden establecerse indicadores posibles de alcanzar por los estudiantes, que diferentes actividades de evaluación pueden reflejar los mismos indicadores\citep{barrio_minton_evaluating_2016}.

La evaluación por competencias ofrece nuevas oportunidades a los estudiantes al generar entornos significativos de aprendizaje que acercan sus experiencias académicas al mundo profesional, y donde pueden desarrollar una serie de capacidades integradas y orientadas a la acción, con el objetivo de ser capaces de resolver problemas prácticos o enfrentarse a situaciones cotidianas \citep{carriveau_connecting_2016}.

Hoy día existen herramientas que ayudan al alumno a potenciar su aprendizaje y algunas de ellas son los sistemas de gestión de aprendizajes y los sistemas de gestión de evaluaciones basadas en competencias, cuyas diferencias se mostrarán a posteriori.
\section{Sistemas de gestión de aprendizajes}
Las plataformas LMS\footnote{de sus siglas en inglés, Learning Management System, que significa en español sistema de gestión de aprendizajes.} son espacios virtuales de aprendizaje orientados a facilitar la experiencia de aprendizaje a distancia, donde permite una mejor interacción de profesores y alumnos. También se pueden hacer evaluaciones, intercambiar archivos y participar en foros y chats, además de otras herramientas de interacción estudiante a profesor.

La centralización y automatización de la gestión del aprendizaje es una de las principales características de los LMS. La plataforma puede ser adaptada tanto a los planes de estudio de la institución como a los contenidos y estilo pedagógico de la misma.

El usuario se convierte en el protagonista de su propio aprendizaje a través del autoservicio y los servicios guiados por los tutores o profesores mediante la herramienta. Además, permite utilizar los cursos desarrollados por terceros, personalizando el contenido y reutilizando el conocimiento adquirido. Además, posee prestaciones y características que hacen que cada plataforma sea adecuada según los requerimientos y necesidades de los usuarios.

Los LMS que almacenan información de programas académicos no involucran el aspecto de resultados pedagógicos en sus estructuras de datos o procesos de aprobación. Esta aprobación se realiza a través de flujos de trabajo\citep{aalst_workflow_2004}.
\section{Sistemas de gestión de evaluación basados en competencias}
Un AMS es un sistema, generalmente basado en tecnologías web, que permite a la institución la recolección, el manejo, y reporte de datos relacionados a las evaluaciones del estudiante, por lo general basadas en competencias. Los AMS permiten a la institución y a los educadores listar sus competencias, guardar, y mantener datos para cada una, facilitar conexiones a competencias similares de la institución, y generar reportes\citep{cartwright2009student}.

Estas aplicaciones permiten que las competencias sean enlazadas a nivel institucional, departamental, de programas, y de división. Esta estructura permite examinar la competencia
acorde al nivel que pertenece. Una representación común de este tipo de enlace es el mapa curricular\citep{oakleaf_choosing_2013}. Algunos de estos sistemas de manejo de evaluaciones generan sus propios mapas curriculares.

Varios sistemas comerciales existen; incluyendo Blackboard Learn, Campus Lab, eLumen, LiveText, TaskStream, TracDat/Webfolio, Waypoint Outcomes y WEAVEOnline. También existen otras desarrolladas por las propias instituciones para manejar sus datos de evaluaciones.

Mientras que cada AMS tiene un conjunto diferente de capacidades, todas manejan, mantienen, y permiten generar reportes de los datos de las evaluaciones. Generalmente, teniendo como ejemplo los distintos sistemas, los AMS tienen una estructura jerárquica basada en unidades organizacionales (programas, departamentos, escuelas, colegios o la misma institución), por lo tanto, las metas y/o las competencias también se ven adaptadas a esta estructura.

\subsection{Capacidad de evaluación}
La característica más importante de todo AMS es la capacidad de soportar evaluaciones de diferentes tipos\citep{oakleaf_choosing_2013}. Por ejemplo, algunos AMS se centran en soportar evaluaciones acumulativas; mientras que otros permiten el seguimiento de las evaluaciones formativas. Además, las capacidades de evaluación apoyadas por una AMS pueden residir en una escala de la unidad.

Algunas de las mencionadas anteriormente permiten evaluaciones a nivel de estudiante. Un número cada vez mayor de las AMS apoyan la documentación, desarrollo o aplicación de criterios de evaluación específicos, más comúnmente a las rúbricas que se aplican a los productos creados por los estudiantes.

Como una faceta adicional, muchas de estas herramientas permiten evaluaciones para vincularse con las normas educativas y profesionales, de modo que la información de evaluación de múltiples unidades puede ser adherido como datos de reportes.

\subsection{Alineación de capacidades}
Una importante característica de cualquier AMS es la habilidad de enlazar competencias entre sí.

Primeramente, algunos AMS permiten que las competencias sean enlazadas a nivel de institución, departamento, o programa. Esta estructura permite examinar la competencia acorde al nivel que pertenece. Una representación común de este tipo de enlace es el mapa curricular. Algunos de estos sistemas de manejo de evaluaciones generan sus propios mapas curriculares.

Además, provee soporte a iniciativas de mejora continua educacionales de instituciones y competencias de aprendizaje que necesitan las evaluaciones. Un AMS puede ser levantado y utilizado para mantener y alcanzar altos estándares de calidad y cumpliendo los requisitos de acreditación\citep{kuh_using_2015}.

Esta evaluación es un proceso complejo que requiere la contribución y la retroalimentación de todo el personal, profesores y alumnado de un centro de educación superior. El sistema AMS facilita la esquematización, la recopilación de pruebas, documentación y presentación de las contribuciones que cada uno de los programas académicos de la institución y los servicios de apoyo hace la consecución de los objetivos de calidad y eficacia institucionales.

Un ciclo de evaluación completo incluye la coordinación, la planificación, la medición, la reflexión y la toma de acción del proceso de evaluación de la institución enteras, programas o cursos.
\section{Diferencias entre LMS y AMS}
Por lo general, en las universidades norteamericanas utilizan los AMS y LMS para mejorar el proceso de evaluación de sus estudiantes. Sin embargo, para las personas que no están familiarizadas con estos sistemas pueden quedar ambiguas las diferencias entre estos sistemas de aprendizaje. Estas diferencias se encuentran sumarizadas y tabuladas en la tabla \ref{comparacion_ams_lms}.

\begin{table}[H]
\centering
\caption{Comparación de características entre plataformas AMS y LMS.}
  \label{comparacion_ams_lms}
  \resizebox{\columnwidth}{!}{%
	\begin{tabular}{lllcc}
		\toprule
		\multicolumn{3}{l}{Características}                                                                     & AMS       & LMS       \\
		\midrule
		\multicolumn{3}{l}{Soporte de evaluaciones a los estudiantes.}                                          & $\checkmark$         & $\checkmark$         \\
		\multicolumn{3}{l}{Soporte de evaluación colectiva.}                                                    & $\checkmark$         &           \\
		\multicolumn{3}{l}{Diferentes tipos de rúbricas para evaluaciones.}                                     & $\checkmark$         &           \\
		\multicolumn{3}{l}{Permite acceder a cursos realizados por terceros (aula virtual para el estudiante).} &           & $\checkmark$         \\
		\multicolumn{3}{l}{Permite la evaluación de desempeño estudiantil.}                                     & $\checkmark$         &           \\
		\multicolumn{3}{l}{Permite generar reportes de cursos, progresos, etc.}                                 & $\checkmark$         & $\checkmark$         \\
		\multicolumn{3}{l}{Soporte de presupuestos (solicitud de profesores, coordinadores, etc.).}             & $\checkmark$         &           \\
		\multicolumn{3}{l}{Soporta competencias de aprendizaje del estudiante o Student Learning Outcomes.}     & $\checkmark$         &           \\
		\multicolumn{3}{l}{Soporte de alineación de competencias.}                                              & $\checkmark$         &           \\
		\multicolumn{3}{l}{E-portfolio y evaluación de proyectos.}                                              & $\checkmark$         & $\checkmark$         \\
		\multicolumn{3}{l}{Soporte de comunicación entre estudiantes y profesores (foro).}                      &           & $\checkmark$         \\
		\multicolumn{3}{l}{Soporte de evaluación sumativa y formativa.}                                         & $\checkmark$         &           \\
		\multicolumn{3}{l}{Software distribuido y desarrollado libremente.}                                     &           & $\checkmark$         \\
		\bottomrule
	\end{tabular}
	}
\end{table}

Los LMS permiten al estudiante a una capacitación flexible a distancia sin limitaciones de horarios con costos reducidos\citep{de2016design}. Además, como es una plataforma intuitiva, permite a las personas con nivel de conocimiento básico en informática un aprendizaje constante y actualizado a través de la interacción entre alumnos y profesores. En cambio, los AMS son más bien utilizados para evaluaciones de las competencias de los alumnos que como vínculo entre el alumno y el profesor mediante un aula virtual, y permiten la mejora continua del aprendizaje estudiantil.

Los LMS permiten la integración de competencias mediante herramientas externas, pero de una manera superficial en comparación a los AMS.
\section{Programas de estudio}
En términos generales, se puede definir un programa de estudios como una herramienta educativa que regula y ordena el proceso de enseñanza-aprendizaje a desarrollar en una unidad de aprendizaje determinada, orientando las actividades que profesor y alumno han de llevar a cabo para el logro de los objetivos planteados en dicha unidad, en relación con los objetivos del plan de estudios, de tal manera que el egresado concluya su carrera con el perfil deseado. En pocas palabras, es un esquema organizado de los contenidos situados dentro de una determinada unidad de aprendizaje\citep{lalor_ensuring_2017}.

El termino \enquote{unidad de aprendizaje} sustituye al de \enquote{asignatura} o \enquote{materia} que evocan los tradicionales cursos unidisciplinarios, generalmente teóricos y sobrecargados de información. Un programa resume las características de la unidad de aprendizaje, su contenido mínimo obligatorio, y sus objetivos, principalmente.

En la primera etapa de diseño curricular, se requiere la elaboración de la propuesta de los programas para su aprobación de parte de las autoridades con la colaboración de las academias y, de ser necesario, con la asesoría de externos de la unidad académica.

\subsection{Proceso de diseño curricular} \label{procesoCurricular}
Para aquellas personas ajenas al entorno educativo, en específico aquellas que no forman parte del proceso de diseño curricular puede ser un tanto complejo el proceso de creación y revisión de material curricular en las instituciones. En la actualidad, un formulario de creación o revisión de competencias, cursos, y programas debe pasar por una serie de evaluadores (figura \ref{flujo_autoridades}) que son los encargados de revisar y verificar que los datos sean válidos.
\begin{figure}
\centering
\includegraphics[scale=0.5]{Capitulos/MarcoTeorico/Imagenes/flujo_autoridades}
\caption{Flujo de creación y revisión de propuestas.}
  \label{flujo_autoridades}
\end{figure}

Hoy día el estado de California cuenta con un patrón de definición de cursos y programas donde la misma sirve como guía para el desarrollo de propuestas para material académico de las universidades. Dicho estándar además contiene una taxonomía de programas e indica cuál es el flujo para la revisión de las propuestas, donde todo es establecido en el PCAH\footnote{de sus siglas en inglés, Program and Course Approval Handbook, que significa en español manual de aprobación de cursos y programas.}.

El flujo inicia con el profesor o encargado del curso o programa, una vez completado pasa por la mesa de recepción donde se verifica que cumpla con el estándar estatal para luego pasar por la oficina departamental y la oficina del decano para su revisión de contenido. Una vez revisado y con el visto bueno de ambas oficinas pasa por una última revisión por parte de la oficina curricular para ser registrada en los sistemas de gestión curricular (figura \ref{course_creation_flow}).

Es un proceso que se hace con formularios en papel donde el profesor o encargado del curso o programa tiene que completar los campos requeridos para que el estado de California cuente al curso como válido. Dicho proceso tiene varias deficiencias como las que estaremos citando a continuación:
\begin{itemize}
	\item La creación o revisión puede tomar meses debido a los formularios que son completados a mano y requieren de revisión de varias oficinas.
	\item Es un flujo de una sola dirección, eso quiere decir que si es que una de las oficinas rechaza el formulario debe volver a iniciar el flujo.
	\item Se puede producir cuellos de botella en los diferentes puntos de revisión.
\end{itemize}

Una vez ya registrado en los sistemas de gestión curricular es accesible de manera pública para el uso de las universidades del estado de California. De esta manera, si una institución académica posee un sistema de gestión de evaluaciones y quiere incluir los cursos o programas válidos para el estado tiene ingresar los nuevos datos del sistema de gestión curricular uno a uno como se puede apreciar en la figura \ref{after_creation}.

\begin{figure}
\centering
\includegraphics[scale=0.5]{Capitulos/MarcoTeorico/Imagenes/course_creation_flow}
\caption{Flujo actual de diseño de cursos, programas y competencias.}
  \label{course_creation_flow}
\end{figure}

\begin{figure}
\centering
\includegraphics[width=125mm,scale=1]{Capitulos/MarcoTeorico/Imagenes/after_creation}
\caption{Esquema de tipos de computación en la nube.}
  \label{after_creation}
\end{figure}
\section{Sistemas de gestión curricular}
Un CMS \footnote{de sus siglas en inglés, Curriculum Management System, que significa sistema de gestión curricular} es un aplicación automatizada que apoya todo el proceso curricular, desde la planificación hasta la implementación y evaluación. Posee una interfaz única y cohesiva en línea que permite proponer, crear, evaluar, revisar, aprobar y aplicar cursos, programas y competencias \citep{harden2001amee}.

Curriculum es una mezcla sofisticada de estrategias educativas, contenido del curso, resultados de aprendizaje, experiencias educativas y evaluación. Esta visión amplia de un CMS se deriva del ambiente actual de educación elemental y secundaria que es impulsado por los estándares de contenido de cursos obligatorios federales y estatales, y la necesidad de auditorías continuas de currículo \citep{west2000technology}.

Los enfoques actuales del desarrollo de Curriculum por lo general gira en torno a los comités curriculares. Un comité curricular departamental comienza el proceso de desarrollo curricular considerando como entrada cualquiera de los aspectos mostrados en la figura \ref{diseno_curricular}.

\begin{figure}[H]
\centering
\includegraphics[width=125mm,scale=1]{Figuras/diseno_curricular}
\caption{Esquema de diseño curricular y sus variables determinantes.}
  \label{diseno_curricular}
\end{figure}

Dichos aspectos, sin embargo, sólo se consideran en un alto nivel de abstracción basado en la comprensión tácita de los miembros del comité sobre la disciplina.
\section{Aplicaciones web}
La necesidad de ejecución de operaciones complejas de manera remota y portable, tales como el uso de software sin depender de la potencia del hardware del usuario, la disponibilidad de uso en múltiples plataformas, y muchas otras, han guiado al desarrollo y evolución de las aplicaciones web. En las ciencias de la computación, una aplicación web es un software con arquitectura cliente-servidor donde el cliente (o interfaz de usuario) corre en un navegador web\citep{net_app_architecture}.

Los usuarios acceden a la aplicación utilizando un navegador web y no es necesario descargarse ningún tipo de software adicional, ya que la aplicación se ejecuta en el servidor. Para esclarecer el panorama, la lógica de la aplicación web se ejecuta remotamente, y el navegador web sólo se limita a la representación de los datos.

La evolución de las aplicaciones web ha crecido tan vertiginosamente que además de ofrecer una multitud de servicios, mejoran la UX a través de interfaces gráficas impactantes e intuitivas para los usuarios\citep{myers_past_2009}.
\section{Cloud computing}
La computación en la nube es una metáfora para abastecimiento y consumición de recursos de infraestructura. El nivel de abstracción ofrecida por la nube puede variar de hardware virtual a complejos sistemas distribuidos, debido a que los recursos están disponibles a demanda en enormes cantidades y pagados por uso \citep{wittig_amazon_2016}. Casi toda solución de infraestructura remota es basada hoy día con esta tecnología.

La infraestructura remota o nube puede ser manejada por una organización abierta para uso público, o puede ser privada, donde una nube que virtualiza y comparte la infraestructura con una sola organización o híbrida como son mostrados en la figura \ref{cloud_types}.

\begin{figure}[H]
\centering
\includegraphics[width=125mm,scale=1]{Figuras/cloud_computing_types}
\caption{Esquema de tipos de computación en la nube.}
  \label{cloud_types}
\end{figure}
\section{Software as a service}
Software as a service, más conocida como SaaS, es un paradigma de entrega de software donde la misma se encuentra alojada por lo general en la nube y se entrega como servicio a través de Internet a un gran número de usuarios a través de un modelo de suscripción. Se trata de un modelo de entrega de negocio en el que tanto la aplicación y el alojamiento son gestionados y compartidos con varias empresas, que alquilan y utilizan los servicios de aplicaciones de forma centralizada\citep{gupta_software_2014}.

El software como servicio es equivalente a servicios de proveedores externos que manejan todo mantenimiento, personalización, actualización y cobro para los servicios que su cliente utiliza de manera mensual o anual. El proveedor se encarga de ofrecer el software basado en un conjunto de códigos y datos definidos junto a las diferentes configuraciones para los diferentes clientes. Los suscriptores al servicio acceden a la aplicación con la sensación de que son los únicos usuarios de la aplicación. Sin embargo, los cambios de configuraciones como cambios de datos, flujo de trabajo, interfaz y el flujo de negocio son realizados de manera masiva y transparente para ellos\citep{kumar_cloud_2012}\citep{kang_web_2012}.

\subsection{Características}
Entre las características relevantes de las aplicaciones SaaS, y las que las remarcan como aplicaciones bien construidas, son construidas a la medida, escalables y soportan multitenancy\footnote{En español se conoce como multiples clientes inquilinos, se refiere cuando varios clientes pueden utilizar la misma instancia.}. No todas las aplicaciones SaaS comparten todas las características, pero las más comunes son las siguientes: [1][3][4][5]

\subsubsection{Configurabilidad}
Las aplicaciones con esta característica poseen el mismo código base y provee a instancias con múltiples opciones de configuración tal que cada cliente pueda tener sus propias configuraciones de software únicas y pueda tener la sensación de que es el único usuario en utilizar la aplicación. Esta es la clave del éxito para las aplicaciones SaaS. Por ejemplo, cada cliente puede tener configurado su sitio para que muestren fondos de pantallas o logos en las páginas de inicio de sesión o páginas principales que ellos especifican. Esta característica también puede ser llamada personalización de la aplicación.

\subsubsection{Multitenancy}
Con esta característica, una sola instancia de la aplicación corriendo puede servir a una cantidad de clientes. Los diferentes modelos de datos que están disponibles para soportar SaaS son las bases de datos aisladas, arquitectura de bases de datos aisladas compartidas y la construcción de datos. Utilizando la arquitectura multitenant los proveedores de aplicaciones SaaS pueden innovar de forma sencilla y ahorrar tiempo valioso gastado en mantener varias versiones de código deprecado y/o desactualizado.

\subsubsection{Escalabilidad}
Esta característica es la más complicada de agregar a una aplicación SaaS debido a su elevado costo. La escalabilidad es soportada por la virtualización, pero teniendo en cuenta el costo y el problema de complejidad muchas veces el desarrollador de la aplicación no se complica con esta característica.

\subsection{Ventajas}
Además, aparte de estas características, algunas ventajas que poseen las aplicaciones con arquitectura SaaS son las siguientes: [4][6][11]
\begin{itemize}
	\item Las aplicaciones SaaS pueden ser utilizadas por los usuarios por medio de sus navegadores Web. Esto ahorra en costos operacionales para el usuario junto con los requerimientos de hardware mínimo, por lo tanto, reduce el costo que el usuario necesita gastar en hardware.  Además de los costos de mantenimiento, costos de licencias del software también son minimizados.
	\item Mejor utilización de recursos, debido a que los recursos requeridos por las aplicaciones con arquitectura SaaS son mínimos.
	\item El avance de la tecnología Web permite que los proveedores de SaaS se ubiquen en el extranjero y también ofrezcan servicios de alta calidad. De esta manera, permite a los usuarios de la aplicación ahorrar en infraestructura.
	\item Usualmente, las soluciones SaaS residen en entornos en la nube donde son escalables y poseen integración con otras ventajas SaaS. En comparación al modelo tradicional, los usuarios no tienen que comprar otro servidor o software.
\end{itemize}
\section{Metodología ágil de desarrollo de software}
La metodología Ágil envuelve un enfoque para la toma de decisiones en los proyectos de software, que se refiere a métodos de ingeniería del software basados en el desarrollo iterativo e incremental, donde los requisitos y soluciones evolucionan con el tiempo según la necesidad del proyecto. Los métodos tradicionales, como Waterfall, pretenden ser capaces de modelar completamente el dominio del problema de entrada y luego esperar que se produzcan pequeños cambios (o inclusive ninguno)\citep{davis_agile_2015}. Los métodos ágiles asumen que el cambio es inevitable, por lo que abordan el desarrollo de software de tal manera a facilitar la adaptación de los nuevos requisitos mientras vayan surgiendo.

Las metodologías ágiles, en comparación a otras metodologías de desarrollo, ofrecen un modelo de diseño flexible que fomenta al desarrollo evolutivo. Los desarrolladores trabajan en pequeños módulos cada vez y la retroalimentación proveída por el cliente ocurre simultáneamente en el desarrollo. Además, la metodología puede ser bastante útil en situaciones donde los objetivos finales del proyecto no están claramente definidos donde los requisitos del cliente se clarificarán gradualmente a medida que el proyecto avance.

El uso de la metodología ágil como método de entregas de las características del módulo entra como requisito no funcional.

\subsection{Historias de usuario}
Las historias de usuario conforman la parte central de muchas metodologías de desarrollo ágil, tales como XP\footnote{de sus siglas en inglés, eXtreme Programming, que significa en español programación extrema}, Scrum, entre otras. Estas definen lo que se debe construir en el proyecto de software, tienen una prioridad asociada definida por el cliente de manera a indicar cuales son las más importantes para el resultado final. Son divididas en tareas y su tiempo es estimado por los desarrolladores.

Por lo general, se espera que una estimación de tiempo de cada historia de usuario se sitúe entre horas y el tiempo máximo de iteración. Estimaciones superiores a este tiempo máximo son indicativas de que la historia es muy compleja y debe ser dividida en varias historias.

Una historia de usuario es una representación de un requisito escrito en una o dos frases utilizando el lenguaje común del usuario\citep{davis_agile_2015}. Ellas son utilizadas para la especificación de requisitos acompañadas de las discusiones con aquellos y las pruebas de validación.

Cada historia de usuario debe ser limitada. La metodología estipula que las mismas deben ser escritas por los clientes. Son una forma rápida de administrar los requisitos sin tener que elaborar gran cantidad de documentos formales y sin requerir de mucho tiempo para administrarlos.

Las historias de usuario deben ser:
\begin{itemize}
    \item Independientes unas de otras. De ser necesario, combinar las historias dependientes o buscar otra forma de dividir las historias de manera que resulten independientes.
    \item Negociables. La historia en sí misma no es lo suficientemente explicita para considerarse un contrato, la discusión con los usuarios debe permitir esclarecerse y éste debe dejarse explicito bajo la forma de pruebas de validación.
    \item Valoradas por los clientes o usuarios. Los intereses de los clientes y de los usuarios no siempre coinciden, pero en todo caso, cada historia debe ser más importante para los clientes que para el desarrollador.
    \item Pequeñas. Las historias grandes son difíciles de estimar e imponen restricciones sobre la planificación de un desarrollo iterativo. Generalmente se recomienda la consolidación de historias muy cortas en una sola historia.
    \item Verificables. Las historias de usuario cubren requerimientos funcionales, por lo generalmente son verificables. Cuando sea posible, la verificación debe automatizarse, de manera que pueda ser verificada en cada entrega del proyecto.
\end{itemize}

Las iniciales de estas características, con sus nombres en inglés, forman la palabra INVEST, que significa “inversión”. Es porque toda historia de usuario, si se construye bien, es una inversión.

Al momento de implementar las historias, los desarrolladores deben tener la posibilidad de discutirlas con los clientes. El estilo sucinto de las historias podría dificultar su interpretación, podría requerir conocimientos de base sobre el modelo, o podría haber cambiado desde que fue escrita.

Cada historia de usuario debe tener en algún momento pruebas de validación asociadas, lo que permitirá al desarrollador, y más tarde al cliente, verificar si la historia ha sido completada. Como no se dispone de una formulación de requisitos precisa, la ausencia de pruebas de validación concertadas abre la posibilidad de discusiones largas y no constructivas al momento de la entrega del producto.

\subsection{Épicas}
Una épica es esencialmente una historia de usuario de un tamaño mucho mayor, siempre superior al tiempo de iteración máximo, y tiene como propósito el de asociar historias de usuario individuales relacionadas con un propósito de más alto nivel que cumplir. La misma es, por lo general, muy grande para que un equipo del proyecto pueda trabajar directamente sin partir en diversas historias de usuario\citep{cobb2015project}.

El uso de las épicas en proyectos de gran tamaño ayuda a organizar tareas complejas en un tipo de estructura para que la interrelación de historias de usuario esté bien entendida. Por lo tanto, el diseño de épicas en el proceso de desarrollo es fundamental antes de comenzar el proyecto.
\section{Interacción humano-computador}
En HCI definen la funcionalidad y la usabilidad de los sistemas que se desarrollan, donde la funcionalidad de un sistema es definida por un conjunto de acciones o servicios que son proveídas a los usuarios, sin embargo, el valor de la funcionalidad es verificada cuando es eficientemente utilizada por el usuario \citep{shneiderman_designing_2010}. La usabilidad de un sistema con cierta funcionalidad es el rango y grado por el cual el mismo puede ser utilizada de manera eficiente y adecuada para cumplir ciertas metas para ciertos usuarios. La eficiencia de un sistema es alcanzada cuando se cumple un balance entre la usabilidad y la funcionalidad \citep{nielsen_usability_2010}.

HCI es un diseño que debe producir un ajuste entre el usuario, la máquina y los servicios requeridos con el fin de lograr un balance óptimo entre la calidad y la eficiencia de los servicios.

La definición de la estrategia de UI\footnote{de sus siglas en inglés, User Interface, que significa en español experiencia de usuario.} es importante para una mejor usabilidad del sistema, donde este proceso debería comenzar antes que el diseño y desarrollo de las aplicaciones. Es la visión de una solución que necesite ser verificado con potenciales usuarios que prueben que necesite el mercado \citep{levy_ux_2015}.

% ESTADO DEL ARTE
\chapter{Estado del arte} % Main chapter title
En este capítulo realizamos un relevamiento y análisis de herramientas de informática educativa que intentan resolver problemas relacionados de forma cercana a los problemas planteados en este trabajo de tesis.

\label{capitulo3} % Change X to a consecutive number; for referencing this chapter elsewhere, use \ref{capitulo3}
\section{CurricUNET}
CurricUNET es una aplicación web diseñada para automatizar la emisión y aprobación del plan de estudios emitido por profesores y/o encargados de universidades norteamericanas; que incluyen programas, cursos y competencias.

En la misma se desarrollan propuestas de cursos y programas de estudio mediante formularios de la aplicación, con el objetivo de reemplazar solicitudes en papel que universidades utilizaban para emitir propuestas. Además, ofrece almacenamiento e información de plan de estudios históricos, activos y propuestos.

Todas las entradas, revisiones y reportes son accedidas por la web desde los navegadores. Posee un sistema de notificaciones integrado que permite al usuario un mejor seguimiento del progreso de las propuestas y cursos en revisión. Además, dispone de un control de versionamiento de cursos, planes de estudio y competencias.

Los usuarios del sistema pueden acceder a los reportes e historial de versiones de sus cursos y planes de estudio por lo que ayuda a una mejora continua del programa universitario.
\section{Courseleaf}
El módulo de Curriculum de CourseLeaf es una solución de gestión basada en la web, mejorando los procesos de profesores y del comité de Curriculum de al menos 70 instituciones.

Cuando el módulo de Curriculum de CourseLeaf se combina con su módulo de catálogo, colegios y universidades son capaces de gestionar y realizar un seguimiento de la información del programa, desde la propuesta hasta publicar en una aplicación integrada con facilidad.

CourseLeaf Curriculum es una solución de gran alcance con una completa funcionalidad. El software proporciona la generación de flujo de trabajo automático, notificaciones automáticas. Además, identifica todos los cursos, programas y departamentos que se ven afectados por los cambios propuestos en el inicio del proceso de propuesta y puede ayudar en la actualización con los cambios completados. El módulo de Curriculum de CourseLeaf se puede implementar con o sin su catálogo de Cursos.

El módulo de Curriculum dispone de cuatro componentes:
\begin{itemize}
	\item Un listado de cursos donde los usuarios pueden encontrar información sobre el curso. También pueden iniciar, editar y presentar propuestas de cursos.
	\item Una interfaz de los usuarios autorizados a opinar, anotar, rechazar, y aprobar las propuestas.
	\item Un formulario de curso, un formulario dinámico con campos inteligentes y el formato de respuesta que optimiza la experiencia del usuario.
	\item Un formulario de los programas, utilizando la misma funcionalidad que los formularios de curso para proponer y mantener la información y los requerimientos para los programas y grados.
\end{itemize}
\section{DECA: Curriculum Navigator}
DECA ofrece, mediante su módulo de Curriculum Navigator, una solución de desarrollo y administración de Curriculum.

Frecuentemente, descrito como complejo e ineficaz, los métodos tradicionales basados en papel de gestión de programas de estudio proporcionan una visibilidad limitada, lo que resulta en una visión restringida de las etapas implicadas en la creación, modificación y aprobación de planes de estudio. Además, busca eliminar esta complejidad al acelerar el desarrollo curricular y el proceso de aprobación.

Cuando se utiliza como módulo integrado de su Catálogo, denominado como Catalog Navigator, proporciona una solución que les permite iniciar el camino de aprobación de alguna carrera de grado. Asesores y administradores, por otra parte, utilizan tanto Curriculum Navigator y Catalog Navigator para desarrollar y proporcionar datos públicamente del plan de estudios para sus estudiantes actuales y futuros.

Curriculum Navigator ofrece:
\begin{itemize}
	\item El acceso a un repositorio de datos curricular.
	\item Editar, guardar y proponer planes de estudio o carreras de grado.
	\item Historial de cambios y de accesos de revisión.
	\item Seguimiento en tiempo real de propuestas abiertas.
\end{itemize}
\section{Comparación entre plataformas}
Una investigación para comprobar otros proyectos o productos con las mismas características propuestas, buscando innovación para el mercado es expuesta en la tabla \ref{relacion-sistemas}.

\begin{table}[H]
\centering
\resizebox{\columnwidth}{!}{%
	\begin{tabular}{lllccl}
		\toprule
		\multicolumn{3}{l}{Características}                                                & CurricUNET                       & CourseLeaf            & DECA         \\
		\midrule
		\multicolumn{3}{l}{Creación y versionamiento de competencias.}                     &                                  &                       &              \\
		\multicolumn{3}{l}{Creación y versionamiento de cursos.}                           & $\checkmark$                     & $\checkmark$          & $\checkmark$ \\
		\multicolumn{3}{l}{Creación y versionamiento de programas de estudio.}             & $\checkmark$                     & $\checkmark$          &              \\
		\multicolumn{3}{l}{Cumple los Estándares de códigos de California.} 			   & $\checkmark$                     &                       &              \\
		\multicolumn{3}{l}{Historial de versiones de competencias.}     			       & 			                      & 		              &  			 \\
		\multicolumn{3}{l}{Historial de versiones de cursos.}     			               & $\checkmark$                     & $\checkmark$          & $\checkmark$ \\
		\multicolumn{3}{l}{Historial de versiones de programas de estudio.}     		   & $\checkmark$                     &  			          & 			 \\
		\multicolumn{3}{l}{Reporte de Comparación entre versiones de cursos.}              & $\checkmark$                     &                       & $\checkmark$ \\
		\multicolumn{3}{l}{Soporta competencias de aprendizaje del estudiante.}            &                      			  &                       &              \\
		\multicolumn{3}{l}{Plantilla de flujo de trabajo customizable.}                    & $\checkmark$                     &                       &              \\
		\multicolumn{3}{l}{Permite asignar roles evaluadores en la aplicación.}            & $\checkmark$                     & $\checkmark$          &              \\
		\multicolumn{3}{l}{Permite asignar usuarios como colaboradores.}                   & $\checkmark$                     &                       &              \\
		\multicolumn{3}{l}{Sistema de alertas para colaboradores y evaluadores.}           & $\checkmark$                     & $\checkmark$          &              \\
		\multicolumn{3}{l}{Buzón de entrada para colaboradores y autoridades.} 			   & $\checkmark$                     &                       & $\checkmark$ \\
		\multicolumn{3}{l}{Soporte de correlatividades entre cursos.}                      & $\checkmark$ 					  &						  &              \\
		\multicolumn{3}{l}{Incluye un catálogo de cursos.}                   		   	   & $\checkmark$					  &	$\checkmark$		  & $\checkmark$ \\
		\multicolumn{3}{l}{Incluye un catálogo de programas de estudio.}                   & $\checkmark$					  &	            		  &              \\
		\multicolumn{3}{l}{Incluye un catálogo de competencias.}                   	       & 								  &						  & 			 \\
		\multicolumn{3}{l}{UX intuitiva y efectiva.}     			   					   &                                  & $\checkmark$          & $\checkmark$ \\
		\bottomrule
	\end{tabular}
}
\caption{Relación entre sistemas de gestión curricular.}
\label{relacion-sistemas}
\end{table}

\section{Relevancia del módulo curricular}
Frecuentemente, descrito como complejo e ineficaz, los métodos tradicionales basados en papel de gestión de programas de estudio proporcionan una visibilidad limitada, lo que resulta en una visión restringida de las etapas implicadas en la creación, modificación y aprobación de planes de estudio. Además, busca eliminar esta complejidad al acelerar el desarrollo curricular y el proceso de aprobación.

La importancia del módulo reside en la posibilidad de automatizar formularios y procesos que requieren la participación de personas ajenas al flujo de trabajo, para iniciar y validar propuestas de creación o revisión de cursos y programas. Sin embargo, hay alternativas que buscan solucionar la misma problemática pero no existe alternativa que pueda soportar el uso de competencias ni que pueda comunicarse con un sistema de gestión de evaluación basadas en competencias.

Como se habló en la sección \ref{procesoCurricular} de proceso curricular; una vez finalizado el proceso de diseño y revisión curricular de parte de las oficinas, se procede a publicar la nueva competencia, curso, o programa para que cada universidad tenga la información necesaria para ir cargando la misma en sus correspondientes sistemas. 

En el caso de las universidades comunitarias del estado de California, se utilizan los AMS para gestionar y evaluar las competencias de sus estudiantes. El proceso de registro de las nuevas entidades en los AMS es individual; eso quiere decir que un encargado del AMS debe encargarse de cargar uno por uno las nuevas entidades aprobadas y publicadas por el comité curricular.

En la propuesta de solución del capítulo \ref{capitulo5} hablaremos de como el proyecto final busca solucionar la problemática y unir los procesos de los cuales hablamos en el párrafo anterior.

\section{Sumario}
Se han expuesto las investigaciones hechas por el equipo de expertos en didáctica en modo de encuestas a los usuarios finales de las necesidades y funcionalidades que buscan o utilizan en las alternativas de CMS que utilizan en sus universidades. 

Además, se realizaron análisis como investigación de campo de las herramientas que buscan resolver la misma problemática, en una tabla comparativa de las diferentes opciones con las necesidades de los usuarios finales expuestas anteriormente y la relevancia del proyecto.

% PROPUESTA DE SOLUCION
\chapter{Propuesta de solución} % Main chapter title
En este capítulo se presenta en detalle la arquitectura del sistema planteada como solución, con un correspondiente estudio para la determinación de las herramientas para cada paso. Se expone además el flujo de datos y se presenta un diseño del mismo, donde queda gráficamente los pasos a seguir. Se incluyen también las técnicas de validación y evaluación que se deben utilizar.

\label{capitulo4} % Change X to a consecutive number; for referencing this chapter elsewhere, use \ref{capitulo5}
En esta sección se proveerá una descripción detallada de los modelos obtenidos; en donde, se explicarán las funciones y los principales componentes de cada parte. Se inlcuyen figuras ilustrativas de los resultados de clustering obtenidos, así como la explicación de los parámetros considerados para cada algoritmo.

Una vez que los requerimientos iniciales han sido fijados y aclarados se busca la manera de automatizar los procesos, investigar tecnologías, y metodologías que ayuden al equipo de desarrollo para entregar funcionalidades de manera iterativa y evolutiva. Durante este proceso se diseñan modelos donde se propone el módulo a ser desarrollado (Figura \ref{curriculum_model}) y se busca unir procesos separados del diseño curricular (Figura \ref{course_creation_flow}) con el flujo de agregar las competencias, cursos, y programas al AMS (Figura \ref{after_creation}).

\begin{figure}[]
\centering
\includegraphics[scale=0.5]{Capitulos/PropuestadeSolucion/Imagenes/curriculum_model}
\caption{Modelo propuesto del módulo curricular adherido a un sistema de gestión de evaluaciones basadas en competencias.}
  \label{curriculum_model}
\end{figure}

Actualmente, cuando el encargado de curso o programa completa su formulario y lo entrega en la mesa de recepción, la misma se encarga de verificar que los datos completados sean válidos y cumpla con el estándar de creación de cursos y programas. 

El módulo propuesto se encargará de automatizar dicho proceso sacando la mesa de recepción como iniciador del flujo de validación mediante un formulario web. Además, se agrega un nuevo tipo de formulario para las competencias de las universidades comunitarias de California.

Luego, pasa por las oficinas del departamento, del decano, y del comité curricular para sus correspondientes revisiones. Si es que una de las oficinas rechaza el formulario debe volver al inicio con el encargado del mismo para volver a ser completado, y una vez terminado puede volver a pasar a la oficina que rechazó el formulario sin necesidad de volver a iniciar todo el proceso de corrección.

Y finalmente, una vez que el comité curricular acepta el formulario se procede a generar los nuevos cursos, programas, o competencias en el AMS. También, otro proceso a ser automatizado por el módulo ya que hoy día dicha creación se hace de manera manual, como se aprecia en la figura \ref{after_creation}.

Se agrega también la funcionalidad de mensajes generados y notificaciones a los integrantes del flujo para evitar de esta manera los cuellos de botella con las revisiones, donde se notifican los pendientes y alertan trabajos en deuda.

\section{Modelo de arquitectura de módulo}

El proyecto final (figura \ref{arquitectura}) fue diseñado como módulo de un AMS utilizado en universidades del estado de California.

\begin{figure}[]
\centering
\includegraphics[scale=0.6]{Capitulos/PropuestadeSolucion/Imagenes/arquitectura}
\caption{Arquitectura del módulo curricular.}
  \label{arquitectura}
\end{figure}

El AMS utilizado como base tiene una trayectoria de largos años de uso en varias universidades. Utiliza MySQL como motor de base de datos, Java como lenguaje de programación para la lógica de la aplicación y como conector a la base de datos se utiliza, y Bootstrap y JQuery para la interfaz de usuario. 

El proyecto final va a utilizar la misma base de datos, lenguaje de programación, y Bootstrap como requisitos no funcionales para el módulo curricular. Se optó cambiar JQuery a AngularJS debido a que al contar con un modelo MVC\footnote{de sus siglas en inglés, Model View Controler, que significa en español modelo vista controlador.} en la capa de presentación desde el comienzo resultó muy atractivo para acelerar el ritmo de trabajo y poder comenzar a implementar interfaces más complejas sin tener que preocuparse por las cuestiones más triviales que Angular maneja con directivas ya definidas como el uso de \enquote{data binding}, además, cuenta con una cantidad de documentación de parte de la comunidad que lo hacía aún más atractivo.
\section{Requerimientos no funcionales} \label{reqnofuncional}
Se ha realizado una breve revisión de las tecnologías y abordajes brindadas como requerimientos funcionales. Sin embargo, al utilizar un abordaje ágil, la validación de las decisiones tecnológicas depende en última instancia de la validación del proceso de desarrollo realizada por expertos y usuarios.

Las decisiones de tecnologías para el módulo de gestión curricular fueron basadas en los conocimientos adquiridos de los miembros del equipo de desarrollo, con el fin de optimizar tiempos utilizados en curvas de aprendizaje. Sin embargo, se hicieron análisis previos a su uso para corroborar que dichas tecnologías cumplen con el propósito de desarrollo.

\subsection{Java}
La elección del lenguaje de programación, establecida por la organización, fue utilizada como lenguaje para la lógica del módulo curricular ya que facilita la integración con el código ya existente. El equipo de desarrollo posee conocimiento en este lenguaje de programación o en lenguajes orientados a objetos, por lo que se aprovechó el tiempo que pudo haber sido utilizado en curvas de aprendizaje del lenguaje para investigar buenas prácticas para el proyecto.

Java fue diseñado para alcanzar los desafíos del desarrollo de aplicaciones en el contexto de ambientes heterogéneos y distribuidos en red \citep{eckel_thinking_2006}. Lo más importante de solucionar entre estos desafíos era la entrega segura de aplicaciones que consumen lo mínimo de recursos del sistema, correr en cualquier plataforma de hardware y/o software, y que pueda ser extendido de manera dinámica \citep{sierra2005head}.

Operar en múltiples plataformas con redes heterogéneas invalida la arquitectura tradicional de distribución de binarios, \enquote{release}, actualización, parcheo, etc \citep{arnold2005java}.

Hoy día, Java es uno de los lenguajes de programación más utilizados a nivel mundial debido a su portabilidad y evolución con el paso del tiempo, como se observa en la Figura \ref{graph_java}. Además, al ser un lenguaje de programación multiplataforma cualquier proyecto se puede desarrollar en cualquier sistema operativo o plataforma para luego ser levantada en el servidor independiente a su plataforma. 

El lenguaje es reconocido por ser intuitivo a la hora de programar, altamente portátil y portable \citep{hunt2011java}. Por otra parte, la comunidad es uno de los fuertes de Java. Además, en la Web hay una gran cantidad de foros y librerías de la comunidad para resolver diferentes problemáticas de los desarrolladores.

\begin{figure}[]
\centering
\includegraphics[width=125mm,scale=1]{Figuras/tecnologias/java}
\caption{Gráfico de uso de lenguajes de programación con respecto al tiempo \citep{tiobe_programming}.}
  \label{graph_java}
\end{figure}

El lenguaje Java proporciona un nivel de rendimiento adecuado para la plataforma donde se aloja el AMS.

Dicho uso del lenguaje fue validado durante el proceso de desarrollo al poder resolver los criterios de aceptación de las historias de usuario.

\subsection{MySQL}
Para ayudar a que el AMS sea adaptable y fácil de mantener, se extenderá la base de datos ya utilizada por el AMS. Por lo tanto, la elección de la base de datos del módulo queda a criterio de la organización y como requisito no funcional para el proyecto final. El sistema de base de datos que se utiliza para el sistema de evaluación de competencias es MySQL, por ser un sistema open source y con una de las comunidades más grandes entre las bases de datos existentes.

MySQL es el sistema de base de datos open source más popular disponible. Es particularmente eficaz para sitios web públicos que requieren base de datos rápidas y estables \citep{dyer2015learning}. Para añadir, acceder y procesar datos almacenados en una base de datos informática se necesita de un sistema de administración de Base de Datos como MySQL Server. 

Como las computadoras son muy buenas manejando grandes cantidades de datos, los sistemas de administración de base de datos juegan un rol principal en la computación como utilidades autónomas o como partes de otras aplicaciones.

Para representar los datos que se almacenan, utiliza el modelo lógico de datos relacional donde guarda sus datos en tablas separadas, antes que consolidar todos los datos en un solo lugar. La estructura de la base de datos está organizada en archivos físicos optimizados para mayor velocidad \citep{ronstrom2004mysql}. 

El modelo lógico con objetos tales como bases de datos, tablas, vistas, filas, y columnas, ofrecen un ambiente de programación flexible donde se establecen reglas que gobiernan las relaciones entre las de los distintos tipos de campos, tales como uno a uno, uno a muchos, únicos, requeridos, u opcionales y punteros entre diferentes tablas.

En las siguientes figuras \ref{graph_db_1} y \ref{graph_db_2} podemos observar que MySQL es un sistema de manejo de base de datos muy utilizado por la comunidad y va en aumento de popularidad hasta casi alcanzar a uno de los gigantes que es Oracle.

\begin{figure}[H]
\centering
\includegraphics[width=125mm,scale=1]{Figuras/tecnologias/rank_db_1}
\caption{Gráfico que muestra el puntaje de uso de motores de base de datos \citep{db_engines_page}.}
  \label{graph_db_1}
\end{figure}

\begin{figure}[H]
\centering
\includegraphics[width=125mm,scale=1]{Figuras/tecnologias/rank_db_2}
\caption{Gráfico de uso de motores de base de datos con respecto al tiempo \citep{db_engines_page}.}
  \label{graph_db_2}
\end{figure}

El motor de base de datos es otro requerimiento no funcional, ya que es otra tecnología utilizada para la aplicación base. Sin embargo, se hizo un breve análisis de vulnerabilidades para verificar que el uso de MySQL como base de datos relacional podría ser efectivo para el proyecto final, además, los desarrolladores están familiarizados con la misma.

\subsection{Amazon Web Services}
Amazon Web Services, o más conocida como AWS, es una plataforma de servicios web que ofrece soluciones de procesado, almacenamiento y redes en diferentes capas de abstracción. Se pueden usar estos servicios para hospedar sitios web, correr aplicaciones complejas y para minería de grandes cantidades de datos \citep{wittig2015amazon}. 

La interfaz web puede ser manejada por máquinas o por usuarios mediante de una interfaz de usuario gráfica. Los servicios más utilizados son EC2 en la cual ofrece servidores virtuales, y S3 que es utilizada para almacenamiento.

AWS es una nube pública, donde tiene las siguientes clasificaciones como nube:
\begin{itemize}
	\item \textbf{Infraestructura como Servicio:} más conocida como IaaS, ofrece recursos fundamentales como procesamiento, almacenamiento, y capacidades de servicios, utilizando servidores virtuales tales como Amazon EC2, Google Compute Engine y Microsoft Azure en las máquinas virtuales.
	\item \textbf{Plataforma como Servicio:} más conocida como PaaS, provee plataformas para desplegar aplicaciones customizadas a la nube, tales como AWS Elastic Beanstalk, Google App Engine, y Heroku.
	\item \textbf{Software como Servicio:} más conocida como SaaS, combina infraestructura y software corriendo en la nube, incluyendo aplicaciones de oficina como Amazon WorkSpaces, Google Apps for Work, y Microsoft Office 365.
\end{itemize}

En la figura \ref{graph_cloud} se puede mostrar como AWS lidera entre las alternativas del mercado para soluciones de computación en la nube. Luego, lo sigue Azure de Microsoft como siguiente alternativa más utilizada. AWS y Azure son líderes en opciones de cloud debido a su constante innovación en el mercado como se puede apreciar en la figura \ref{gartner_cloud}.

\begin{figure}[H]
\centering
\includegraphics[width=125mm,scale=1]{Capitulos/PropuestadeSolucion/Imagenes/rank_cloud}
\caption{Ranking de compañías que brindan servicios de cloud computing \citep{statista_ranking}.}
  \label{graph_cloud}
\end{figure}

\begin{figure}[H]
\centering
\includegraphics[scale=0.5]{Capitulos/PropuestadeSolucion/Imagenes/gartner_cloud}
\caption{Figura de Gartner que muestra las alternativas de cloud computing \citep{gartner_webpage}.}
  \label{gartner_cloud}
\end{figure}

El AMS se encuentra alojado en la plataforma de servicios AWS. Por lo tanto, se utilizaron los servidores ya en línea para correr el módulo curricular para la aplicación.

\subsection{Git}
Como un equipo de desarrollo requiere de un sistema de control de versiones eficiente, se utilizó Git por ser el líder en VCS \citep{loeliger2012version}. Para mejorar la integración del código que produce el equipo, la organización utiliza un repositorio en \enquote{GitHub}\footnote{Plataforma de desarrollo colaborativo de software para alojar proyectos utilizando el sistema de control de versiones Git.}. Además, el uso de un sistema de versionamiento permite minimizar y optimizar el tiempo de unión de módulos que van agregando o actualizando los miembros del equipo de desarrollo.

El control de versiones es un sistema que registra los cambios realizados sobre un archivo o conjunto de archivos a lo largo del tiempo, de esta forma permite recuperar versiones específicas más adelante \citep{chacon2014pro}.

Git modela sus datos más como un conjunto de instantáneas de un pequeño sistema de archivos. Cada vez que se confirma el cambio de un archivo, o se guarda el estado de un proyecto se hace una foto del aspecto de todos los archivos en ese momento y guarda una referencia a esa instantánea. Para ser eficiente, si los archivos no se han modificado, Git no almacena el archivo de nuevo, solo un enlace al archivo idéntico anterior que ya tiene almacenado. Este comportamiento se puede observar en la figura \ref{graph_git}.

\begin{figure}[H]
\centering
\includegraphics[width=125mm,scale=1]{Figuras/tecnologias/git_over_time}
\caption{Flujo de versiones de Git \citep{chacon2014pro}.}
  \label{graph_git}
\end{figure}

Uno de los fuertes importantes de Git como VCS\footnote{de sus siglas en inglés, Version Control System, que significa en español sistema de control de versionamiento.} es su integridad, debido a que toda versión es verificada mediante una suma de comprobación\footnote{También conocida como checksum.} antes de ser almacenada, y es identificada a partir de ese momento dicha suma. Esta suma es utilizada para volver a un estado anterior del repositorio en caso de ser necesario o ver cuáles fueron los cambios que se realizaron en la porción de código trabajado o más conocido como commit \citep{chacon2014pro}.

\subsection{Spring}
El AMS cuenta con un framework propio desarrollado internamente, el cual esta deprecado y uno nuevo en el cual se estuvo trabajando para los nuevos módulos que se fueron desarrollando. El mismo es Spring MVC y fue utilizado para la parte lógica de la aplicación.

Spring es un framework que facilita el desarrollo de aplicaciones escritas en Java. El propósito de Spring es manejar la infraestructura de las aplicaciones utilizando el método de inversión de control. Es por ello, que el programador se encargará de programar la lógica de negocio usando objetos simples de Java o POJOs (Plain Old Java Objets) y Spring se encargará de añadir las capacidades de empresa o J2EE a la aplicación \citep{bauer2005hibernate}.

Spring tiene las siguientes características:
\begin{itemize}
	\item Simplicidad y acoplamiento débil donde permite programar Java de manera sencilla. Busca ser simple y se basa en la inyección de dependencias para obtener un acoplamiento débil.
	\item Funciona como contenedor ya que gestiona el ciclo de vida de los objetos y como se relacionan entre ellos. Proporciona una gran infraestructura que permite que el programador se dedique a la lógica de la aplicación.
	\item Ligero porque es muy rápido en tiempo de procesamiento y no es invasivo a la hora de programar.
	\item Orientado a aspectos, lo que permite facilitar una capa de servicios que son ideales para este tipo de programación como auditoría, o gestión de transacciones.
\end{itemize} 

Spring se utiliza en el proyecto como framework para toda la aplicación, por lo tanto, usar Spring es también un requerimiento no funcional.

\subsection{AngularJS}
Para empezar a trabajar con la interfaz de usuario se hizo un análisis previo de las ventajas entre Frameworks como JQuery y AngularJS, debido a que la organización que brinda el AMS dio total libertad a la hora de elegir cual utilizar. Por lo tanto, la elección del framework Javascript del módulo de Curriculum entra como decisión de diseño para el proyecto final. 

Con la llegada de los framework de Javascript, tales como JQuery, las páginas web ya no tenían la necesidad de volver a renderizar las páginas cada vez que se necesitaba información del servidor, ya que con la aparición de las llamadas asíncronas se ha logrado mejorar la experiencia del usuario \citep{ruebbelke2015angularjs}.

JQuery ha hecho un excepcional trabajo al proveer de herramientas que manipulen el DOM de una página, pero no ofrece una guía real de cómo organizar el código en la estructura de la aplicación. Ante la desesperada búsqueda de escribir aplicaciones grandes y fáciles de mantener en Javascript ha dado a luz a un renacimiento de frameworks de Javascript, entre ellos se encuentra el framework de Google más conocida como AngularJS.

AngularJS es un framework de aplicaciones web de código abierto que ofrece a un desarrollador una base estable de código con una comunidad enorme y un entorno rico de librerías hechas por la comunidad \citep{darwin2013angularjs}.
\section{Diseño}
El módulo como proyecto de desarrollo enfocado a la metodología Ágil se encuentra dividido en varias épicas para partir en las funcionalidades.

Una épica se encuentra dividida en varias historias de usuario, donde las historias de usuario tienen el propósito de entregar valores de negocio al cliente en un periodo establecido de 2 semanas como sprint. Estas historias de usuario pueden ser a la vez divididas buscando la simplicidad de las historias donde cada una debe seguir la práctica INVEST de la metodología Ágil.

Cada historia puede estar compuesta de tareas que tienen como propósito servir al desarrollador como recordatorio de algunas labores pendientes a la hora de desarrollar la historia. Cada tarea debía tener un encargado, pero eso no significaba que esa persona debía hacer sola la implementación.

\begin{figure}[H]
\centering
\includegraphics[width=125mm,scale=1]{Capitulos/PropuestadeSolucion/Imagenes/epic_diagram}
\caption{Diagrama de definición de épicas en la metodología ágil}
  \label{epic}
\end{figure}

El PO\footnote{de sus siglas en inglés, Product Owner, que significa en español dueño del producto.} se encarga de la creación de épicas e historias de usuario, en caso de que la historia sea muy grande para terminar en un solo sprint o iteración se vuelve a partir en historias más pequeñas. 

En el caso de estudio, cada sprint consta de 2 semanas de trabajo, donde los desarrolladores como equipo se comprometen a entregar cierto valor de negocio que ellos estiman poder terminar en dicho periodo. Sin embargo, en caso de que el equipo considere que la totalidad de historias no podrán ser entregadas antes de que termine el periodo se pasa al siguiente sprint o se achica la historia minimizando los criterios de aceptación y los restantes se agregan en otra historia de usuario para las siguientes iteraciones.

Cada equipo tiene un líder, donde cada líder tiene como rol ser la brecha que une al PO con los desarrolladores. El PO se reúne con el líder de cada equipo para verificar las prioridades de las historias de usuario que están pendientes en el backlog\footnote{Bolsa de historias de usuarios pendientes.}. 

En la figura \ref{workflow} se puede apreciar el ciclo de vida de las historias de usuario, donde una vez que es creada pasa al estado de \enquote{TODO}, que quiere decir que está pendiente a ser desarrollada. Una vez que un miembro del equipo de desarrollo comienza una historia o tarea pasa al estado de \enquote{IN PROGRESS} y cuando termina pasa al estado de \enquote{UNDER REVIEW}. 

En dicho estado se revisa la funcionalidad mediante validaciones de parte de los miembros del equipo de desarrollo y de parte del equipo de expertos en dominios de didáctica en universidades norteamericanas incluyendo a un PhD en educación, donde se deben cumplir los criterios de aceptación para que pase al estado de \enquote{CLOSED} que quiere decir que se terminó y que la historia fué aprobada.  

En caso de que la historia no consiga cumplir los criterios de aceptación correspondientes durante la validación se considera que la historia no está terminada y que debe pasar al estado de \enquote{REOPEN}, en este estado se puede pasar ya sea desde el estado \enquote{UNDER REVIEW} o si ya está en el estado \enquote{CLOSED}.

Cualquier otro problema o error de código que tenga la nueva funcionalidad se debe crear un ticket de error o bug especificando como reproducir el problema y el comportamiento esperado. En caso de no poder reproducir este comportamiento se pide más información al respecto o pasa al estado de \enquote{CLOSED} en caso de que el comportamiento ya no se pueda reproducir.

\begin{figure}[H]
\centering
\includegraphics[width=125mm,scale=1]{Figuras/workflow}
\caption{Flujo de desarrollo de historias de usuario.}
  \label{workflow}
\end{figure}

Al inicio del diseño de la aplicación se llevará a cabo una serie de diseños de funcionalidad y usabilidad que llevar a la mejor experiencia de uso del módulo de gestión curricular, donde dichos diseños serán validados por el equipo en los Estados Unidos antes de iniciar el desarrollo.

\subsection{SCRUM}
Scrum es un proceso en el que se aplican de manera regular un conjunto de prácticas para trabajar colaborativamente, en equipo, y obtener el mejor resultado posible de un proyecto. Estas prácticas se apoyan unas a otras y su selección tiene origen en un estudio de la manera de trabajar de equipos altamente productivos.

En Scrum se realizan entregas parciales y regulares del producto final, priorizadas por el beneficio que aportan al PO. Por ello, Scrum está especialmente indicado para proyectos en entornos complejos donde se necesita obtener resultados con el mínimo esfuerzo y los requisitos son cambiantes o poco definidos. Además, en dichos ambientes la innovación, la competitividad, la flexibilidad, y la productividad son fundamentales.

Scrum también se utiliza para resolver situaciones en que no se está entregando al cliente lo que necesita, cuando las entregas se alargan demasiado, los costes se disparan o la calidad no es aceptable, cuando se necesita capacidad de reacción ante la competencia, cuando la moral de los equipos es baja y la rotación alta, cuando es necesario identificar y solucionar ineficiencias sistemáticamente o cuando se quiere trabajar utilizando un proceso especializado en el desarrollo de producto. 

\subsection{Proceso}
En Scrum un proyecto se ejecuta en bloques temporales cortos y fijos que los conocemos como sprints o iteraciones. Estas iteraciones por lo general duran 2 semanas aunque en algunos equipos son de 3 y hasta 4 semanas, límite máximo de feedback y reflexión\citep{davis_agile_2015}. Cada iteración tiene que proporcionar un resultado completo, un incremento de producto final que sea susceptible de ser entregado con el mínimo esfuerzo al cliente cuando lo solicite.

El proceso parte de la lista de objetivos o requisitos priorizada del producto, que actúa como plan del proyecto. En esta lista el cliente prioriza los objetivos balanceando el valor que le aportan respecto a su coste y quedan repartidos en sprints y entregas.

\begin{figure}[H]
\centering
\includegraphics[width=125mm,scale=1]{Figuras/flujo_scrum}
\caption{Flujo de la técnica SCRUM.}
  \label{flujo_scrum}
\end{figure}

\subsection{Planificación de iteraciones}
El primer día de la iteración se realiza la reunión de planificación de la iteración y consta de dos partes:
\begin{itemize}
    \item \textbf{Selección de requisitos} (4 horas máximo) – El PO presenta al equipo la lista de requisitos priorizada del producto o proyecto. El equipo pregunta al PO las dudas que surgen y selecciona los requisitos prioritarios que se compromete a completar en la iteración, de manera que puedan ser entregados en caso de ser solicitados.
    \item \textbf{Planificación de la iteración o sprint} (4 horas máximo) – El equipo elabora la lista de tareas de la iteración necesarias para desarrollar los requisitos a que se ha comprometido. La estimación de esfuerzo se hace de manera conjunta y los miembros del equipo se asignan las tareas.
\end{itemize}

\subsection{Ejecución del Sprint}
El equipo realiza una reunión diaria (15 minutos aproximadamente). Cada miembro del equipo inspecciona el trabajo que el resto está realizando (dependencias entre tareas, progreso hacia el objetivo de la iteración, obstáculos que pueden impedir este objetivo) para poder hacer las adaptaciones necesarias que permitan cumplir con el compromiso adquirido. En la reunión cada miembro del equipo responde a tres preguntas:
\begin{itemize}
    \item ¿Qué he hecho desde la última reunión diaria?
    \item ¿Qué voy a hacer a partir de este momento?
    \item ¿Qué impedimentos tengo o voy a tener?
\end{itemize}
Durante la iteración el Scrum Master se encarga de que el equipo pueda cumplir con su compromiso y de que no se merme la productividad del equipo. Además, elimina los obstáculos que el equipo no puede resolver por sí mismo.

Durante el sprint, el PO junto con el equipo refinen la lista de requisitos para prepararlos para los siguientes sprints y, si es necesario, cambian o vuelven a planificar los objetivos del proyecto para maximizar la utilidad de lo que se desarrolla y el retorno de inversión.

\subsection{Inspección y adaptación}
El último día de la iteración se realiza la reunión de revisión del sprint la cual consta de dos partes:
\begin{itemize}
    \item \textbf{Demostración} (3 horas aproximadamente) – El equipo presenta al PO los requisitos completados en la iteración, en forma de incremento de producto preparado para ser entregado con el mínimo esfuerzo. En función de los resultados mostrados y de los cambios ocurridos en el contexto del proyecto, el PO realiza las adaptaciones necesarias de manera objetiva, ya desde la primera iteración, volviendo a planificar el proyecto.
    \item \textbf{Retrospectiva} (1 hora) - El equipo analiza cómo ha sido su manera de trabajar y cuáles son los problemas que podrían impedirle progresar adecuadamente, mejorando de manera continua su productividad. El Scrum Master se encargará de ir eliminando los obstáculos identificados.
\end{itemize}

% DISEÑO Y DESARROLLO DE LA APLICACION
\chapter{Desarrollo de la aplicación} % Main chapter title
\label{capitulo6} % Change X to a consecutive number; for referencing this chapter elsewhere, use \ref{capitulo6}

\section{Análisis de herramientas utilizadas y posibles potenciales para el desarrollo del módulo}
La primera etapa del proyecto fue realizar una encuesta a los posibles integrantes del nuevo equipo de desarrollo donde se verifican las capacidades adquiridas en cuanto a lenguajes de programación y tecnologías utilizadas, como asi también de los conocimientos de dominio de la aplicación.
Dicha encuesta tiene como propósito de permitir una mejor organización de los miembros de equipos para permitir de esta manera una distribución eficaz de conocimientos y dominio de la aplicación para resolver las diferentes posibles problemáticas que podría afectar al módulo currilar.
Una vez formado lo que sería el equipo de desarrollo se procedió a hacer análisis de las herramientas que podrían resolver la problemática entre ellas las que eran tomadas como requerimientos no funcionales para el módulo de gestión curricular.
\section{Diseño de modelo de datos para versionamiento}
Al iniciar el proceso de desarrollo se debía iniciar un \enquote{spike} para buscar la manera de modelar los datos y tablas ya existentes en el sistema de competencias, cursos, y programas. 

Un \enquote{spike} es un término que se utiliza en el desarrollo Ágil para incluir una tarea en un sprint que no pertenece necesariamente a una historia de usuario. Es necesario ya que sirve para incluir tareas que ayudan de alguna manera en el futuro desarrollo de las historias de usuario, pero que no implica directamente un incremento al producto que se está desarrollando ni un valor de negocio para el cliente\citep{leffingwell2010agile}.

Se diseñó de una manera que fuera lo más general posible en caso de que cualquier otra entidad se decida versionar en el futuro. En futuras iteraciones se llegó a la conclusión que la idea fué acertada, debido a que se abarcaría el versionamiento de evaluaciones de manera automática por el sistema (sección \ref{versionamiento_encadenado}) y la misma contiene una de las entidades versionables que es la de competencias.

\begin{table}[H]
\centering
\resizebox{\columnwidth}{!}{%
\begin{tabular}{@{}lllll@{}}
\toprule
Historias de usuario                                                        & HE        & HC         & PH    & Sprints               \\ \midrule
Diseño del modelo de versionamiento para competencias, cursos, y programas. & \multicolumn{1}{c}{61} & \multicolumn{1}{c}{61} & \multicolumn{1}{c}{5} & \multicolumn{1}{c}{1} \\ \bottomrule
\end{tabular}
}
\caption{Historias de usuario para el diseño de modelo de datos para versionamiento}
\label{epic:1}
\end{table}


\subsection{Diseño del modelo de versionamiento para competencias, cursos y programas}
Al iniciar con las historias de versionamiento, se definió una tarea que tenía como propósito principal el diseño de una lógica de negocios que permita adaptar las tablas existentes de las competencias, evaluaciones, cursos, y programas para que soporten una revisión o versionamiento de sus registros.

Como resúmen de actividades del modelo desarrollado (figura \ref{version_model}) se puede resaltar lo siguiente:
\begin{itemize}
	\item Cada tabla de cualquier entidad posee un identificador único. Las tablas entidades versionables son las de competencias, cursos, programas, y evaluaciones.
	\item Se decidió agregar una nueva columna \enquote{entity_atid} que tiene como propósito el de apuntar al origen de la versión. Por ejemplo; si el usuario crea un nuevo curso para el año lectivo, este curso tiene su identificador \enquote{course_id} y su \enquote{course_atid} apuntando a su mismo identificador por ser el origen para las versiones posteriores. Luego, se crea una nueva versión para el año posterior, esta nueva versión tiene su propio identificador pero su campo denominado como \enquote{course_atid} que apunta al primer curso creado u origen.
	\item Para hacer más sencilla la búsqueda de competencias, cursos, programas o evaluaciones actuales se agregó un campo a cada tabla identificando los actuales. Este campo denominado \enquote{is_current} o “es actual” es una bandera que indica la validez del registro.
	\item Además de registrar el origen, se registra la versión previa o de donde parte el registro con el campo \enquote{previous_entity_id}.
	\item Como cada registro de cualquier tabla ahora tiene un periodo de validez, se diseñaron tablas de relación entre cada tabla y la tabla de periodos lectivos denominada como \enquote{calendar}. Por ejemplo; \enquote{slo_term_rel} para las competencias, \enquote{new_course_term_rel} para los cursos, \enquote{asmt_term_rel} para las evaluaciones y \enquote{credential_term_rel} para los programas.
\end{itemize}

\begin{figure}
\centering
\includegraphics[width=125mm,scale=1]{Capitulos/DesarrollodelaAplicacion/Imagenes/version_model}
\caption{Modelo de datos para el versionamiento de competencias, cursos y programas.}
  \label{version_model}
\end{figure}
\section{Flujo de trabajo para el versionamiento de competencias}

% Please add the following required packages to your document preamble:
% \usepackage{booktabs}
\begin{table}[H]
\centering
\caption{Historias de usuario para el flujo de trabajo para el versionamiento de competencias}
\label{epic:3}
\begin{tabular}{@{}lllll@{}}
\toprule
Historias de usuario                & HE  & HC  & PH & Sprints \\ \midrule
Versionamiento de competencias      & 148 & 170 & 13 & 3       \\
Flujo de trabajo simple             & 78  & 78  & 8  & 1       \\
Aprobar pasos del flujo de trabajo  & 44  & 44  & 5  & 1       \\
Rechazar pasos del flujo de trabajo & 52  & 53  & 5  & 1       \\ \bottomrule
\end{tabular}
\end{table}

\subsection{Versionamiento de competencias}
Esta historia de usuario tenía como descripción: \enquote{\textit{Como encargado del sistema de gestión de competencias, me gustaría ser capaz de versionar competencias con la finalidad de que se puedan redefinir competencias con el paso del tiempo, sin perder datos de corrección de las mismas}}.

Algunas tareas que se definieron en la historia de usuario son las siguientes:
\begin{itemize}
	\item Investigar y diseñar el versionamiento de las competencias. La misma fue desarrollada de manera en que toda competencia versionada apunta al origen y el origen se apunta a sí mismo, de esta manera se puede saber la familia de versiones de una competencia.
	\item Actualizar todos esos lugares de la aplicación que listan las competencias, donde solamente deberían traer las competencias actuales.
	\item Manejar la distribución de competencias a periodos futuros, de manera que una competencia no pueda ser distribuida a periodos en las que no tiene validez.
	\item Diseñar y mantener pruebas automatizadas.
\end{itemize}

Fue desarrollado durante tres iteraciones con un total de 170 horas cargadas en el sistema, debido a la complejidad a la hora de migrar los datos ya existentes de todas las universidades y por la cantidad de servicios que debían ser modificados.

\subsection{Flujo de trabajo simple}
En la siguiente historia de usuario inicia el proceso de creación de flujos de trabajo donde las plantillas de los mismos pueden ser creados, editados, y eliminados por el administrador encargado de la aplicación de cada universidad. Para esta historia se debe diseñar y desarrollar las plantillas de manera que el administrador pueda agregar los diferentes pasos del flujo de trabajo si así lo decide en el futuro. Inicialmente se considera un solo paso para la iteración inicial de desarrollo.

Esta historia de usuario tenía como descripción: \enquote{\textit{Como coordinador del AMS, me gustaría ser capaz de crear flujos de trabajo simples para administrar la aprobación de revisiones de competencias y que se pueda tener un mejor manejo de las creaciones y aprobaciones de las mismas en el campus}}.

La historia tiene los siguientes criterios de aceptación:
\begin{itemize}
	\item Diseñar e implementar plantillas de flujo de trabajos simples para creación y revisión de todos los niveles de competencias.
	\item Diseñar e implementar un flujo de trabajo simple sin aprobación por parte de evaluadores, donde el iniciador del flujo puede revisar y aprobar su propio formulario.
	\item La plantilla de flujo de trabajo simple debe soportar el uso de pasos personalizados.
	\item El que inició el flujo es el único que puede llenar los campos del formulario.
\end{itemize}

El usuario debe ser capaz de agregar pasos personalizados para la plantilla de flujo de trabajo de la institución. Estos pasos personalizados son pasos que puede diseñar el usuario, donde puede colocar una pregunta como título y por cada título tiene un campo que puede llenar el usuario. Por lo general, un paso personalizado puede tener una o más preguntas definida por el usuario.

En los mockups entregados para el desarrollo se contemplan trabajos futuros donde cada paso tiene que ser aprobado por un rol del AMS, donde cualquier usuario con dicho rol puede aprobar o rechazar el flujo de trabajo con solo rechazar uno de los pasos. 

Además, se da inicio al desarrollo de plantillas de flujos de trabajo con las competencias, donde se podía asignar un tipo de flujo para cada plantilla ya sea de creación o versionamiento de los diferentes niveles de competencias.

Como las plantillas era una funcionalidad conocida y utilizada en otra parte de la aplicación, se imitó el comportamiento de la misma utilizando las mismas tablas para el almacenamiento de los datos en la base de datos relacional como requerimiento no funcional de la organización. Se diseñaron las nuevas pantallas con la definición de las plantillas de flujo de trabajo y también la pantalla para listar las mismas. En la misma el usuario administrador puede crear, editar si aún no ha sido usada, eliminar, y clonar plantillas.

El flujo simple de competencias consta con campos para agregar nombre, descripción, y periodo donde empieza a ser válido. Se empezó con el desarrollo del flujo para competencias como era el proceso de desarrollo inicial más simple, ya que la información para la creación de competencias en el sistema sin el módulo curricular solo requería de un nombre para la competencia en el nivel en la que se está creando. Además, los flujos simple también pueden ser personalizados, la única diferencia que adquiría es que se agregan campos personalizados para la definición del formulario institucional, para que puedan responder los que proponen la nueva competencia.

La historia de usuario fue desarrollada durante una iteración con un total de 78 horas cargadas en el sistema.

\subsection{Aprobación de pasos completados de flujos de trabajo}

Luego de la historia en la que se diseñaron las plantillas y fue desarrollado un flujo de trabajo simple inicial para creación o versionamiento de competencias, el siguiente paso es que un usuario designado desde la plantilla pueda iniciar y otro pueda aprobar el proceso de creación o revisión de competencias de cualquier nivel.

Esta historia de usuario tenía como descripción: \enquote{\textit{Como evaluador de un flujo de trabajo, me gustaría un simple proceso paso por paso en el que pueda revisar y/o aprobar competencias de manera sencilla e intuitiva}}.

La historia de usuario tiene los siguientes criterios de aceptación:
\begin{itemize}
	\item Soporte de asignaciones de tareas de creación y revisión por roles del AMS en las plantillas de flujos de trabajo.
	\item Diseño e implementación de vista de revisión para el flujo de trabajo.
\end{itemize}

Cada paso del flujo de trabajo debe estar terminado para que pase a la etapa de revisión por parte de los encargados. Luego de enviar el formulario, cada rol debe hacer su revisión para que el sistema pueda agregar la nueva competencia.

Como en las reuniones de demostración de cada sprint se notaban ciertos aspectos de las historias de usuario que no llenaban las expectativas de los clientes, los desarrolladores decidieron diseñar maquetas de pantallas que mostraban el posible diseño de la página. Luego de recibir feedback de parte de los clientes, se empezaba a desarrollar las nuevas pantallas. Finalizando la historia de usuario con pruebas automatizadas.

La historia de usuario fue desarrollada durante una iteración en un periodo de tiempo de 44 horas cargadas en el sistema.

\subsection{Rechazar pasos completados del flujo de trabajo}
Esta historia de usuario tenía como descripción: \enquote{\textit{Como evaluador de flujos de trabajo, me gustaría ser capaz de rechazar partes de los mismos y poder dar feedback a partes que no cumplen con nuestros estándares}}.

\begin{itemize}
	\item Diseño e implementación de funcionalidad de rechazo de pasos en los flujos de trabajo.
	\item Diseño e implementación de funcionalidad de retroalimentación de parte de los evaluadores y encargados.
\end{itemize}

Esta funcionalidad tiene como propósito permitir a la persona que hace la revisión de los pasos rechazar y dejar feedback para que se puedan hacer los cambios correspondientes. Cuando se rechaza un paso, se rechaza el flujo de trabajo, y por lo tanto vuelven a estar activos los campos para que se hagan los cambios correspondientes.

El trabajo se inició la actualización del modelo de base de datos actual, luego de crear las clases correspondientes en el código para su utilización. Luego, se actualizaron las páginas donde el usuario puede aprobar los pasos para que soporte rechazar pasos y poder así dejar algunos comentarios. 

La historia de usuario fue desarrollada durante una iteración en un periodo de tiempo de 53 horas cargadas en el sistema.

\section{Buzón de entrada para evaluadores y colaboradores del flujo de trabajo}
\subsection{Workflow Queue Visibility or Inbox}
La siguiente historia tiene como propósito mostrar a cada usuario la lista de workflows pendientes que requiere de su aporte. Además de adaptar el nuevo buzón de entrada para otros rasgos de la aplicación como son las evaluaciones, los planes de acción y preguntas de parte del usuario a profesores.

La historia tiene como descripción: “Como aprobador de eLumen, me gustaría una vista unificada de los workflows que tengo que revisar – además de mis evaluaciones, planes de acción y mis preguntas a profesores – para que no vaya cazando workflows por la aplicación”

Las tareas de la historia de usuario eran las de crear la página que listen los workflows inicialmente, luego de ese hacer las pruebas correspondientes. Luego de que funcione la lista de workflows, agregar los planes de acción y RFI en la lista a la misma lista y volver a hacer las pruebas de funcionamiento. 

Se estimó con 5 puntos de historia y tuvo una duración de 2 sprints debido a inconvenientes en el camino con un total de 56hs de desarrollo.

\subsection{Notificaciones con soporte a etapas}
La historia de usuario tiene como descripción lo siguiente \enquote{Como presidente curricular, me gustaría que el equipo de diseño y revisión curricular reciban notificaciones cuando tengan alertas de deuda de trabajo (y alertas cuando pase el tiempo), para que se puedan manejar mejor de esa manera los procesos curriculares.}.

Como criterios de aceptación se encuentran los siguientes:
\begin{itemize}
	\item Establecer notificaciones cuando las partes del flujo de trabajo son asignadas a los roles de las personas.
	\item Establecer notificaciones de alerta a asignaciones de partes y etapas. Por ejemplo, 5 días después de su asignación.
	\item Mandar notificaciones por mail.
\end{itemize}

Algunas de las tareas identificadas en la planificación de las iteraciones eran los siguientes:
\begin{itemize}
	\item Diseñar e implementar nuevos modelos de datos que permitan soportar el uso de roles para creadores y editores de partes.
	\item Actualizar el sistema de notificaciones del AMS.
	\item Diseñar e implementar la página de configuración de notificaciones.
\end{itemize}

La historia fue finalizada en tres iteraciones con una cantidad de 60 horas cargadas en el sistema.
\section{Versionamiento encadenado de evaluaciones debido al versionamiento de competencias}
\subsection{Versioning for Assessments}
La historia de versionamiento de evaluaciones tiene como propósito permitir el versionamiento automático de evaluaciones existentes que utilizar competencias del sistema. Por ejemplo, en caso de que una evaluación hecha por un profesor tenga una nueva versión en el nuevo periodo de su sección, el sistema versiona la evaluación para ese periodo obteniendo las competencias actuales.

Esta historia de usuario se inició en el sprint 48 con un total de 13 puntos de historia, se finalizó en el sprint con un total de 87hs cargadas en el JIRA.

Esta historia de usuario tenía como descripción los siguiente “Como usuario de eLumen, me gustaría ser capaz de versionar mis evaluaciones, para que se observen los cambios a través del tiempo (y que la interfaz y los reportes sigan presentando datos para los diseños históricos)”.

Algunas de las tareas de la historia fueron las siguientes:
\begin{itemize}
	\item Adaptar versionamiento para el modelo de datos de las evaluaciones.
	\item Actualizar la biblioteca de evaluaciones de los usuarios para que soporte versionamiento de las mismas.
	\item Actualizar el selector de evaluaciones de los profesores, que puedan seleccionar evaluaciones actuales.
	\item Actualizar el widget de profesores que utilizan las evaluaciones como datos.
\end{itemize}

\section{Flujo de trabajo para el versionamiento de cursos}
\subsection{Course Versioning}
Esta historia de usuario se inició en el sprint 47, se estimó terminar en un sprint, pero debido a la cantidad de partes que suponía cambios se utilizaron dos sprints para terminar la historia. Se estimó que la historia era de unos 13 puntos, se finalizó en el sprint 49 con un total de 96hs cargadas en el JIRA.

Esta historia de usuario tenía como descripción los siguiente “Como coordinador de eLumen, me gustaría poder hacer una versión de mi plan de curso para que pueda realizar un seguimiento de los cambios para cosas como la revisión de programas y los acuerdos de articulación y transferencia en eLumen.”.

Algunas de las tareas realizadas en la historia fueron las siguientes:

\begin{itemize}
	\item Buscar técnicas y herramientas de versionamiento para verificar posibles implementaciones parecidas para implementar.
	\item Luego de buscar algunas técnicas de versionamiento, diseñar una posible solución a la problemática.
	\item Implementar cambios en la base de datos mediante scripts en el proyecto.
	\item Actualizar clases de Java existentes en el proyecto de cursos.
	\item Implementar la solución para el flujo de creación de cursos de Curriculum, conocida como Workflow en inglés. Además, incluir nuevas clases de Java para las mismas.
	\item Actualizar la creación de curso con los cambios aplicados mediante scripts de base de datos.
	\item Adaptar la relación de cursos y competencias para que soporte el versionamiento de los mismos.
	\item Actualizar la lista de competencias por cursos.
\end{itemize}

\subsection{Course Detail: Course Cover Info}
Esta historia de usuario tiene como propósito de diseñar páginas que permitan al usuario completar la información básica de curso que buscan diseñar.

Tiene la siguiente descripción: “Como miembro del comité de Curriculum, me gustaría ser capaz de administrar la página de información básica de cursos, para que no tenga que buscar por documentos a la hora de crear o versionar cursos”.

\begin{itemize}
	\item Diseñar un modelo de datos que soporte el nuevo paso de información de curso.
	\item Adaptar tablas existentes y crear clases nuevas para las nuevas entidades de base de datos.
	\item Actualizar la plantilla de creación de Workflow para que soporte el nuevo paso.
	\item Actualizar el visualizador de Workflow.
	\item Diseño de pruebas automatizadas.
\end{itemize}

La historia fue estimada con 5 puntos de historia y se terminó en un sprint con un total de 53hs de desarrollo.

\subsection{Course Details: Units and Hours}
En esta historia se desarrolló un nuevo paso para el desarrollo de Workflow, en la cual el que inició el mismo va a poder detallar las horas y unidades que requiere el curso o cree que se va a requerir.

La organización proveyó de algunas muestras de cómo debería ser la página y era un criterio de aceptación de parte del ticket que siga el modelo de la misma.

La historia tenía la siguiente descripción: “Como aprobador de Curriculum de eLumen, me gustaría tener una página de horas y métricas para que pueda conseguir información básica sobre mi curso en eLumen”.

La historia a desarrollar se dividió entre miembros del equipo de desarrollo en las siguientes tareas:
\begin{itemize}
	\item Crear scripts en la base de datos para adaptar el modelo de datos para que soporte los nuevos campos de curso.
	\item Crear y/o editar las clases de las entidades de Java que utiliza o utilizará la aplicación.
	\item Actualizar la plantilla de workflows.
	\item Actualizar el visualizador de workflows.
	\item Pruebas de funcionalidad.
\end{itemize}

La historia fue estimada con 3 puntos de historia y se terminó en un sprint con un total de 66hs de desarrollo.

\subsection{Curriculum: Course Specifications}
En esta historia de usuario se desarrolló un nuevo paso para el Workflow, el cual era un paso de tipo formulario en la cual el que inició el Workflow puede agregar objetivos, información acerca de los métodos de evaluación de la materia, algunos equipos requeridos y libros que se necesitará en el curso.

La organización proporcionó de modelos de pantallas para el nuevo paso y era un criterio de aceptación de parte del ticket que siga el mismo formato.

La historia de usuario proporciona la siguiente descripción: “Como especialista de Curriculum de eLumen, me gustaría ser capaz de agregar o editar especificaciones de curso como parte del Workflow de creación y/o versionamiento de mi curso, con el objetivo de no hacerlo en papel.”

Las tareas fueron separadas y desarrolladas por los desarrolladores y eran las siguientes:
\begin{itemize}
	\item Crear scripts de base de datos para que soporte el nuevo formato de cursos.
	\item Crear y/o editar clases de Java para las nuevas entidades de la base de datos.
	\item Actualizar la página de creación y/o edición de plantillas de Workflow para que soporte el nuevo paso.
	\item Actualizar el visualizador de Workflow.
	\item Actualizar los servicios de guardado para creación y versionamiento de cursos y workflows.
	\item Actualizar el servicio de aprobación de Workflow.
	\item Test de la historia.
\end{itemize}

La historia fue terminada en el sprint 50 con 5 puntos de historia de usuario y 40hs de desarrollo cargadas en el sistema

\subsection{Prereq \& Entrance Skills}
Esta historia tiene como objetivo que el usuario pueda colocar una lista de cursos como pre-requisitos, co-requisitos, anti-requisitos y recomendaciones para su nuevo curso. Además de ciertas capacidades que el alumno debe tener como requisito para tomar el curso.

Para entrar un poco en contexto de la historia vamos a definir cuáles son los tipos de requisitos que puede tener un curso:

\begin{itemize}
	\item \textbf{Pre-requisito:} es un tipo de requisito que impide al usuario tomar o cursar un curso sin haber pasado antes del curso que está como pre-requisito.
	\item \textbf{Co-requisito:} es un tipo de requisito impide al usuario tomar o cursar un curso si no toma también el curso que tiene como co-requisito.
	\item \textbf{Anti-requisito:} es un requisito impide al usuario tomar o cursar un curso si ya curso o va a cursar un curso que tiene como anti-requisito.
	\item \textbf{Recomendación:} es una recomendación de parte del sistema que curso tomar para aprovechar mejor la materia o curso. Es opcional.
\end{itemize}

La historia de usuario tiene como descripción: “Como especialista de Curriculum de eLumen, me gustaría ser capaz de introducir requisitos para cursos y capacidades de entrada en la creación o revisión de Workflow, para que podamos seguir durante su desarrollo y aprobación”.

La historia fue dividida en partes para que los desarrolladores puedan trabajar en partes independientes durante el proceso de la misma, y eran las siguientes:
\begin{itemize}
	\item Crear scripts para el nuevo modelo de datos.
	\item Generar o editar clases de entidades para el nuevo modelo de datos.
	\item Actualizar la plantilla de Workflow para que soporte un nuevo paso.
	\item Actualizar el visualizador de Workflow.
	\item Actualizar los servicios de guardado y aprobación.
	\item Pruebas de funcionalidad.
\end{itemize}

La historia de usuario fue cerrada en el sprint 50 con 3 puntos de historia y 44hs de desarrollo cargadas.

\subsection{Course Creation/Approval Review}
La historia de usuario tenía como criterio de aceptación los siguientes puntos:
\begin{itemize}
	\item Las páginas para revisar los Workflows tienen una región de retroalimentación o feedback debajo de cada paso, con la opción de ocultar y mostrar, para que el usuario que está revisando el Workflow en desarrollo pueda dejar comentarios al encargado de la creación o versionamiento del curso.
	\item La UI tiene elementos de status que indican que cierta parte es “new” o nueva, “approved” o aprobada y “rejected” o rechazada.
	\item Los pasos tienen regiones que permiten aceptar o rechazar los campos propuestos por los desarrolladores del curso. Por lo tanto, deben tener elementos de UI que indiquen al usuario que puede aprobar o rechazar cada parte.
\end{itemize}
Además de los criterios de aceptación, había que volver a actualizar el buzón de entrada para que acepten los cambios que tiene la historia de usuario. Debido a que más de una persona puede revisar el Workflow y podría trancar el proceso si es que no se le notifica debidamente que hay nuevos cambios que revisar.

Algunas de las tareas descompuestas de la historia de usuario son las siguientes:
\begin{itemize}
	\item Actualizar el visualizador de Workflow para que pueda soportar la nueva característica de aprobación o rechazo de cada parte.
	\item Actualizar el buzón de entrada de Cursos.
	\item Pruebas de funcionamiento.
\end{itemize}

La historia se cerró en el sprint 50 con 3 puntos de historia y 68hs cargadas de desarrollo

\subsection{Learning Outcomes}
Esta historia tiene como propósito de crear o versionar competencias para el curso a ser creado o versionado. 

La organización proporcionó de modelos de pantallas para el nuevo paso y era un criterio de aceptación de parte del ticket que siga el mismo formato.

La historia de usuario tiene como descripción: “Como especialista de Curriculum de eLumen, me gustaría ser capaz de articular las competencias de mi nuevo curso”.

Como criterio de aceptación de la historia fue la de agregar el Workflow de competencias en el Workflow de cursos. Algunas de las tareas de la historia fueron:
\begin{itemize}
	\item Modificar la base de datos para que soporte el nuevo modelo de datos de las competencias dentro de Workflow de curso.
	\item Modificar o agregar clases de las entidades que van a ser usadas durante la historia.
	\item Actualizar la plantilla de Workflow para que soporte el nuevo paso para la creación o versionamiento de competencias.
	\item Actualizar el visualizador de Workflow para que soporte el nuevo paso de competencias.
	\item Actualizar los servicios de guardado y de versionamiento de cursos y competencias.
	\item Pruebas de nuevas funcionalidades.
\end{itemize}

La historia fue cerrada en el sprint 51 e iniciada en el sprint 50 con un total de 76hs cargadas en el sistema.

\subsection{Curriculum: Assign Classification Codes}
La siguiente historia tiene como propósito la de asignar Classification Codes a los cursos.

Para entrar en contexto, habría que definir primero que son los Taxonomy of Programs (TOP) o taxonomía de programas en español.

La taxonomía de programas o TOP es un sistema numérico de códigos usados a nivel de Estado para recolectar y reportar información en cursos y programas, en diferentes instituciones educativas [sacado de TOP manual] por todo el Estado. 

TOP ha sido diseñado para agregar información acerca de los programas. Sin embargo, un código TOP debe ser asignado a cada curso del sistema. Aunque el TOP no contiene tantas opciones específicas como lo haría un sistema diseñado para cursos, a cada curso se le debe dar el código TOP que se aproxima a describir el contenido del curso.

Algunos usos a los códigos TOP:
\begin{itemize}
	\item En el inventario de programas aprobados y rechazados, para tener información que tipos de cursos y programas son ofrecidas por el Estado.
	\item En bases de datos de administración de información, para recolectar y reportar información en logros estudiantiles (licenciaturas y certificados) en ciertos programas.
	\item En contabilidad vocacional estudiantil, para reportes de compleción de programas y cursos de ciertos programas vocacionales.
\end{itemize}

La descripción de la historia fue la siguiente: “Como miembro del comité curricular, me gustaría ser capaz de asignar a mis cursos de códigos de clasificación como parte de la aprobación de mis workflows para asegurar que estén correctos, como esto es motivo de rechazo en la oficina del canciller del Estado”.

Algunas de las tareas fueron las siguientes:
\begin{itemize}
	\item Diseño del nuevo modelo y creación de scripts de base de datos, donde se debían generar tablas para cada nueva entidad del modelo de datos ajustado para las taxonomías de programas. Además, cargar todos los datos de códigos de cursos existentes para el estado de California.
	\item Creación de clases Java de las nuevas entidades de la base de datos.
	\item Crear páginas CRUD de las nuevas tablas (disciplina, sub-disciplina, campo).
	\item Diseño e implementación de la nueva página de asignación de códigos de clasificación para los cursos en proceso de diseño.
	\item Hacer servicios para cada una de las nuevas páginas.
	\item Pruebas de funcionalidad.
\end{itemize}

La historia de usuario, con 5 puntos de historia, fue terminada con un total de 80 horas en el sprint 51.
\section{Flujo de trabajo para el versionamiento de programas de estudio}
\begin{table}[H]
\centering
\resizebox{\columnwidth}{!}{%
\begin{tabular}{@{}lllll@{}}
\toprule
Historias de usuario               & HE  & HC  & PH & Sprints \\ \midrule
Información básica del programa    & 54  & 56  & 8  & 1       \\
Competencias de carrera o programa & 48  & 68  & 5  & 1       \\
Bloques de curso                   & 58  & 60  & 5  & 1       \\
Visualizar cambios en los campos   & 180 & 210 & 13 & 4       \\ \bottomrule
\end{tabular}
}
\caption{Historias de usuario para flujo de trabajo para el versionamiento de programas de estudio}
\label{epic:8}
\end{table}

\subsection{Información básica del programa}
La historia de usuario tiene como descripción lo siguiente \enquote{\textit{Como coordinador del departamento o encargado del AMS, me gustaría ser capaz de agregar o revisar programas en el módulo de gestión curricular, para que de esta forma pueda manejar mejor mis registros de la institución en el sistema}}. Y los criterios de aceptación consistían en el desarrollo de la pantalla que se puede apreciar en la figura \ref{program_cover_info}.

Algunas de las tareas identificadas en la planificación de las iteraciones eran los siguientes:
\begin{itemize}
	\item Adaptar la base de datos para soportar los nuevos campos a ser guardados por el flujo de trabajo.
	\item Luego de hacer los cambios en la base de datos, actualizar o agregar nuevas clases de Java para su posterior uso.
	\item Desarrollar la página de información básica del programa.
	\item Actualizar la plantilla de flujos de trabajo institucional para que soporte la creación y revisión de programas.
	\item Diseñar servicios para guardar los registros de la nueva página.
	\item Diseñar servicios de aprobación de flujo de trabajo de programas.
\end{itemize}

La historia fue finalizada en una iteración con una cantidad de 56 horas cargadas en el sistema.

\begin{figure}[H]
\centering
\includegraphics[width=125mm,scale=1]{Capitulos/DesarrollodelaAplicacion/Imagenes/program_cover_info}
\caption{Mockup de la pantalla de información básica del programa.}
  \label{program_cover_info}
\end{figure}

\subsection{Competencias de carrera o programa}
La historia de usuario tiene como descripción lo siguiente \enquote{\textit{Como coordinador, me gustaría ser capaz de administrar las competencias asociadas a mi programas durante el flujo de creación y revisión del mismo, para que pueda hacer una revisión comprensiva de los programas que tiene el AMS}}. Y los criterios de aceptación consistían en el desarrollo de la pantalla que se puede apreciar en la figura \ref{program_learning_outcomes}.

Algunas de las tareas identificadas en la planificación de las iteraciones eran los siguientes:
\begin{itemize}
	\item Adaptar la base de datos para soportar los nuevos campos a ser guardados por el flujo de trabajo.
	\item Luego de hacer los cambios en la base de datos, actualizar o agregar nuevas clases de Java para su posterior uso.
	\item Desarrollar la página de información básica del programa.
	\item Actualizar los servicios de guardado de campos para el flujo de trabajo.
	\item Actualizar los servicios de aprobación de flujo de trabajo de programas.
\end{itemize}

La historia fue finalizada en una iteración con una cantidad de 68 horas cargadas en el sistema.

\begin{figure}[H]
\centering
\includegraphics[width=125mm,scale=1]{Capitulos/DesarrollodelaAplicacion/Imagenes/program_learning_outcomes}
\caption{Mockup de la pantalla de competencias del programa.}
  \label{program_learning_outcomes}
\end{figure}

\subsection{Bloques de cursos}
La historia de usuario tiene como descripción lo siguiente \enquote{\textit{Como coordinador de departamento, me gustaría ser capaz de diseñar bloques de cursos para mis programas, para que de esta manera pueda diseñar la malla para mis programas de estudio}}. Y los criterios de aceptación consistían en el desarrollo de la pantalla que se puede apreciar en la figura \ref{program_course_blocks}.

Algunas de las tareas identificadas en la planificación de las iteraciones eran los siguientes:
\begin{itemize}
	\item Diseño e implementación del modelo de datos.
	\item Diseño e implementación de clases Java.
	\item Desarrollar la página de paso para creación de bloques de cursos en los diferentes flujos de trabajo.
	\item Desarrollar servicios de guardado y aprobación de la funcionalidad.
	\item Actualizar la plantilla de flujos de trabajo.
\end{itemize}

La historia fue finalizada en una iteración con una cantidad de 60 horas cargadas en el sistema.

\begin{figure}[H]
\centering
\includegraphics[width=125mm,scale=1]{Capitulos/DesarrollodelaAplicacion/Imagenes/program_course_blocks}
\caption{Mockup de la pantalla de bloques de cursos de programa.}
  \label{program_course_blocks}
\end{figure}

\subsection{Visualizar cambios en los campos}
La historia de usuario tiene como descripción lo siguiente \enquote{\textit{Como coordinador de departamento, me gustaría ser capaz de diseñar bloques de cursos para mis programas, para que de esta manera pueda diseñar la malla para mis programas de estudio}}. Y los criterios de aceptación consistían en el desarrollo de la pantalla que se puede apreciar en la figura \ref{visualize_changes}.

Como criterios de aceptación se encuentran los siguientes:
\begin{itemize}
	\item Los campos borrados se deben marcar en rojo.
	\item Los nuevos campos se deben marcar en verde.
	\item Se debe visualizar el estado anterior y el nuevo con una forma de identificar con el usuario que hizo la modificación.
	\item Limitado para los cambios del programa.
	\item Diseño de interfaz aprobada por el equipo de validación.
	\item Las diferencias limitada a dos versiones.
\end{itemize}

Algunas de las tareas identificadas en la planificación de las iteraciones eran los siguientes:
\begin{itemize}
	\item Diseño e implementación del modelo de datos.
	\item Diseño e implementación de clases Java.
	\item Desarrollar la página de paso para creación de bloques de cursos en los diferentes flujos de trabajo.
	\item Desarrollar servicios de guardado y aprobación de la funcionalidad.
	\item Actualizar la plantilla de flujos de trabajo.
\end{itemize}

La historia fue finalizada en una iteración con una cantidad de 60 horas cargadas en el sistema.

\begin{figure}[H]
\centering
\includegraphics[width=125mm,scale=1]{Capitulos/DesarrollodelaAplicacion/Imagenes/visualize_changes}
\caption{Mockup de la pantalla de la funcionalidad de visualización de cambios.}
  \label{visualize_changes}
\end{figure}

\section{Soporte de etapas en los flujos de trabajo}
\begin{table}[H]
\centering
\caption{Historias de usuario para soporte de etapas en los flujos de trabajo}
\label{epic:8}
\resizebox{\columnwidth}{!}{%
\begin{tabular}{@{}lllll@{}}
\toprule
Historias de usuario                                             & HE  & HC  & PH & Sprints \\ \midrule
Roles de creación y edición para las partes de flujos de trabajo & 102 & 112 & 13 & 1       \\
Diseño e implementación de etapas                                & 288 & 505 & 21 & 3       \\
Mejora en comportamientos para las etapas por roles              & 84  & 108 & 8  & 2       \\
Composición de etapas y partes                                   & 216 & 391 & 13 & 3       \\
Etapas y partes opcionales en la revisión del flujo              & 64  & 76  & 8  & 1       \\ \bottomrule
\end{tabular}
}
\end{table}

\subsection{Roles de creación y edición para las partes de flujos de trabajo}
La historia de usuario tiene como descripción lo siguiente \enquote{\textit{Como participante en el proceso curricular, podría no solo revisar nuevos cursos y programas, sino que también realizar pequeñas revisiones como parte del proceso o participe también en el paso de compleción del formulario}}.

Como criterios de aceptación se encuentran los siguientes:
\begin{itemize}
	\item Diseñar el privilegio de creador que permitan a un iniciador de flujo o un diseñador de flujo que designe a ciertos roles para escribir o llenar cada paso.
	\item Diseñar el privilegio de editor que permita editar partes a los que revisan el flujo de trabajo antes de aceptar o rechazar.
	\item Configurar flujos de trabajo para que en cada parte o etapa de un flujo de trabajo puedan ser asignados por roles para la tarea de creador o evaluador, o ambos.
\end{itemize}

Algunas de las tareas identificadas en la planificación de las iteraciones eran los siguientes:
\begin{itemize}
	\item Diseñar e implementar nuevos modelos de datos que permitan soportar el uso de roles para creadores y editores de partes.
	\item Diseñar e implementar servicios de visualización para las diferentes secciones de flujos de trabajo.
	\item Actualizar las plantillas de flujo de trabajo para agregar el soporte de roles de creación y edición.
	\item Actualizar el flujo de trabajo para soportar la compleción de secciones por paso.
	\item Implementar la funcionalidad de edición.
\end{itemize}

La historia fue finalizada en tres iteraciones con una cantidad de 112 horas cargadas en el sistema.


\subsection{Diseño e implementación de Etapas}
La historia de usuario tiene como descripción lo siguiente \enquote{\textit{Como administrador curricular quiero ser capaz de configurar mi plantilla de flujo de trabajo para que pueda dividir en etapas donde se especifiquen que roles pueden completar que funciona en una o múltiples secciones o partes de mi programa o curso}}. Y los criterios de aceptación consistían en el desarrollo de la pantalla que se puede apreciar en la figura \ref{visualize_changes}.

Como criterios de aceptación se encuentran los siguientes:
\begin{itemize}
	\item En un diseño de flujo de trabajo se debe elegir un rol, sección o parte y la acción (completar, revisar, aprobar).
	\item Para la primera etapa solo la acción de completar debe estar disponible.
	\item Las acciones de revisar y aprobar deben estar disponibles si una etapa anterior tiene las mismas secciones o partes con la acción de completar.
	\item Para la primera etapa todas las secciones o partes deben ser completadas en orden secuencial.
	\item Cualquier etapa después de la primera puede tener la opción de mostrar comentarios para la persona que completó los datos antes de transicionar a la siguiente etapa.
\end{itemize}

Algunas de las tareas identificadas en la planificación de las iteraciones eran los siguientes:
\begin{itemize}
	\item Analizar las zonas posibles a ser afectadas por la nueva funcionalidad.
	\item Actualizar la plantilla de flujos de trabajo.
	\item Actualizar el flujo de trabajo de cursos.
	\item Actualizar el flujo de trabajo de programas.
	\item Diseñar e implementar un modelo de datos que soporte la nueva funcionalidad.
	\item Actualizar el buzón de entrada.
	\item Actualizar las notificaciones a colaboradores.
	\item Diseñar e implementar migraciones de datos.
\end{itemize}

La historia fue finalizada en tres iteraciones con una cantidad de 505 horas cargadas en el sistema.

\begin{figure}[H]
\centering
\includegraphics[width=125mm,scale=1]{Capitulos/DesarrollodelaAplicacion/Imagenes/workflow_stage}
\caption{Mockup de la pantalla de plantillas soportando las etapas.}
  \label{workflow_stage}
\end{figure}

\begin{figure}[H]
\centering
\includegraphics[width=125mm,scale=1]{Capitulos/DesarrollodelaAplicacion/Imagenes/workflow_template_stage}
\caption{Mockup de la pantalla de plantillas con etapas por flujo.}
  \label{workflow_template_stage}
\end{figure}

\subsection{Mejora en comportamientos para las etapas por roles}
La historia de usuario tiene como descripción lo siguiente \enquote{\textit{Como coordinador de educación a distancia, yo solo necesito revisar el esquema del curso para aquellos cursos diseñados para educación a distancia}}.

El criterio de aceptación de la historia consistía en permitir que una etapa que no tiene roles asignados sea opcional.

Algunas de las tareas identificadas en la planificación de las iteraciones eran los siguientes:
\begin{itemize}
	\item Diseñar un plan de pruebas, en estas se identifican las posibles zonas afectadas por la nueva funcionalidad.
	\item Actualizar la pantalla de creación de flujos de trabajo.
	\item Actualizar el mecanismo de transición de etapas.
	\item Mejorar la UI de la vista de etapas de flujos.
\end{itemize}

La historia fue finalizada en tres iteraciones con una cantidad de 108 horas cargadas en el sistema.

\subsection{Composición de etapas y partes}
La historia de usuario tiene como descripción lo siguiente \enquote{\textit{Como presidente curricular, me gustaría ser capaz de componer flujos de trabajo de cursos o programas y etapas (incluyendo actores en cada etapa y sus acciones), para que de esa manera se pueda modelar el proceso curricular en la aplicación}}.

Como criterios de aceptación se encuentran los siguientes:
\begin{itemize}
	\item Cada etapa puede contener una o más partes.
	\item Los actores de cada etapa pueden asignarse actiones a cada parte, o a todas.
	\item Etapas equivalentes no pueden bloquearse entre sí. Por ejemplo, si Suzy y Joe son evaluadores de tres partes, Joe no tiene que esperar que Suzy revise la parte 1 antes de que el pueda evaluar la parte 2, es decir, ambos pueden evaluar sus partes en simultáneo.
\end{itemize}

Algunas de las tareas identificadas en la planificación de las iteraciones eran los siguientes:
\begin{itemize}
	\item Diseño de mockups para su aprobación previa al desarrollo.
	\item Diseño e implementación del modelo de datos que soporte la nueva funcionalidad.
	\item Actualizar las plantillas de flujos de trabajo.
	\item Actualizar los flujos de trabajo de cursos y programas.
	\item Actualizar el sistema de notificaciones.
\end{itemize}

La historia fue finalizada en tres iteraciones con una cantidad de 391 horas cargadas en el sistema.

\subsection{Etapas y partes opcionales en la revisión del flujo}
La historia de usuario tiene como descripción lo siguiente \enquote{\textit{Como administrador curricular quiero ser capaz de configurar mi plantilla de flujo de trabajo para que pueda dividir en etapas donde se especifiquen que roles pueden completar que funciona en una o múltiples secciones o partes de mi programa o curso}}.

Como criterios de aceptación se encuentran los siguientes:
\begin{itemize}
	\item Los roles pueden ser configurados para la acción de creación o revisar, o ambos.
	\item El rol de creación lleva el nombre de creador y para la revisión lleva el nombre de evaluador.
\end{itemize}

Algunas de las tareas identificadas en la planificación de las iteraciones eran los siguientes:
\begin{itemize}
	\item Diseñar un plan de pruebas, en estas se identifican las posibles zonas afectadas por la nueva funcionalidad.
	\item Actualizar la pantalla de creación de flujos de trabajo.
	\item Actualizar el mecanismo de transición de etapas.
\end{itemize}

La historia fue finalizada en tres iteraciones con una cantidad de 76 horas cargadas en el sistema.
\section{Reportes y notificaciones para flujos de trabajo}
\begin{table}[H]
\centering
\resizebox{\columnwidth}{!}{%
\begin{tabular}{@{}lllll@{}}
\toprule
Historias de usuario                     			& HE & HC & PH & Sprints \\ \midrule
Notificaciones para las partes del flujo de trabajo & 58 & 60 &  5 &  1 \\
Reporte de esquemas de curso 						& 88 & 91 &  8 &  2 \\ \bottomrule
\end{tabular}
}
\caption{Historias de usuario para los reportes y notificaciones de versiones de cursos}
\label{epic:9}
\end{table}

\subsection{Notificaciones para las partes del flujo de trabajo}
La historia de usuario tiene como descripción lo siguiente \enquote{\textit{Como especialista curricular, me gustaría que mi equipo de diseño y revisión de flujos de trabajo reciban las notificaciones cuando tengan trabajos pendientes (y alertas cuando este retrasado), para que pueda manejar mejor mis procesos curriculares}}.

Como criterios de aceptación se encuentran los siguientes:
\begin{itemize}
	\item Establecer notificaciones cuando las partes del flujo de trabajo son asignadas a los roles de las personas.
	\item Establecer notificaciones de alerta a asignaciones de partes y etapas. Por ejemplo, 5 días después de su asignación.
	\item Mandar notificaciones por mail.
\end{itemize}

Algunas de las tareas identificadas en la planificación de las iteraciones eran los siguientes:
\begin{itemize}
	\item Diseñar e implementar nuevos modelos de datos que permitan soportar el uso de roles para creadores y editores de partes.
	\item Actualizar el sistema de notificaciones del AMS.
	\item Diseñar e implementar la página de configuración de notificaciones.
\end{itemize}

La historia fue finalizada en tres iteraciones con una cantidad de 60 horas cargadas en el sistema.

\subsection{Reporte de esquemas de curso}
La historia de usuario tiene como descripción lo siguiente \enquote{\textit{Como especialista curricular, me gustaría ser capaz de hacer reportes de registro de esquemas de curso, para que pueda de esta manera ser compartidas por los diferentes colaboradores}}.

Como criterios de aceptación se encuentran los siguientes:
\begin{itemize}
	\item Reporte diseñado por el equipo de diseño curricular.
	\item Incluir competencias es opcional para el reporte.
	\item Incluir alineación de competencias es opcional para el reporte.
	\item Formatos en DOC, PDF o HTML (no Excel).
	\item Se puede ejecutar desde la lista de cursos o de la lista de reportes.
	\item Cuando se corre desde la lista de reportes se pueden elegir uno o más cursos.
\end{itemize}

Algunas de las tareas identificadas en la planificación de las iteraciones eran los siguientes:
\begin{itemize}
	\item Diseñar e implementar el reporte.
	\item Implementar los métodos que acceden a la base de datos.
	\item Diseñar e implementar la página de generación de reportes.
\end{itemize}

La historia fue finalizada en tres iteraciones con una cantidad de 111 horas cargadas en el sistema.
\section{Soporte de versionamiento en el AMS}
\begin{table}[H]
\centering
\caption{Historias de usuario para soporte de versionamiento en el AMS}
\label{epic:10}
\resizebox{\columnwidth}{!}{%
\begin{tabular}{llllll}
\hline
Historias de usuario                                                           & HE & HC & PH & PA & CS \\ \hline
Interfaz de alineación de códigos TOP/CIP                                      & 32 & 36 &  3 & 3  &  1 \\ 
Renombrar/Reorganizar pestañas para una mejor apariencia del módulo curricular & 32 & 44 &  3 & 3  &  1 \\
Lista curricular mejorada para cursos y programas                              &132 &135 &  8 & 8  &  2 \\ \hline
\end{tabular}
}
\end{table}

\subsection{Interfaz de alineación de códigos TOP/CIP}
La historia de usuario tiene como descripción lo siguiente \enquote{\textit{Como encargado del sistema de gestión de evaluaciones, me gustaría ser capaz de mantener alineados los codigos TOP a los códigos federales CIP y de esta manera no tener que depender de los administradores para manejar o interpretar estos datos}}.

Como criterios de aceptación se encuentran los siguientes:
\begin{itemize}
	\item Diseñar e implementar una página CRUD para códigos TOP/CIP.
	\item Permitir alinear los códigos.
\end{itemize}

Algunas de las tareas identificadas en la planificación de las iteraciones eran los siguientes:
\begin{itemize}
	\item Diseñar mockups para la nueva página.
	\item Diseñar e implementar el modelo de datos que soporte la funcionalidad.
	\item Desarrollar la página y los métodos de guardado.
\end{itemize}

La historia fue finalizada en una iteración con una cantidad de 36 horas cargadas en el sistema.

\subsection{Renombrar/Reorganizar pestañas para una mejor apariencia del módulo curricular}
La historia de usuario tiene como descripción lo siguiente \enquote{\textit{Como presidente del comité curricular, me gustaría ver las pestañas que están enfocadas a mi trabajo para que pueda navegar y manejar mi tiempo en la aplicación}}.

Como criterios de aceptación se encuentran los siguientes:
\begin{itemize}
	\item Crear pestaña curricular.
	\item Identificar y esconder pestañas no relevantes para el rol.
\end{itemize}

Algunas de las tareas identificadas en la planificación de las iteraciones eran los siguientes:
\begin{itemize}
	\item Diseño de pestañas.
	\item Implementación de pestañas y modificaciones de espacio.
\end{itemize}

La historia fue finalizada en una iteración con una cantidad de 44 horas cargadas en el sistema.


\subsection{Lista curricular mejorada para cursos y programas}
La historia de usuario tiene como descripción lo siguiente \enquote{\textit{Como presidente curricular o miembro del plantel de profesores, me gustaría ser capaz de ordenar/filtrar/visualizar mis cursos y programas, para que de esta manera pueda ver que es importante para mi para luego tomar la acción correspondiente desde esta vista y mi trabajo pueda ser realizado de manera eficiente y con buena visibilidad}}.

Como criterios de aceptación se encuentran los siguientes:
\begin{itemize}
	\item Filtrar por estado de flujo de trabajo, fecha de inicio, fecha de fin y otros a ser designados por el equipo de validación.
	\item Ser capaz de tomar acciones desde la lista (mirar reporte de esquema, empezar una revisión, iniciar el flujo, otros).
	\item Columnas configurables para la vista (resultados de curso, semestre inicial, estado de flujo de trabajo, perfomance de competencia, etc.).
\end{itemize}

Algunas de las tareas identificadas en la planificación de las iteraciones eran los siguientes:
\begin{itemize}
	\item Diseñar mockups para la nueva página.
	\item Diseñar e implementar el modelo de datos que soporte la funcionalidad.
	\item Implementar los filtros.
	\item Implementar página para las columnas configurables.
	\item Actualizar lista.
	\item Actualizar el reporte de esquema de cursos para que soporte múltiples cursos.
\end{itemize}

La historia fue finalizada en dos iteraciones con una cantidad de 135 horas cargadas en el sistema.
\section{Retoques finales}
\begin{table}[H]
\centering
\resizebox{\columnwidth}{!}{%
\begin{tabular}{@{}lllll@{}}
Historias de usuario                                                           & HE & HC & PH & Sprints \\ \hline
Retoques finales para el flujo de trabajo de cursos                            & 86 & 162&  8 &  1 \\ \hline
\end{tabular}
}
\caption{Historias de usuario para los retoques finales}
\label{epic:11}
\end{table}

\subsection{Retoques finales para el flujo de trabajo de cursos}
La historia de usuario tiene como descripción lo siguiente \enquote{\textit{Como especialista curricular, me gustaría poder visualizar y editar todos los campos requeridos por el PCAH}}.

Como criterios de aceptación se encuentran los siguientes:
\begin{itemize}
	\item El flujo de trabajo soporta todos los campos del PCAH.
	\item El flujo de trabajo soporta los campos almacenados adicionales.
\end{itemize}

Algunas de las tareas identificadas en la planificación de las iteraciones eran los siguientes:
\begin{itemize}
	\item Diseñar mockups para todas las partes del flujo de trabajo.
	\item Actualizar las pantallas de las partes del flujo de trabajo.
	\item Actualizar los procedimientos de guardado y versionamiento de entidades.
\end{itemize}

La historia fue finalizada en tres iteraciones con una cantidad de 162 horas cargadas en el sistema.

% VALIDACION
% Chapter 7

\chapter{Validación del desarrollo} % Main chapter title

\label{capitulo7} % Change X to a consecutive number; for referencing this chapter elsewhere, use \ref{capitulo7}

% Evaluación comparativa de los modelos

% Analisis de Forcasting en base al mejor modelo

% Documentación de los resultados obtenidos: Una vez obtenidos los resultados se
% procederá a documentarlos en forma clara y detallada.

\section{Validación por parte del equipo educativo}

\section{Validación por parte de los usuarios}

% CONCLUSION
% Chapter 8

\chapter{Conclusión y aportes} \label{capitulo7} 

\section{Conclusiones generales}
En el proyecto final se tomó como caso de estudio con enfoque en la interacción humano-computador y con observación participante el diseño e implementación de un módulo de gestión curricular para un sistema de gestión de evaluaciones basadas en competencias académicas, la cual tiene sus cimientos en el mercado y también dispone de clientes utilizando la misma. 

Se diseñó la manera de integrar y estructurar procesos separados de validación de competencias, cursos, y programas para el estado de California en un sistema de gestión de evaluaciones basadas en competencias. Dicho proceso era un proceso que se hacía en papel y tenía sus falencias debido a que el proceso requería mucho tiempo en revisar y aprobar el formulario como se muestra en la figura \ref{course_creation_flow}, y la complejidad del flujo aumentaba cuando habían más personas colaboradoras o evaluadoras en el proceso.

Debido a que los requerimientos eran cambiantes y el equipo que diseñaba no disponía de un panorama completo de las funcionalidades del módulo curricular, la elección de la metodología ágil para el desarrollo del proyecto fue acertada debido a que la misma permitía el desarrollo iterativo e incremental del software con validaciones del cliente como proceso de desarrollo, que en este caso el equipo de validación tomaba el rol de cliente debido al conocimiento y experiencia en didáctica de sus miembros.

Por lo tanto, algunas de las conclusiones al utilizar la metodología ágil son las siguientes:
\begin{itemize}
	\item Posee el potencial para mejorar rápidamente la elección de que construir, pero sólo si se tiene en cuenta las señales del mercado que sugieren el cambio, y sólo si luego se toman medidas de cambios.
	\item Posee el potencial para entregar de manera más eficiente, pero sólo si se invierte tiempo en modular en unidades de trabajo más pequeñas. 
	\item Posee el potencial para proporcionar mejores predicciones, pero sólo si usted invierte en habilidades de estimación, y sólo si se hace una distinción entre la predicción y el compromiso.
	\item Posee el potencial para ofrecer productos de mayor calidad, pero sólo si se recopilan los comentarios de los clientes sobre el enfoque de la solución y sólo si se realizan los cambios apropiados.
\end{itemize}

Al diseñar los flujos de trabajo para el diseño y revisión de formularios de competencias, cursos, y programas permitió a los profesores encargados de los mismos y evaluadores de dichos formularios, seguir el proceso de una manera intuitiva buscando la mejor experiencia de usuario y con menos cuellos de botella. Como ayuda, que cada paso genera un mensaje que el usuario puede acceder en su buzón de entrada en caso de que tuviera trabajo pendiente.

En el desarrollo del módulo se utilizaron muchas de las tecnologías y herramientas que disponía el sistema como requerimiento no funcional de parte de la organización. El uso de estas tablas en común para la funcionalidad de plantillas de flujos de trabajo fue una decisión errónea, debido a que agregaba complejidad a las mismas y además las pruebas de componentes se convertían en pruebas de regresión debido a la complejidad de la estructura. 

Sin embargo, al utilizar MySQL para guardar el flujo de trabajo y todos sus datos temporales no fue la mejor decisión debido a que la funcionalidad y el estándar tienen cambios constantes en cuanto a datos que tendrían que guardarse, y migrar los datos y columnas de los usuarios aumenta siempre la complejidad de la historia de usuario.

Debido a la capa adicional de comunicaciones con el usuario final personificada por el equipo de Estados Unidos (que en nuestro caso actúa como cliente), la realimentación de valor real o valor aún necesario provisto al usuario final es lenta e implica grandes cambios luego de varios sprints.

A pesar de todas las falencias de desarrollo, el módulo tuvo resultados positivos por parte de los usuarios finales ya que es una herramienta que automatiza trabajos de validación curricular para las instituciones. Además, al tener comentarios acerca de qué habría que mejorar en la aplicación y con la utilización de la metodología ágil se permitió que se creen nuevas historias de usuario para algunos retoques futuros en el módulo curricular. 

\section{Aportes del proyecto}
Se puede afirmar que la llegada del módulo curricular representa la tecnología al servicio de la educación, en específico para automatizar el proceso de diseño curricular. 

Una buena parte de su importancia radica en las amplias posibilidades que ofrece, entre las cuáles las más sobresalientes son la capacidad otorgada por este sistema para gestionar el diseño y validación de material curricular en el estado de California, y la comunicación que ofrece entre los encargados del diseño de los formularios y los evaluadores de los mismos.

Además, podemos citar otros aportes:
\begin{itemize}
	\item El módulo desarrollado reemplaza la preparación manual de formularios, los procesos de revisión y aprobación curricular de colegios ya sea para universidades, cursos, y programas.
	\item Al implementar el módulo en un ambiente que puede ser accedido desde cualquier navegador y la capacidad de compartir comentarios, permite a las personas trabajar en conjunto sin necesidad de agendar reuniones para desarrollar el formulario o las revisiones.
	\item Al permitir almacenar los datos nuevos, históricos, propuestos y activos de competencias, cursos, y programas.
	\item Al proveer notificaciones automatizadas cuando hay cambios de estado en los flujos de trabajo.
	\item Se redujo el tiempo promedio de formulación y revisión de cada flujo de diseño e implementación de flujos de trabajo, debido a que se facilitó llenar formularios utilizando información ya existente en el AMS.
\end{itemize}


\section{Proyectos futuros}
En consenso con los usuarios finales de la aplicación y los que diseñan las historias de usuarios se observaron ciertas características que podrían dar una mayor utilidad al proyecto, donde citaremos algunos de los trabajos futuros ya creadas como épicas del proyecto:
\begin{itemize}
	\item \textbf{Importador de cursos:} muchos de los cursos que ya fueron agregados al sistema de gestión curricular del estado de California existen y como trabajo futuro para el módulo curricular es la forma de importar todos estos cursos al AMS sin necesidad de hacer todo el flujo de trabajo para las competencias, cursos y programas válidos actualmente.
	\item \textbf{Migración de motor de base de datos de los flujos de trabajo:} como ya comentamos, la decisión de tecnología en cuanto al motor de base de datos fue una decisión errónea, debido a que el estándar de California para diseño y revisión de cursos y programas puede variar con el tiempo y la utilización de MySQL como motor de base de datos dificulta el desarrollo en sí debido al constante cambio del modelo de datos, por lo que se considera un motor de base de datos no relacional basada en documentos, e.g. \enquote{MongoDB} como una opción a futuro.
	\item \textbf{Acceso de información a través de API\footnote{de sus siglas en inglés, Application Programming Interface, que sirve como un conjunto de reglas para que las aplicaciones puedan comunicarse entre ellas.} pública:} es una práctica que exige el estado de California que todos los datos en cuanto a cursos y programas de las universidades puedan ser accedidas desde una API pública. Por lo tanto, se debe diseñar e implementar una interfaz pública que permita comunicarse y acceder a los datos del módulo curricular.
	\item \textbf{Catálogo de cursos:} los CMS del estado tienen la opción de mostrar sus cursos válidos de manera pública para que las universidades puedan cargar en sus sistemas académicos como se aprecia en la figura \ref{after_creation}. Por lo tanto, como trabajo futuro para el módulo curricular se busca catalogar los cursos de manera pública desde una interfaz web.
	\item \textbf{Flujo de trabajo para evaluaciones:} el mismo flujo de trabajo que se diseño e implementó para las competencias, cursos, y programas, se debe implementar para las evaluaciones.
\end{itemize}

%----------------------------------------------------------------------------------------
%	THESIS CONTENT - APPENDICES
%----------------------------------------------------------------------------------------

%\appendix % Cue to tell LaTeX that the following "chapters" are Appendices

% Include the appendices of the thesis as separate files from the Appendices folder

%\include{Apendices/AppendixA}
%\include{Apendices/AppendixB}
%\include{Apendices/AppendixC}

%----------------------------------------------------------------------------------------
%	BIBLIOGRAPHY
%----------------------------------------------------------------------------------------

\renewcommand{\bibname}{Bibliografía}

\printbibliography[heading=bibintoc]

%----------------------------------------------------------------------------------------

\end{document}

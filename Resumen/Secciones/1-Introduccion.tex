\section{Introducción}

\subsection{Justificación}
Existe un consenso, tanto entre los empleadores como los funcionarios de gobierno, de que en los Estados Unidos se necesita un mayor número de estudiantes en nivel técnico y universitario que posean habilidades comprobadas \citep{kuh_knowing_2014}. Esto ha llevado a una gran presión sobre las instituciones educativas para que provean más y mejor evidencia de los logros del proceso educativo \citep{kuh_knowing_2014}. Para intentar llenar este vacío se ha creado la evaluación basada en competencias, la cual se enfoca en valorar las habilidades adquiridas por un estudiante en el proceso de un programa educativo \citep{cartwright2009student}.
 
De esta manera, las aplicaciones de evaluación académica basadas en competencias han adquirido mucha importancia en los últimos años \citep{barrio_minton_evaluating_2016}. En dichas aplicaciones se busca conocer información acerca de las fortalezas y debilidades del estudiante de una manera modular, en contraposición a los métodos cuantitativos de evaluación. Dichos aprendizajes y competencias son expresados por segmentos de estudios o actividades, mediante resultados esperados medibles a nivel institucional, de programa, grado, o de curso; expresados en calificaciones asociadas a habilidades específicas y no a módulos o cursos de un programa de estudio \citep{kuh_using_2015}. 
 
Para poder realizar una evaluación basada en competencias de un programa educativo, el mismo programa tiene que estar diseñado con esta perspectiva educativa: los programas de estudio e incluso los mismos cursos que lo conforman, tienen que estar diseñados con una perspectiva de orientación a competencias \citep{lalor_ensuring_2017}. La creación de un programa de estudio es una tarea colaborativa que involucra a un gran número de personas que proponen el diseño curricular, los cursos, los resultados esperados del programa. Cada uno de estos elementos debe pasar por un proceso iterativo de creación, revisión y aprobación que involucra a varios funcionarios, autoridades y profesores de las instituciones involucradas \citep{boyle_curriculum_2016}. 
 
En el ámbito de las aplicaciones académicas, si bien existen aplicaciones que abarcan el diseño y la publicación de planes de estudio por parte de los profesores o encargados de las universidades, además de su revisión y posterior aprobación por el comité curricular, no se ha encontrado durante el proceso de relevamiento de herramientas existentes una aplicación que integre todos estos elementos en un sistema de gestión de evaluaciones basadas en competencias \citep{curricunet_webpage}\citep{courseleaf_webpage}\citep{deca_webpage}.
 
En este caso de estudio \citep{runeson2012case} planteamos la posibilidad de crear un sistema de gestión de programas de estudio orientado a resultados pedagógicos y exploramos los desafíos tanto técnicos como de interacción humano computador asociados a dicho desarrollo.

\subsection{Objetivos}
\subsubsection{Objetivo general}
Diseñar una aplicación que permita integrar y estructurar tareas separadas de un sistema académico para una gestión de programas educativos orientados a resultados pedagógicos, que provea soporte a flujos de trabajo para sus diferentes etapas de aprobación.

\subsubsection{Objetivos específicos}
  \begin{itemize}
    \item Realizar un relevamiento de los requerimientos y comparación de las herramientas existentes que aborden la problemática planteada.
    \item Realizar el proyecto en un marco de programación ágil, estimando y desarrollando la aplicación de acuerdo a las directrices brindadas por esta metodología.
    \item Validar la herramienta desarrollada con expertos del dominio principalmente y de manera preliminar con experiencias limitadas con los usuarios finales.
  \end{itemize}

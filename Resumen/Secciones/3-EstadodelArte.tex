\section{Estado del arte}
Una investigación para comprobar otros proyectos o productos con las mismas características propuestas, buscando innovación para el mercado es expuesta en la tabla \ref{relacion-sistemas}.

\begin{table}[H]
\centering
\begin{tabular}{lllccl}
\toprule
\multicolumn{3}{l}{Características}                                                & CurricUNET                       & CourseLeaf            & DECA         \\
\midrule
\multicolumn{3}{l}{Creación y versionamiento de competencias.}                     &                                  &                       &              \\
\multicolumn{3}{l}{Creación y versionamiento de cursos.}                           & $\checkmark$                     & $\checkmark$          & $\checkmark$ \\
\multicolumn{3}{l}{Creación y versionamiento de programas de estudio.}             & $\checkmark$                     & $\checkmark$          &              \\
\multicolumn{3}{l}{Cumple los Estándares de códigos de California.} 			   & $\checkmark$                     &                       &              \\
\multicolumn{3}{l}{Historial de versiones de competencias.}     			       & 			                      & 		              &  			 \\
\multicolumn{3}{l}{Historial de versiones de cursos.}     			               & $\checkmark$                     & $\checkmark$          & $\checkmark$ \\
\multicolumn{3}{l}{Historial de versiones de programas de estudio.}     		   & $\checkmark$                     &  			          & 			 \\
\multicolumn{3}{l}{Reporte de Comparación entre versiones de cursos.}              & $\checkmark$                     &                       & $\checkmark$ \\
\multicolumn{3}{l}{Soporta competencias de aprendizaje del estudiante.}            &                      			  &                       &              \\
\multicolumn{3}{l}{Plantilla de flujo de trabajo customizable.}                    & $\checkmark$                     &                       &              \\
\multicolumn{3}{l}{Permite asignar roles evaluadores en la aplicación.}            & $\checkmark$                     & $\checkmark$          &              \\
\multicolumn{3}{l}{Permite asignar usuarios como colaboradores.}                   & $\checkmark$                     &                       &              \\
\multicolumn{3}{l}{Sistema de alertas para colaboradores y evaluadores.}           & $\checkmark$                     & $\checkmark$          &              \\
\multicolumn{3}{l}{Buzón de entrada para colaboradores y autoridades.} 			   & $\checkmark$                     &                       & $\checkmark$ \\
\multicolumn{3}{l}{Soporte de correlatividades entre cursos.}                      & $\checkmark$ 					  &						  &              \\
\multicolumn{3}{l}{Incluye un catálogo de cursos.}                   		   	   & $\checkmark$					  &	$\checkmark$		  & $\checkmark$ \\
\multicolumn{3}{l}{Incluye un catálogo de programas de estudio.}                   & $\checkmark$					  &	            		  &              \\
\multicolumn{3}{l}{Incluye un catálogo de competencias.}                   	       & 								  &						  & 			 \\
\multicolumn{3}{l}{UX intuitiva y efectiva.}     			   					   &                                  & $\checkmark$          & $\checkmark$ \\
\bottomrule
\end{tabular}
\caption{Relación entre sistemas de gestión curricular.}
\label{relacion-sistemas}
\end{table}

En la actualidad existen aplicaciones web que son capaces de generar solicitudes de creación de planes de estudio mediante formularios. Entre ellas se encuentra CurricUNET, que es la aplicación que abarca gran parte de las necesidades de las universidades de California pero no cuenta con una manera de gestionar competencias o capacidad de integrarse a un AMS basado en competencias.

El análisis hecho por las herramientas fue una investigación de campo y entrevistas con los usuarios.

\subsection{CurricUNET}
CurricUNET es una aplicación web diseñada para automatizar la emisión y aprobación del plan de estudios emitido por profesores y/o encargados de universidades norteamericanas; que incluyen programas, cursos y competencias \citep{curricunet_webpage}.

En la misma se desarrollan propuestas de cursos y programas de estudio mediante formularios de la aplicación, con el objetivo de reemplazar solicitudes en papel que universidades utilizaban para emitir propuestas. Además, ofrece almacenamiento e información de plan de estudios históricos, activos y propuestos.

Todas las entradas, revisiones y reportes son accedidas por la web desde los navegadores. Posee un sistema de notificaciones integrado que permite al usuario un mejor seguimiento del progreso de las propuestas y cursos en revisión. Además, dispone de un control de versionamiento de cursos, planes de estudio y competencias.

Los usuarios del sistema pueden acceder a los reportes e historial de versiones de sus cursos y planes de estudio por lo que ayuda a una mejora continua del programa universitario.

\subsection{CourseLeaf}
El módulo de Curriculum de CourseLeaf es una solución de gestión basada en la web, mejorando los procesos de profesores y del comité de Curriculum de al menos 70 instituciones \citep{courseleaf_webpage}.

Cuando el módulo de Curriculum de CourseLeaf se combina con su módulo de catálogo, colegios y universidades son capaces de gestionar y realizar un seguimiento de la información del programa, desde la propuesta hasta publicar en una aplicación integrada con facilidad.

El software proporciona la generación de flujo de trabajo automático, notificaciones automáticas. Además, identifica todos los cursos, programas y departamentos que se ven afectados por los cambios propuestos en el inicio del proceso de propuesta y puede ayudar en la actualización con los cambios completados. El módulo de Curriculum de CourseLeaf se puede implementar con o sin su catálogo de Cursos.

\subsection{DECA: Curriculum Navigator}
DECA ofrece, mediante su módulo de Curriculum Navigator, una solución de desarrollo y administración de Curriculum \citep{deca_webpage}.

Cuando se utiliza como módulo integrado de su Catálogo, denominado como Catalog Navigator, proporciona una solución que les permite iniciar el camino de aprobación de alguna carrera de grado. Asesores y administradores, por otra parte, utilizan tanto Curriculum Navigator y Catalog Navigator para desarrollar y proporcionar datos públicamente del plan de estudios para sus estudiantes actuales y futuros.

\subsection{Relevancia del módulo curricular}
Frecuentemente, descrito como complejo e ineficaz, los métodos tradicionales basados en papel de gestión y diseño de programas de estudio proporcionan una visibilidad limitada, lo que resulta en una visión restringida de las etapas implicadas en la creación, modificación y aprobación de planes de estudio. Por lo tanto, se buscó eliminar esta complejidad al acelerar el desarrollo curricular y el proceso de aprobación.

En el flujo actual, una vez finalizado el proceso de diseño y revisión curricular de parte de las oficinas, se procede a publicar la nueva competencia, curso, o programa para que cada universidad tenga la información necesaria para ir cargando la misma en sus correspondientes sistemas. 

En el caso de las universidades comunitarias del estado de California, se utilizan los AMS para gestionar y evaluar las competencias de sus estudiantes. El proceso de registro de las nuevas entidades en los AMS es individual; eso quiere decir que un encargado del AMS tiene como trabajo el de cargar uno por uno las nuevas entidades aprobadas y publicadas por el comité curricular.

La importancia del módulo reside en la automatización de formularios y procesos que requieren la participación de personas ajenas al flujo de trabajo, para iniciar y validar propuestas de creación o revisión de cursos y programas. Luego, una vez que ha sido aprobada por el comité curricular se procede a la creación de las entidades de manera automática en el AMS.

Actualmente, hay alternativas que buscan solucionar la misma problemática pero no se ha encontrado durante el proceso de relevamiento de herramientas existentes una plataforma que pueda soportar el uso de competencias ni que pueda comunicarse con los sistemas de gestión de evaluaciones basadas en competencias.